\documentclass{article}\usepackage{graphicx, color}
%% maxwidth is the original width if it is less than linewidth
%% otherwise use linewidth (to make sure the graphics do not exceed the margin)
\makeatletter
\def\maxwidth{ %
  \ifdim\Gin@nat@width>\linewidth
    \linewidth
  \else
    \Gin@nat@width
  \fi
}
\makeatother

\IfFileExists{upquote.sty}{\usepackage{upquote}}{}
\definecolor{fgcolor}{rgb}{0.2, 0.2, 0.2}
\newcommand{\hlnumber}[1]{\textcolor[rgb]{0,0,0}{#1}}%
\newcommand{\hlfunctioncall}[1]{\textcolor[rgb]{0.501960784313725,0,0.329411764705882}{\textbf{#1}}}%
\newcommand{\hlstring}[1]{\textcolor[rgb]{0.6,0.6,1}{#1}}%
\newcommand{\hlkeyword}[1]{\textcolor[rgb]{0,0,0}{\textbf{#1}}}%
\newcommand{\hlargument}[1]{\textcolor[rgb]{0.690196078431373,0.250980392156863,0.0196078431372549}{#1}}%
\newcommand{\hlcomment}[1]{\textcolor[rgb]{0.180392156862745,0.6,0.341176470588235}{#1}}%
\newcommand{\hlroxygencomment}[1]{\textcolor[rgb]{0.43921568627451,0.47843137254902,0.701960784313725}{#1}}%
\newcommand{\hlformalargs}[1]{\textcolor[rgb]{0.690196078431373,0.250980392156863,0.0196078431372549}{#1}}%
\newcommand{\hleqformalargs}[1]{\textcolor[rgb]{0.690196078431373,0.250980392156863,0.0196078431372549}{#1}}%
\newcommand{\hlassignement}[1]{\textcolor[rgb]{0,0,0}{\textbf{#1}}}%
\newcommand{\hlpackage}[1]{\textcolor[rgb]{0.588235294117647,0.709803921568627,0.145098039215686}{#1}}%
\newcommand{\hlslot}[1]{\textit{#1}}%
\newcommand{\hlsymbol}[1]{\textcolor[rgb]{0,0,0}{#1}}%
\newcommand{\hlprompt}[1]{\textcolor[rgb]{0.2,0.2,0.2}{#1}}%

\usepackage{framed}
\makeatletter
\newenvironment{kframe}{%
 \def\at@end@of@kframe{}%
 \ifinner\ifhmode%
  \def\at@end@of@kframe{\end{minipage}}%
  \begin{minipage}{\columnwidth}%
 \fi\fi%
 \def\FrameCommand##1{\hskip\@totalleftmargin \hskip-\fboxsep
 \colorbox{shadecolor}{##1}\hskip-\fboxsep
     % There is no \\@totalrightmargin, so:
     \hskip-\linewidth \hskip-\@totalleftmargin \hskip\columnwidth}%
 \MakeFramed {\advance\hsize-\width
   \@totalleftmargin\z@ \linewidth\hsize
   \@setminipage}}%
 {\par\unskip\endMakeFramed%
 \at@end@of@kframe}
\makeatother

\definecolor{shadecolor}{rgb}{.97, .97, .97}
\definecolor{messagecolor}{rgb}{0, 0, 0}
\definecolor{warningcolor}{rgb}{1, 0, 1}
\definecolor{errorcolor}{rgb}{1, 0, 0}
\newenvironment{knitrout}{}{} % an empty environment to be redefined in TeX

\usepackage{alltt}
% you do not really need this, since png has higher priority to pdf by default
\DeclareGraphicsExtensions{.png,.pdf,.jpeg,.jpg}

\begin{document}

This is an example showing you how to convert PDF figures generated by tikz to PNG via ImageMagick:




The plot in this chunk is converted to PNG:

\begin{knitrout}
\definecolor{shadecolor}{rgb}{0.969, 0.969, 0.969}\color{fgcolor}\begin{kframe}
\begin{alltt}
(x = \hlfunctioncall{rnorm}(20))
\end{alltt}
\begin{verbatim}
##  [1] -0.56048 -0.23018  1.55871  0.07051  0.12929  1.71506
##  [7]  0.46092 -1.26506 -0.68685 -0.44566  1.22408  0.35981
## [13]  0.40077  0.11068 -0.55584  1.78691  0.49785 -1.96662
## [19]  0.70136 -0.47279
\end{verbatim}
\begin{alltt}
\hlfunctioncall{par}(mar = \hlfunctioncall{c}(4.5, 4, 0.1, 0.1))
\hlfunctioncall{hist}(x, main = \hlstring{""}, xlab = \hlstring{"$x$ (how the fonts look like here?)"}, 
    ylab = \hlstring{"$\textbackslash{}\textbackslash{}hat\{f\}(x) = \textbackslash{}\textbackslash{}frac\{1\}\{nh\}\textbackslash{}\textbackslash{}sum_\{i=1\}^n \textbackslash{}\textbackslash{}cdots$"})
\end{alltt}
\end{kframe}
\includegraphics[width=\maxwidth]{figure/047-tikz-png-test-a} 

\end{knitrout}


This chunk uses the PDF device, and it is not converted:

\begin{knitrout}
\definecolor{shadecolor}{rgb}{0.969, 0.969, 0.969}\color{fgcolor}\begin{kframe}
\begin{alltt}
\hlfunctioncall{par}(mar = \hlfunctioncall{c}(4.5, 4, 0.1, 0.1))
\hlfunctioncall{plot}(x, pch = 19)
\end{alltt}
\end{kframe}\includegraphics[width=\maxwidth]{figure/047-tikz-png-test-b} 
\end{knitrout}


\end{document}
