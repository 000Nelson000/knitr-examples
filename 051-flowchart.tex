\documentclass{article}\usepackage[]{graphicx}\usepackage[]{color}
%% maxwidth is the original width if it is less than linewidth
%% otherwise use linewidth (to make sure the graphics do not exceed the margin)
\makeatletter
\def\maxwidth{ %
  \ifdim\Gin@nat@width>\linewidth
    \linewidth
  \else
    \Gin@nat@width
  \fi
}
\makeatother

\definecolor{fgcolor}{rgb}{0.2, 0.2, 0.2}
\newcommand{\hlnumber}[1]{\textcolor[rgb]{0,0,0}{#1}}%
\newcommand{\hlfunctioncall}[1]{\textcolor[rgb]{0.501960784313725,0,0.329411764705882}{\textbf{#1}}}%
\newcommand{\hlstring}[1]{\textcolor[rgb]{0.6,0.6,1}{#1}}%
\newcommand{\hlkeyword}[1]{\textcolor[rgb]{0,0,0}{\textbf{#1}}}%
\newcommand{\hlargument}[1]{\textcolor[rgb]{0.690196078431373,0.250980392156863,0.0196078431372549}{#1}}%
\newcommand{\hlcomment}[1]{\textcolor[rgb]{0.180392156862745,0.6,0.341176470588235}{#1}}%
\newcommand{\hlroxygencomment}[1]{\textcolor[rgb]{0.43921568627451,0.47843137254902,0.701960784313725}{#1}}%
\newcommand{\hlformalargs}[1]{\textcolor[rgb]{0.690196078431373,0.250980392156863,0.0196078431372549}{#1}}%
\newcommand{\hleqformalargs}[1]{\textcolor[rgb]{0.690196078431373,0.250980392156863,0.0196078431372549}{#1}}%
\newcommand{\hlassignement}[1]{\textcolor[rgb]{0,0,0}{\textbf{#1}}}%
\newcommand{\hlpackage}[1]{\textcolor[rgb]{0.588235294117647,0.709803921568627,0.145098039215686}{#1}}%
\newcommand{\hlslot}[1]{\textit{#1}}%
\newcommand{\hlsymbol}[1]{\textcolor[rgb]{0,0,0}{#1}}%
\newcommand{\hlprompt}[1]{\textcolor[rgb]{0.2,0.2,0.2}{#1}}%

\usepackage{framed}
\makeatletter
\newenvironment{kframe}{%
 \def\at@end@of@kframe{}%
 \ifinner\ifhmode%
  \def\at@end@of@kframe{\end{minipage}}%
  \begin{minipage}{\columnwidth}%
 \fi\fi%
 \def\FrameCommand##1{\hskip\@totalleftmargin \hskip-\fboxsep
 \colorbox{shadecolor}{##1}\hskip-\fboxsep
     % There is no \\@totalrightmargin, so:
     \hskip-\linewidth \hskip-\@totalleftmargin \hskip\columnwidth}%
 \MakeFramed {\advance\hsize-\width
   \@totalleftmargin\z@ \linewidth\hsize
   \@setminipage}}%
 {\par\unskip\endMakeFramed%
 \at@end@of@kframe}
\makeatother

\definecolor{shadecolor}{rgb}{.97, .97, .97}
\definecolor{messagecolor}{rgb}{0, 0, 0}
\definecolor{warningcolor}{rgb}{1, 0, 1}
\definecolor{errorcolor}{rgb}{1, 0, 0}
\newenvironment{knitrout}{}{} % an empty environment to be redefined in TeX

\usepackage{alltt}
\IfFileExists{upquote.sty}{\usepackage{upquote}}{}
\begin{document}
\begin{knitrout}
\definecolor{shadecolor}{rgb}{0.969, 0.969, 0.969}\color{fgcolor}\begin{kframe}
\begin{alltt}
\hlfunctioncall{library}(diagram)
\end{alltt}


{\ttfamily\noindent\itshape\color{messagecolor}{\#\# Loading required package: shape}}\begin{alltt}
\hlfunctioncall{read_demo}(\hlstring{"flowchart"}, package = \hlstring{"diagram"}, labels = \hlstring{"demo-flowchart"})
\end{alltt}
\end{kframe}
\end{knitrout}
\begin{knitrout}
\definecolor{shadecolor}{rgb}{0.969, 0.969, 0.969}\color{fgcolor}\begin{kframe}
\begin{alltt}
\hlcomment{## Flowchart examples}
\hlfunctioncall{par}(ask = TRUE)

\hlcomment{## MODELLING DIAGRAM}
mar <- \hlfunctioncall{par}(mar = \hlfunctioncall{c}(1, 1, 1, 1))
\hlfunctioncall{openplotmat}(main = \hlstring{"from Soetaert and herman, book in prep"}, cex.main = 1)
elpos <- \hlfunctioncall{coordinates}(\hlfunctioncall{c}(1, 1, 1, 1, 1, 1, 1, 1), mx = -0.1)
\hlfunctioncall{segmentarrow}(elpos[7, ], elpos[2, ], arr.pos = 0.15, dd = 0.3, arr.side = 3, 
    endhead = TRUE)
\hlfunctioncall{segmentarrow}(elpos[7, ], elpos[3, ], arr.pos = 0.15, dd = 0.3, arr.side = 3, 
    endhead = TRUE)
\hlfunctioncall{segmentarrow}(elpos[7, ], elpos[4, ], arr.pos = 0.15, dd = 0.3, arr.side = 3, 
    endhead = TRUE)

pin <- \hlfunctioncall{par}(\hlstring{"pin"})  # size of plotting region, inches
xx <- 0.2
yy <- xx * pin[1]/pin[2] * 0.15  \hlcomment{# used to make circles round}

sx <- \hlfunctioncall{rep}(xx, 8)
sx[7] <- 0.05

sy <- \hlfunctioncall{rep}(yy, 8)
sy[6] <- yy * 1.5
sy[7] <- sx[7] * pin[1]/pin[2]

\hlfunctioncall{for} (i in \hlfunctioncall{c}(1:7)) \hlfunctioncall{straightarrow}(to = elpos[i + 1, ], from = elpos[i, ], lwd = 2, 
    arr.pos = 0.6, endhead = TRUE)
lab <- \hlfunctioncall{c}(\hlstring{"Problem"}, \hlstring{"Conceptual model"}, \hlstring{"Mathematical model"}, \hlstring{"Parameterisation"}, 
    \hlstring{"Mathematical solution"}, \hlstring{""}, \hlstring{"OK?"}, \hlstring{"Prediction, Analysis"})

\hlfunctioncall{for} (i in \hlfunctioncall{c}(1:5, 8)) \hlfunctioncall{textround}(elpos[i, ], sx[i], sy[i], lab = lab[i])

\hlfunctioncall{textround}(elpos[6, ], xx, yy * 1.5, lab = \hlfunctioncall{c}(\hlstring{"Calibration,sensitivity"}, \hlstring{"Verification,validation"}))
\hlfunctioncall{textdiamond}(elpos[7, ], sx[7], sy[7], lab = lab[7])

\hlfunctioncall{textplain}(\hlfunctioncall{c}(0.7, elpos[2, 2]), yy * 2, lab = \hlfunctioncall{c}(\hlstring{"main components"}, \hlstring{"relationships"}), 
    font = 3, adj = \hlfunctioncall{c}(0, 0.5))
\hlfunctioncall{textplain}(\hlfunctioncall{c}(0.7, elpos[3, 2]), yy, \hlstring{"general theory"}, adj = \hlfunctioncall{c}(0, 0.5), font = 3)
\hlfunctioncall{textplain}(\hlfunctioncall{c}(0.7, elpos[4, 2]), yy * 2, lab = \hlfunctioncall{c}(\hlstring{"literature"}, \hlstring{"measurements"}), 
    font = 3, adj = \hlfunctioncall{c}(0, 0.5))
\hlfunctioncall{textplain}(\hlfunctioncall{c}(0.7, elpos[6, 2]), yy * 2, lab = \hlfunctioncall{c}(\hlstring{"field data"}, \hlstring{"lab measurements"}), 
    font = 3, adj = \hlfunctioncall{c}(0, 0.5))
\end{alltt}
\end{kframe}
\includegraphics[width=\maxwidth]{figure/051-flowchart-demo-flowchart1} 
\begin{kframe}\begin{alltt}

\hlcomment{##### DIAGRAM}

\hlfunctioncall{par}(mar = \hlfunctioncall{c}(1, 1, 1, 1))
\hlfunctioncall{openplotmat}()
elpos <- \hlfunctioncall{coordinates}(\hlfunctioncall{c}(1, 1, 2, 4))
fromto <- \hlfunctioncall{matrix}(ncol = 2, byrow = TRUE, data = \hlfunctioncall{c}(1, 2, 2, 3, 2, 4, 4, 7, 4, 
    8))
nr <- \hlfunctioncall{nrow}(fromto)
arrpos <- \hlfunctioncall{matrix}(ncol = 2, nrow = nr)
\hlfunctioncall{for} (i in 1:nr) arrpos[i, ] <- \hlfunctioncall{straightarrow}(to = elpos[fromto[i, 2], ], from = elpos[fromto[i, 
    1], ], lwd = 2, arr.pos = 0.6, arr.length = 0.5)
\hlfunctioncall{textellipse}(elpos[1, ], 0.1, lab = \hlstring{"start"}, box.col = \hlstring{"green"}, shadow.col = \hlstring{"darkgreen"}, 
    shadow.size = 0.005, cex = 1.5)
\hlfunctioncall{textrect}(elpos[2, ], 0.15, 0.05, lab = \hlstring{"found term?"}, box.col = \hlstring{"blue"}, shadow.col = \hlstring{"darkblue"}, 
    shadow.size = 0.005, cex = 1.5)
\hlfunctioncall{textrect}(elpos[4, ], 0.15, 0.05, lab = \hlstring{"related?"}, box.col = \hlstring{"blue"}, shadow.col = \hlstring{"darkblue"}, 
    shadow.size = 0.005, cex = 1.5)
\hlfunctioncall{textellipse}(elpos[3, ], 0.1, 0.1, lab = \hlfunctioncall{c}(\hlstring{"other"}, \hlstring{"term"}), box.col = \hlstring{"orange"}, 
    shadow.col = \hlstring{"red"}, shadow.size = 0.005, cex = 1.5)
\hlfunctioncall{textellipse}(elpos[3, ], 0.1, 0.1, lab = \hlfunctioncall{c}(\hlstring{"other"}, \hlstring{"term"}), box.col = \hlstring{"orange"}, 
    shadow.col = \hlstring{"red"}, shadow.size = 0.005, cex = 1.5)
\hlfunctioncall{textellipse}(elpos[7, ], 0.1, 0.1, lab = \hlfunctioncall{c}(\hlstring{"make"}, \hlstring{"a link"}), box.col = \hlstring{"orange"}, 
    shadow.col = \hlstring{"red"}, shadow.size = 0.005, cex = 1.5)
\hlfunctioncall{textellipse}(elpos[8, ], 0.1, 0.1, lab = \hlfunctioncall{c}(\hlstring{"new"}, \hlstring{"article"}), box.col = \hlstring{"orange"}, 
    shadow.col = \hlstring{"red"}, shadow.size = 0.005, cex = 1.5)

dd <- \hlfunctioncall{c}(0, 0.025)
\hlfunctioncall{text}(arrpos[2, 1] + 0.05, arrpos[2, 2], \hlstring{"yes"})
\hlfunctioncall{text}(arrpos[3, 1] - 0.05, arrpos[3, 2], \hlstring{"no"})
\hlfunctioncall{text}(arrpos[4, 1] + 0.05, arrpos[4, 2] + 0.05, \hlstring{"yes"})
\hlfunctioncall{text}(arrpos[5, 1] - 0.05, arrpos[5, 2] + 0.05, \hlstring{"no"})
\end{alltt}
\end{kframe}
\includegraphics[width=\maxwidth]{figure/051-flowchart-demo-flowchart2} 
\begin{kframe}\begin{alltt}


\hlcomment{#####}
\hlfunctioncall{par}(mfrow = \hlfunctioncall{c}(2, 2))
\hlfunctioncall{par}(mar = \hlfunctioncall{c}(0, 0, 0, 0))
\hlfunctioncall{openplotmat}()
elpos <- \hlfunctioncall{coordinates}(\hlfunctioncall{c}(2, 3))
\hlfunctioncall{treearrow}(from = elpos[1:2, ], to = elpos[3:5, ], arr.side = 2, path = \hlstring{"H"})
\hlfunctioncall{for} (i in 1:5) \hlfunctioncall{textrect}(elpos[i, ], 0.15, 0.05, lab = i, cex = 1.5)

\hlfunctioncall{openplotmat}()
elpos <- \hlfunctioncall{coordinates}(\hlfunctioncall{c}(3, 2), hor = FALSE)
\hlfunctioncall{treearrow}(from = elpos[1:3, ], to = elpos[4:5, ], arr.side = 2, arr.pos = 0.2, 
    path = \hlstring{"V"})
\hlfunctioncall{for} (i in 1:5) \hlfunctioncall{textrect}(elpos[i, ], 0.15, 0.05, lab = i, cex = 1.5)

\hlfunctioncall{openplotmat}()
elpos <- \hlfunctioncall{coordinates}(\hlfunctioncall{c}(1, 4))
\hlfunctioncall{treearrow}(from = elpos[1, ], to = elpos[2:5, ], arr.side = 2, arr.pos = 0.7, 
    path = \hlstring{"H"})
\hlfunctioncall{for} (i in 1:5) \hlfunctioncall{textrect}(elpos[i, ], 0.05, 0.05, lab = i, cex = 1.5)

\hlfunctioncall{openplotmat}()
elpos <- \hlfunctioncall{coordinates}(\hlfunctioncall{c}(2, 1, 2, 3))
elpos[1, 1] <- 0.3
elpos[2, 1] <- 0.7
\hlfunctioncall{treearrow}(from = elpos[1:3, ], to = elpos[4:8, ], arr.side = 2, path = \hlstring{"H"})
\hlfunctioncall{for} (i in 1:8) \hlfunctioncall{bentarrow}(from = elpos[i, ], to = elpos[i, ] + \hlfunctioncall{c}(0.1, -0.05), 
    arr.pos = 1, arr.type = \hlstring{"circle"}, arr.col = \hlstring{"white"}, arr.length = 0.2)
\hlfunctioncall{for} (i in 1:8) \hlfunctioncall{textrect}(elpos[i, ], 0.05, 0.05, lab = i, cex = 1.5)
\hlfunctioncall{mtext}(side = 3, outer = TRUE, line = -2, \hlstring{"treearrow"}, cex = 1.5)
\end{alltt}
\end{kframe}
\includegraphics[width=\maxwidth]{figure/051-flowchart-demo-flowchart3} 
\begin{kframe}\begin{alltt}



\hlfunctioncall{par}(mfrow = \hlfunctioncall{c}(1, 1))

\hlfunctioncall{par}(mar = \hlfunctioncall{c}(0, 0, 0, 0))
\hlfunctioncall{openplotmat}()
elpos <- \hlfunctioncall{coordinates}(\hlfunctioncall{c}(1, 1, 2, 1))
\hlfunctioncall{straightarrow}(to = elpos[2, ], from = elpos[1, ])
\hlfunctioncall{treearrow}(from = elpos[2, ], to = elpos[3:4, ], arr.side = 2, path = \hlstring{"H"})
\hlfunctioncall{treearrow}(from = elpos[3:4, ], to = elpos[5, ], arr.side = 2, path = \hlstring{"H"})
\hlfunctioncall{segmentarrow}(from = elpos[5, ], to = elpos[2, ], dd = 0.4)
\hlfunctioncall{curvedarrow}(from = elpos[5, ], to = elpos[2, ], curve = 0.8)
col <- \hlfunctioncall{femmecol}(5)
\hlfunctioncall{texthexa}(mid = elpos[1, ], radx = 0.1, angle = 20, shadow.size = 0.01, rady = 0.05, 
    lab = 1, box.col = col[1])
\hlfunctioncall{textrect}(mid = elpos[2, ], radx = 0.1, shadow.size = 0.01, rady = 0.05, lab = 2, 
    box.col = col[2])
\hlfunctioncall{textround}(mid = elpos[3, ], radx = 0.05, shadow.size = 0.01, rady = 0.05, lab = 3, 
    box.col = col[3])
\hlfunctioncall{textellipse}(mid = elpos[4, ], radx = 0.05, shadow.size = 0.01, rady = 0.05, 
    lab = 4, box.col = col[4])
\hlfunctioncall{textellipse}(mid = elpos[5, ], radx = 0.05, shadow.size = 0.01, rady = 0.08, 
    angle = 45, lab = 5, box.col = col[5])
\end{alltt}
\end{kframe}
\includegraphics[width=\maxwidth]{figure/051-flowchart-demo-flowchart4} 
\begin{kframe}\begin{alltt}



\hlfunctioncall{par}(mar = \hlfunctioncall{c}(1, 1, 1, 1))
\hlfunctioncall{openplotmat}(main = \hlstring{"Arrowtypes"})
elpos <- \hlfunctioncall{coordinates}(\hlfunctioncall{c}(1, 2, 1), mx = 0.1, my = -0.1)
\hlfunctioncall{curvedarrow}(from = elpos[1, ], to = elpos[2, ], curve = -0.5, lty = 2, lcol = 2)
\hlfunctioncall{straightarrow}(from = elpos[1, ], to = elpos[2, ], lty = 3, lcol = 3)
\hlfunctioncall{segmentarrow}(from = elpos[1, ], to = elpos[2, ], lty = 1, lcol = 1)
\hlfunctioncall{treearrow}(from = elpos[2:3, ], to = elpos[4, ], lty = 4, lcol = 4)
\hlfunctioncall{bentarrow}(from = elpos[3, ], to = elpos[3, ] - \hlfunctioncall{c}(0.1, 0.1), arr.pos = 1, lty = 5, 
    lcol = 5)
\hlfunctioncall{bentarrow}(from = elpos[1, ], to = elpos[3, ], lty = 5, lcol = 5)
\hlfunctioncall{selfarrow}(pos = elpos[3, ], path = \hlstring{"R"}, lty = 6, curve = 0.075, lcol = 6)
\hlfunctioncall{splitarrow}(from = elpos[1, ], to = elpos[2:3, ], lty = 1, lwd = 1, dd = 0.7, 
    arr.side = 1:2, lcol = 7)

\hlfunctioncall{for} (i in 1:4) \hlfunctioncall{textrect}(elpos[i, ], 0.05, 0.05, lab = i, cex = 1.5)

\hlfunctioncall{legend}(\hlstring{"topright"}, lty = 1:7, legend = \hlfunctioncall{c}(\hlstring{"segmentarrow"}, \hlstring{"curvedarrow"}, \hlstring{"straightarrow"}, 
    \hlstring{"treearrow"}, \hlstring{"bentarrow"}, \hlstring{"selfarrow"}, \hlstring{"splitarrow"}), lwd = \hlfunctioncall{c}(\hlfunctioncall{rep}(2, 6), 
    1), col = 1:7)
\end{alltt}
\end{kframe}
\includegraphics[width=\maxwidth]{figure/051-flowchart-demo-flowchart5} 
\begin{kframe}\begin{alltt}

\hlfunctioncall{openplotmat}(main = \hlstring{"textbox shapes"})
rx <- 0.1
ry <- 0.05
pos <- \hlfunctioncall{coordinates}(\hlfunctioncall{c}(1, 1, 1, 1, 1, 1, 1), mx = -0.2)
\hlfunctioncall{textdiamond}(mid = pos[1, ], radx = rx, rady = ry, lab = LETTERS[1], cex = 2, 
    shadow.col = \hlstring{"lightblue"})
\hlfunctioncall{textellipse}(mid = pos[2, ], radx = rx, rady = ry, lab = LETTERS[2], cex = 2, 
    shadow.col = \hlstring{"blue"})
\hlfunctioncall{texthexa}(mid = pos[3, ], radx = rx, rady = ry, lab = LETTERS[3], cex = 2, shadow.col = \hlstring{"darkblue"})
\hlfunctioncall{textmulti}(mid = pos[4, ], nr = 7, radx = rx, rady = ry, lab = LETTERS[4], cex = 2, 
    shadow.col = \hlstring{"red"})
\hlfunctioncall{textrect}(mid = pos[5, ], radx = rx, rady = ry, lab = LETTERS[5], cex = 2, shadow.col = \hlstring{"darkred"})
\hlfunctioncall{textround}(mid = pos[6, ], radx = rx, rady = ry, lab = LETTERS[6], cex = 2, shadow.col = \hlstring{"black"})
\hlfunctioncall{textempty}(mid = pos[7, ], lab = LETTERS[7], cex = 2, box.col = \hlstring{"yellow"})
pos[, 1] <- pos[, 1] + 0.5
\hlfunctioncall{text}(pos[, 1], pos[, 2], \hlfunctioncall{c}(\hlstring{"textdiamond"}, \hlstring{"textellipse"}, \hlstring{"texthexa"}, \hlstring{"textmulti"}, 
    \hlstring{"textrect"}, \hlstring{"textround"}, \hlstring{"textempty"}))
\end{alltt}
\end{kframe}
\includegraphics[width=\maxwidth]{figure/051-flowchart-demo-flowchart6} 
\begin{kframe}\begin{alltt}


mf <- \hlfunctioncall{par}(mfrow = \hlfunctioncall{c}(2, 2))
\hlfunctioncall{example}(bentarrow)
\end{alltt}
\begin{verbatim}
## 
## bntrrw> openplotmat(main = "bentarrow")

## 
## bntrrw> pos <- cbind( A <- seq(0.1, 0.9, by = 0.2), rev(A))
## 
## bntrrw> text(pos, LETTERS[1:5], cex = 2)
## 
## bntrrw> for (i in 1:4) 
## bntrrw+   bentarrow(from = pos[i,] + c(0.05, 0), to = pos[i+1,] + c(0, 0.05),
## bntrrw+             arr.pos = 1, arr.adj = 1)
## 
## bntrrw> for (i in 1:2) 
## bntrrw+   bentarrow(from = pos[i,] + c(0.05, 0), to = pos[i+1, ] + c(0, 0.05),
## bntrrw+             arr.pos = 0.5, path = "V", lcol = "lightblue", 
## bntrrw+             arr.type = "triangle")
## 
## bntrrw> bentarrow(from = pos[3, ] + c(0.05, 0), to = pos[4, ] + c(0, 0.05),
## bntrrw+           arr.pos = 0.7, arr.side = 1, path = "V", lcol = "darkblue")
## 
## bntrrw> bentarrow(from = pos[4, ] + c(0.05, 0), to = pos[5, ] + c(0, 0.05),
## bntrrw+           arr.pos = 0.7, arr.side = 1:2, path = "V", lcol = "blue")
\end{verbatim}
\begin{alltt}
\hlfunctioncall{example}(coordinates)
\end{alltt}
\begin{verbatim}
## 
## crdnts> openplotmat(main = "coordinates")

## 
## crdnts> text(coordinates(N = 6), lab = LETTERS[1:6], cex = 2)
## 
## crdnts> text(coordinates(N = 8, relsize = 0.5), lab = letters[1:8], cex = 2)
## 
## crdnts> openplotmat(main = "coordinates")

## 
## crdnts> text(coordinates(pos = c(2, 4, 2)), lab = letters[1:8], cex = 2)
## 
## crdnts> plot(0, type = "n", xlim = c(0, 5), ylim = c(2, 8), main = "coordinates")
\end{verbatim}
\end{kframe}
\includegraphics[width=\maxwidth]{figure/051-flowchart-demo-flowchart7} 
\begin{kframe}\begin{verbatim}
## 
## crdnts> text(coordinates(pos = c(2, 4, 3), hor = FALSE), lab = 1:9, cex = 2)
\end{verbatim}
\begin{alltt}
\hlfunctioncall{par}(mfrow = \hlfunctioncall{c}(2, 2))
\hlfunctioncall{example}(curvedarrow)
\end{alltt}
\begin{verbatim}
## 
## crvdrr> openplotmat(main = "curvedarrow")

## 
## crvdrr> pos <- coordinates(pos = 4, my = 0.2)
## 
## crvdrr> text(pos, LETTERS[1:4], cex = 2)
## 
## crvdrr> for (i in 1:3) 
## crvdrr+   curvedarrow(from = pos[1, ] + c(0,-0.05), to = pos[i+1, ] + c(0,-0.05),
## crvdrr+               curve = 0.5, arr.pos = 1)
## 
## crvdrr> for (i in 1:3) 
## crvdrr+   curvedarrow(from = pos[1, ] + c(0, 0.05), to = pos[i+1, ] + c(0, 0.05),
## crvdrr+               curve = -0.25, arr.adj = 1, arr.pos = 0.5, 
## crvdrr+               arr.type = "triangle", arr.col = "blue")
\end{verbatim}
\begin{alltt}
\hlfunctioncall{example}(segmentarrow)
\end{alltt}
\begin{verbatim}
## 
## sgmntr> openplotmat(main="segmentarrow")

## 
## sgmntr> pos <-cbind(A <- seq(0.2, 0.8, by = 0.2), rev(A))
## 
## sgmntr> text(pos, LETTERS[1:4], cex = 2)
## 
## sgmntr> segmentarrow(from = pos[1, ] + c(0, 0.05), to = pos[2, ] + c(0, 0.05),
## sgmntr+              arr.pos = 1, arr.adj = 1, dd = 0.1, 
## sgmntr+              path = "UHD", lcol = "darkred")
## 
## sgmntr> segmentarrow(from = pos[2, ] + c(-0.05, 0), to = pos[3, ] + c(-0.05, 0.01),
## sgmntr+              arr.pos = 1, arr.adj = 1, dd = 0.1,
## sgmntr+              lcol = "black", arr.type = "triangle")
## 
## sgmntr> segmentarrow(from = pos[2, ] + c(0.05, 0), to = pos[3, ] + c(0.05, 0.01),
## sgmntr+              arr.pos = 0.5, dd = 0.3, path = "RVL", arr.side = 1,
## sgmntr+              lcol = "lightblue", arr.type = "simple")  
## 
## sgmntr> segmentarrow(from = pos[3, ] + c(0.05, 0), to = pos[4, ] + c(-0.05, 0.01),
## sgmntr+              arr.pos = 0.5, dd = 0.05, path = "RVL", lcol = "darkblue",
## sgmntr+              arr.type = "ellipse")  
## 
## sgmntr> segmentarrow(from = pos[3, ] + c(0, -0.05), to = pos[4, ] + c(0, 0.05),
## sgmntr+              arr.pos = 0.5, arr.side = 3, dd = 0.05, path = "DHU",
## sgmntr+              lcol = "darkgreen")  
## 
## sgmntr> segmentarrow(from = pos[3,] + c(-0.05, -0.05), to = pos[4, ] + c(0, -0.05),
## sgmntr+              arr.pos = 0.5, arr.side = 1:3, dd = 0.3, path = "DHU",
## sgmntr+              lcol = "green")
\end{verbatim}
\begin{alltt}
\hlfunctioncall{example}(selfarrow)
\end{alltt}
\begin{verbatim}
## 
## slfrrw> openplotmat(main = "selfarrow")

## 
## slfrrw> pos <- coordinates(3, mx = 0.05)
## 
## slfrrw> text(pos, LETTERS[1:3], cex = 2)
## 
## slfrrw> for (i in 1:3) 
## slfrrw+   selfarrow(pos = pos[i, ], path = "R", arr.pos = 0.2,
## slfrrw+             curve = c(0.05, 0.1), lcol = "darkred")
## 
## slfrrw> for (i in 1:3) 
## slfrrw+   selfarrow(pos = pos[i, ], path = "L", arr.pos = 0.7,
## slfrrw+             lcol = "darkblue", curve = c(0.05, 0.05))
## 
## slfrrw> for (i in 1:3) 
## slfrrw+   selfarrow(pos = pos[i, ], path = "L", arr.pos = 0.5,
## slfrrw+             lcol = "darkgreen", code = i, arr.type = "triangle")
\end{verbatim}
\begin{alltt}
\hlfunctioncall{example}(straightarrow)
\end{alltt}
\begin{verbatim}
## 
## strght> openplotmat(main = "straightarrow")
\end{verbatim}
\end{kframe}
\includegraphics[width=\maxwidth]{figure/051-flowchart-demo-flowchart8} 
\begin{kframe}\begin{verbatim}
## 
## strght> pos <- coordinates(c(2, 3, 1))
## 
## strght> for (i in 1:5) 
## strght+   straightarrow(from = pos[i, ], to = pos[i+1, ], arr.pos = 0.5)
## 
## strght> straightarrow(from = pos[6, ], to = pos[6, ] + c(0.3, 0.), 
## strght+               arr.type = "T", arr.pos = 1, arr.lwd = 3)    
## 
## strght> for (i in 1:6) 
## strght+   textrect(pos[i, ], lab = LETTERS[i], radx = 0.05)
\end{verbatim}
\begin{alltt}
\hlfunctioncall{par}(mfrow = \hlfunctioncall{c}(2, 2))
\hlfunctioncall{example}(treearrow)
\end{alltt}
\begin{verbatim}
## 
## trerrw> openplotmat(main = "treearrow")

## 
## trerrw> pos <- coordinates(c(3, 2, 4, 1))
## 
## trerrw> treearrow(from = pos[1:5, ], to = pos[6:10, ])
## 
## trerrw> for (i in 1:10) 
## trerrw+   textrect(pos[i, ], lab = i, cex = 2, radx = 0.05)
## 
## trerrw> openplotmat(main = "treearrow")

## 
## trerrw> pos <- coordinates(c(2, 4), hor = FALSE)
## 
## trerrw> treearrow(from = pos[1:2, ], to = pos[3:6, ], 
## trerrw+           arr.side = 1:2, path = "V")
## 
## trerrw> for (i in 1:6) 
## trerrw+   textrect(pos[i, ], lab = i, cex = 2, radx = 0.05)
## 
## trerrw> openplotmat(main = "treearrow")

## 
## trerrw> pos <- coordinates(c(3, 5, 7, 7, 5, 3))
## 
## trerrw> treearrow(from = pos[1:15, ], to = pos[15:30, ], arr.side = 0)
## 
## trerrw> for (i in 1:30) 
## trerrw+   textrect(pos[i, ], lab = i, cex = 1.2, radx = 0.025)
\end{verbatim}
\begin{alltt}
\hlfunctioncall{par}(mfrow = \hlfunctioncall{c}(2, 2))
\end{alltt}
\end{kframe}
\includegraphics[width=\maxwidth]{figure/051-flowchart-demo-flowchart9} 
\begin{kframe}\begin{alltt}
\hlfunctioncall{example}(splitarrow)
\end{alltt}
\begin{verbatim}
## 
## spltrr> openplotmat(main = "splitarrow")

## 
## spltrr> pos <- coordinates(c(1, 2, 2, 4, 1))
## 
## spltrr> splitarrow(from = pos[1, ], to = pos[2:10, ], 
## spltrr+            arr.side = 1, centre = c(0.5, 0.625))
## 
## spltrr> for (i in 1:10) 
## spltrr+   textrect(pos[i, ], lab = i, cex = 2, radx = 0.05)
## 
## spltrr> openplotmat(main = "splitarrow")

## 
## spltrr> pos <- coordinates(c(1, 3))
## 
## spltrr> splitarrow(from = pos[1,], to = pos[2:4, ], arr.side = 1)
## 
## spltrr> splitarrow(from = pos[1,], to = pos[2:4, ], arr.side = 2)
## 
## spltrr> for (i in 1:4) 
## spltrr+   textrect(pos[i, ], lab = i, cex = 2, radx = 0.05)
## 
## spltrr> openplotmat(main = "splitarrow")

## 
## spltrr> pos <- coordinates(N = 6)
## 
## spltrr> pos <- rbind(c(0.5, 0.5), pos)
## 
## spltrr> splitarrow(from = pos[1, ], to = pos[2:7, ], arr.side = 2)
## 
## spltrr> for (i in 1:7)
## spltrr+   textrect(pos[i, ], lab = i, cex = 2, radx = 0.05)
\end{verbatim}
\begin{alltt}
\hlfunctioncall{par}(mfrow = \hlfunctioncall{c}(2, 2))
\end{alltt}
\end{kframe}
\includegraphics[width=\maxwidth]{figure/051-flowchart-demo-flowchart10} 
\begin{kframe}\begin{alltt}
\hlfunctioncall{example}(textdiamond)
\end{alltt}
\begin{verbatim}
## 
## txtdmn> openplotmat(xlim = c(-0.1, 1.1), main = "textdiamond")

## 
## txtdmn> for (i in 1:10) 
## txtdmn+   textdiamond(mid = runif(2), col = i, radx = 0.1, rady = 0.05,
## txtdmn+               lab = LETTERS[i], cex = 2, angle = runif(1)*360)
\end{verbatim}
\begin{alltt}
\hlfunctioncall{example}(textellipse)
\end{alltt}
\begin{verbatim}
## 
## txtllp> openplotmat(xlim = c(-0.1, 1.1), main = "textellipse")

## 
## txtllp> for (i in 1:10) 
## txtllp+   textellipse(mid = runif(2), col = i, box.col = grey(0.95),
## txtllp+               radx = 0.1, rady = 0.05, lab = LETTERS[i],
## txtllp+               cex = 2, angle = runif(1)*360)
\end{verbatim}
\begin{alltt}
\hlfunctioncall{example}(textempty)
\end{alltt}
\begin{verbatim}
## 
## txtmpt> openplotmat(xlim = c(-0.1, 1.1), col = "lightgrey", main = "textempty")

## 
## txtmpt> for (i in 1:10) 
## txtmpt+   textempty(mid = runif(2), box.col = i, lab = LETTERS[i], cex = 2)
## 
## txtmpt> textempty(mid = c(0.5, 0.5), adj = c(0, 0), 
## txtmpt+   lab = "textempty", box.col = "white")
\end{verbatim}
\begin{alltt}
\hlfunctioncall{example}(texthexa)
\end{alltt}
\begin{verbatim}
## 
## texthx>   openplotmat(xlim = c(-0.1, 1.1), main = "texthexa")
\end{verbatim}
\end{kframe}
\includegraphics[width=\maxwidth]{figure/051-flowchart-demo-flowchart11} 
\begin{kframe}\begin{verbatim}
## 
## texthx>   for (i in 1:20) 
## texthx+     texthexa(mid = runif(2), angle = runif(1)*360, col = i,
## texthx+              box.col = grey(0.95), radx = 0.1, rady = 0.05,
## texthx+              lab = LETTERS[i], cex = 2)
\end{verbatim}
\begin{alltt}
\hlfunctioncall{example}(textmulti)
\end{alltt}
\begin{verbatim}
## 
## txtmlt>   openplotmat(xlim = c(-0.1, 1.1), main = "textmulti")

## 
## txtmlt>   for (i in 1:10) 
## txtmlt+     textmulti(mid = runif(2), col = i, radx = 0.1, rady = 0.1,
## txtmlt+              lab = LETTERS[i], cex = 2, nr = trunc(i/1.5)+3)
\end{verbatim}
\begin{alltt}
\hlfunctioncall{example}(textplain)
\end{alltt}
\begin{verbatim}
## 
## txtpln>   openplotmat(main = "textplain")

## 
## txtpln>   textplain(mid = c(0.5, 0.5), 
## txtpln+             lab = c("this text is", "centered", "4 strings", "on 4 lines"))
## 
## txtpln>   textplain(mid = c(0.5, 0.2), adj = c(0, 0.5), font = 2, height = 0.05,
## txtpln+             lab = c("this text is","left alligned"))
## 
## txtpln>   textplain(mid = c(0.5, 0.8), adj = c(1, 0.5), font = 3, height = 0.05, 
## txtpln+             lab = c("this text is","right alligned"))
\end{verbatim}
\begin{alltt}
\hlfunctioncall{example}(textrect)
\end{alltt}
\begin{verbatim}
## 
## txtrct>   openplotmat(xlim = c(-0.1, 1.1), main = "textrect")

## 
## txtrct>   for (i in 1:10) 
## txtrct+     textrect(mid = runif(2), col = i, radx = 0.1, rady = 0.1,
## txtrct+             lab = LETTERS[i], cex = 2)
## 
## txtrct>   openplotmat(xlim = c(-0.1, 1.1), main = "textparallel")
\end{verbatim}
\end{kframe}
\includegraphics[width=\maxwidth]{figure/051-flowchart-demo-flowchart12} 
\begin{kframe}\begin{verbatim}
## 
## txtrct>   elpos <-coordinates (c(1, 1, 1, 1, 1))
## 
## txtrct>   textparallel(mid = elpos[1,], col = 1, radx = 0.2, rady = 0.1,
## txtrct+             lab = "theta=20", theta = 20)
## 
## txtrct>   textparallel(mid = elpos[2,], col = 1, radx = 0.2, rady = 0.1,
## txtrct+             lab = "theta=60", theta = 60)
## 
## txtrct>   textparallel(mid = elpos[3,], col = 1, radx = 0.2, rady = 0.1,
## txtrct+             lab = "theta=100", theta = 100)
## 
## txtrct>   textparallel(mid = elpos[4,], col = 1, radx = 0.2, rady = 0.1,
## txtrct+             lab = "theta=140", theta = 140)
## 
## txtrct>   textparallel(mid = elpos[5,], col = 1, radx = 0.2, rady = 0.1,
## txtrct+             lab = "theta=170", theta = 170)
\end{verbatim}
\begin{alltt}
\hlfunctioncall{example}(textround)
\end{alltt}
\begin{verbatim}
## 
## txtrnd>   openplotmat(xlim = c(-0.1, 1.1), main = "textround")

## 
## txtrnd>   for (i in 1:10) 
## txtrnd+     textround(mid = runif(2), col = i, 
## txtrnd+               radx = 0.03, rady = 0.075,
## txtrnd+               lab = LETTERS[i], cex = 2)
\end{verbatim}
\begin{alltt}


\hlfunctioncall{par}(mfrow = mf)
\end{alltt}
\end{kframe}
\includegraphics[width=\maxwidth]{figure/051-flowchart-demo-flowchart13} 

\end{knitrout}

\end{document}
