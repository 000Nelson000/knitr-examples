\documentclass[a4paper,titlepage]{tufte-handout}\usepackage[]{graphicx}\usepackage[]{color}
%% maxwidth is the original width if it is less than linewidth
%% otherwise use linewidth (to make sure the graphics do not exceed the margin)
\makeatletter
\def\maxwidth{ %
  \ifdim\Gin@nat@width>\linewidth
    \linewidth
  \else
    \Gin@nat@width
  \fi
}
\makeatother

\definecolor{fgcolor}{rgb}{0.345, 0.345, 0.345}
\newcommand{\hlnum}[1]{\textcolor[rgb]{0.686,0.059,0.569}{#1}}%
\newcommand{\hlstr}[1]{\textcolor[rgb]{0.192,0.494,0.8}{#1}}%
\newcommand{\hlcom}[1]{\textcolor[rgb]{0.678,0.584,0.686}{\textit{#1}}}%
\newcommand{\hlopt}[1]{\textcolor[rgb]{0,0,0}{#1}}%
\newcommand{\hlstd}[1]{\textcolor[rgb]{0.345,0.345,0.345}{#1}}%
\newcommand{\hlkwa}[1]{\textcolor[rgb]{0.161,0.373,0.58}{\textbf{#1}}}%
\newcommand{\hlkwb}[1]{\textcolor[rgb]{0.69,0.353,0.396}{#1}}%
\newcommand{\hlkwc}[1]{\textcolor[rgb]{0.333,0.667,0.333}{#1}}%
\newcommand{\hlkwd}[1]{\textcolor[rgb]{0.737,0.353,0.396}{\textbf{#1}}}%

\usepackage{framed}
\makeatletter
\newenvironment{kframe}{%
 \def\at@end@of@kframe{}%
 \ifinner\ifhmode%
  \def\at@end@of@kframe{\end{minipage}}%
  \begin{minipage}{\columnwidth}%
 \fi\fi%
 \def\FrameCommand##1{\hskip\@totalleftmargin \hskip-\fboxsep
 \colorbox{shadecolor}{##1}\hskip-\fboxsep
     % There is no \\@totalrightmargin, so:
     \hskip-\linewidth \hskip-\@totalleftmargin \hskip\columnwidth}%
 \MakeFramed {\advance\hsize-\width
   \@totalleftmargin\z@ \linewidth\hsize
   \@setminipage}}%
 {\par\unskip\endMakeFramed%
 \at@end@of@kframe}
\makeatother

\definecolor{shadecolor}{rgb}{.97, .97, .97}
\definecolor{messagecolor}{rgb}{0, 0, 0}
\definecolor{warningcolor}{rgb}{1, 0, 1}
\definecolor{errorcolor}{rgb}{1, 0, 0}
\newenvironment{knitrout}{}{} % an empty environment to be redefined in TeX

\usepackage{alltt}
\title{ggplot2 Gallery}
\IfFileExists{upquote.sty}{\usepackage{upquote}}{}
\begin{document}
\maketitle
\tableofcontents



% all geoms in ggplot2



\section{geom\_abline}

\begin{knitrout}
\definecolor{shadecolor}{rgb}{0.969, 0.969, 0.969}\color{fgcolor}\begin{kframe}
\begin{alltt}
\hlcom{### Name: geom_abline}
\hlcom{### Title: Lines: horizontal, vertical, and specified by slope and}
\hlcom{###   intercept.}
\hlcom{### Aliases: geom_abline geom_hline geom_vline}

\hlcom{### ** Examples}

\hlstd{p} \hlkwb{<-} \hlkwd{ggplot}\hlstd{(mtcars,} \hlkwd{aes}\hlstd{(wt, mpg))} \hlopt{+} \hlkwd{geom_point}\hlstd{()}

\hlcom{# Fixed values}
\hlstd{p} \hlopt{+} \hlkwd{geom_vline}\hlstd{(}\hlkwc{xintercept} \hlstd{=} \hlnum{5}\hlstd{)}
\end{alltt}
\end{kframe}
\includegraphics[width=\maxwidth]{figure/021-ggplot2-geoms-geom_abline-1} 
\begin{kframe}\begin{alltt}
\hlstd{p} \hlopt{+} \hlkwd{geom_vline}\hlstd{(}\hlkwc{xintercept} \hlstd{=} \hlnum{1}\hlopt{:}\hlnum{5}\hlstd{)}
\end{alltt}
\end{kframe}
\includegraphics[width=\maxwidth]{figure/021-ggplot2-geoms-geom_abline-2} 
\begin{kframe}\begin{alltt}
\hlstd{p} \hlopt{+} \hlkwd{geom_hline}\hlstd{(}\hlkwc{yintercept} \hlstd{=} \hlnum{20}\hlstd{)}
\end{alltt}
\end{kframe}
\includegraphics[width=\maxwidth]{figure/021-ggplot2-geoms-geom_abline-3} 
\begin{kframe}\begin{alltt}
\hlstd{p} \hlopt{+} \hlkwd{geom_abline}\hlstd{()} \hlcom{# Can't see it - outside the range of the data}
\end{alltt}
\end{kframe}
\includegraphics[width=\maxwidth]{figure/021-ggplot2-geoms-geom_abline-4} 
\begin{kframe}\begin{alltt}
\hlstd{p} \hlopt{+} \hlkwd{geom_abline}\hlstd{(}\hlkwc{intercept} \hlstd{=} \hlnum{20}\hlstd{)}
\end{alltt}
\end{kframe}
\includegraphics[width=\maxwidth]{figure/021-ggplot2-geoms-geom_abline-5} 
\begin{kframe}\begin{alltt}
\hlcom{# Calculate slope and intercept of line of best fit}
\hlkwd{coef}\hlstd{(}\hlkwd{lm}\hlstd{(mpg} \hlopt{~} \hlstd{wt,} \hlkwc{data} \hlstd{= mtcars))}
\end{alltt}
\begin{verbatim}
## (Intercept)          wt 
##      37.285      -5.344
\end{verbatim}
\begin{alltt}
\hlstd{p} \hlopt{+} \hlkwd{geom_abline}\hlstd{(}\hlkwc{intercept} \hlstd{=} \hlnum{37}\hlstd{,} \hlkwc{slope} \hlstd{=} \hlopt{-}\hlnum{5}\hlstd{)}
\end{alltt}
\end{kframe}
\includegraphics[width=\maxwidth]{figure/021-ggplot2-geoms-geom_abline-6} 
\begin{kframe}\begin{alltt}
\hlcom{# But this is easier to do with geom_smooth:}
\hlstd{p} \hlopt{+} \hlkwd{geom_smooth}\hlstd{(}\hlkwc{method} \hlstd{=} \hlstr{"lm"}\hlstd{,} \hlkwc{se} \hlstd{=} \hlnum{FALSE}\hlstd{)}
\end{alltt}
\end{kframe}
\includegraphics[width=\maxwidth]{figure/021-ggplot2-geoms-geom_abline-7} 
\begin{kframe}\begin{alltt}
\hlcom{# To show different lines in different facets, use aesthetics}
\hlstd{p} \hlkwb{<-} \hlkwd{ggplot}\hlstd{(mtcars,} \hlkwd{aes}\hlstd{(mpg, wt))} \hlopt{+}
  \hlkwd{geom_point}\hlstd{()} \hlopt{+}
  \hlkwd{facet_wrap}\hlstd{(}\hlopt{~} \hlstd{cyl)}

\hlstd{mean_wt} \hlkwb{<-} \hlkwd{data.frame}\hlstd{(}\hlkwc{cyl} \hlstd{=} \hlkwd{c}\hlstd{(}\hlnum{4}\hlstd{,} \hlnum{6}\hlstd{,} \hlnum{8}\hlstd{),} \hlkwc{wt} \hlstd{=} \hlkwd{c}\hlstd{(}\hlnum{2.28}\hlstd{,} \hlnum{3.11}\hlstd{,} \hlnum{4.00}\hlstd{))}
\hlstd{p} \hlopt{+} \hlkwd{geom_hline}\hlstd{(}\hlkwd{aes}\hlstd{(}\hlkwc{yintercept} \hlstd{= wt), mean_wt)}
\end{alltt}
\end{kframe}
\includegraphics[width=\maxwidth]{figure/021-ggplot2-geoms-geom_abline-8} 
\begin{kframe}\begin{alltt}
\hlcom{# You can also control other aesthetics}
\hlkwd{ggplot}\hlstd{(mtcars,} \hlkwd{aes}\hlstd{(mpg, wt,} \hlkwc{colour} \hlstd{= wt))} \hlopt{+}
  \hlkwd{geom_point}\hlstd{()} \hlopt{+}
  \hlkwd{geom_hline}\hlstd{(}\hlkwd{aes}\hlstd{(}\hlkwc{yintercept} \hlstd{= wt,} \hlkwc{colour} \hlstd{= wt), mean_wt)} \hlopt{+}
  \hlkwd{facet_wrap}\hlstd{(}\hlopt{~} \hlstd{cyl)}
\end{alltt}
\end{kframe}
\includegraphics[width=\maxwidth]{figure/021-ggplot2-geoms-geom_abline-9} 

\end{knitrout}


\section{geom\_area}

\begin{knitrout}
\definecolor{shadecolor}{rgb}{0.969, 0.969, 0.969}\color{fgcolor}\begin{kframe}
\begin{alltt}
\hlcom{### Name: geom_ribbon}
\hlcom{### Title: Ribbons and area plots.}
\hlcom{### Aliases: geom_area geom_ribbon}

\hlcom{### ** Examples}

\hlcom{# Generate data}
\hlstd{huron} \hlkwb{<-} \hlkwd{data.frame}\hlstd{(}\hlkwc{year} \hlstd{=} \hlnum{1875}\hlopt{:}\hlnum{1972}\hlstd{,} \hlkwc{level} \hlstd{=} \hlkwd{as.vector}\hlstd{(LakeHuron))}
\hlstd{h} \hlkwb{<-} \hlkwd{ggplot}\hlstd{(huron,} \hlkwd{aes}\hlstd{(year))}

\hlstd{h} \hlopt{+} \hlkwd{geom_ribbon}\hlstd{(}\hlkwd{aes}\hlstd{(}\hlkwc{ymin}\hlstd{=}\hlnum{0}\hlstd{,} \hlkwc{ymax}\hlstd{=level))}
\end{alltt}
\end{kframe}
\includegraphics[width=\maxwidth]{figure/021-ggplot2-geoms-geom_area-1} 
\begin{kframe}\begin{alltt}
\hlstd{h} \hlopt{+} \hlkwd{geom_area}\hlstd{(}\hlkwd{aes}\hlstd{(}\hlkwc{y} \hlstd{= level))}
\end{alltt}
\end{kframe}
\includegraphics[width=\maxwidth]{figure/021-ggplot2-geoms-geom_area-2} 
\begin{kframe}\begin{alltt}
\hlcom{# Add aesthetic mappings}
\hlstd{h} \hlopt{+}
  \hlkwd{geom_ribbon}\hlstd{(}\hlkwd{aes}\hlstd{(}\hlkwc{ymin} \hlstd{= level} \hlopt{-} \hlnum{1}\hlstd{,} \hlkwc{ymax} \hlstd{= level} \hlopt{+} \hlnum{1}\hlstd{),} \hlkwc{fill} \hlstd{=} \hlstr{"grey70"}\hlstd{)} \hlopt{+}
  \hlkwd{geom_line}\hlstd{(}\hlkwd{aes}\hlstd{(}\hlkwc{y} \hlstd{= level))}
\end{alltt}
\end{kframe}
\includegraphics[width=\maxwidth]{figure/021-ggplot2-geoms-geom_area-3} 

\end{knitrout}


\section{geom\_bar}

\begin{knitrout}
\definecolor{shadecolor}{rgb}{0.969, 0.969, 0.969}\color{fgcolor}\begin{kframe}
\begin{alltt}
\hlcom{### Name: geom_bar}
\hlcom{### Title: Bars, rectangles with bases on x-axis}
\hlcom{### Aliases: geom_bar stat_count}

\hlcom{### ** Examples}

\hlcom{# geom_bar is designed to make it easy to create bar charts that show}
\hlcom{# counts (or sums of weights)}
\hlstd{g} \hlkwb{<-} \hlkwd{ggplot}\hlstd{(mpg,} \hlkwd{aes}\hlstd{(class))}
\hlcom{# Number of cars in each class:}
\hlstd{g} \hlopt{+} \hlkwd{geom_bar}\hlstd{()}
\end{alltt}
\end{kframe}
\includegraphics[width=\maxwidth]{figure/021-ggplot2-geoms-geom_bar-1} 
\begin{kframe}\begin{alltt}
\hlcom{# Total engine displacement of each class}
\hlstd{g} \hlopt{+} \hlkwd{geom_bar}\hlstd{(}\hlkwd{aes}\hlstd{(}\hlkwc{weight} \hlstd{= displ))}
\end{alltt}
\end{kframe}
\includegraphics[width=\maxwidth]{figure/021-ggplot2-geoms-geom_bar-2} 
\begin{kframe}\begin{alltt}
\hlcom{# To show (e.g.) means, you need stat = "identity"}
\hlstd{df} \hlkwb{<-} \hlkwd{data.frame}\hlstd{(}\hlkwc{trt} \hlstd{=} \hlkwd{c}\hlstd{(}\hlstr{"a"}\hlstd{,} \hlstr{"b"}\hlstd{,} \hlstr{"c"}\hlstd{),} \hlkwc{outcome} \hlstd{=} \hlkwd{c}\hlstd{(}\hlnum{2.3}\hlstd{,} \hlnum{1.9}\hlstd{,} \hlnum{3.2}\hlstd{))}
\hlkwd{ggplot}\hlstd{(df,} \hlkwd{aes}\hlstd{(trt, outcome))} \hlopt{+}
  \hlkwd{geom_bar}\hlstd{(}\hlkwc{stat} \hlstd{=} \hlstr{"identity"}\hlstd{)}
\end{alltt}
\end{kframe}
\includegraphics[width=\maxwidth]{figure/021-ggplot2-geoms-geom_bar-3} 
\begin{kframe}\begin{alltt}
\hlcom{# But geom_point() display exactly the same information and doesn't}
\hlcom{# require the y-axis to touch zero.}
\hlkwd{ggplot}\hlstd{(df,} \hlkwd{aes}\hlstd{(trt, outcome))} \hlopt{+}
  \hlkwd{geom_point}\hlstd{()}
\end{alltt}
\end{kframe}
\includegraphics[width=\maxwidth]{figure/021-ggplot2-geoms-geom_bar-4} 
\begin{kframe}\begin{alltt}
\hlcom{# You can also use geom_bar() with continuous data, in which case}
\hlcom{# it will show counts at unique locations}
\hlstd{df} \hlkwb{<-} \hlkwd{data.frame}\hlstd{(}\hlkwc{x} \hlstd{=} \hlkwd{rep}\hlstd{(}\hlkwd{c}\hlstd{(}\hlnum{2.9}\hlstd{,} \hlnum{3.1}\hlstd{,} \hlnum{4.5}\hlstd{),} \hlkwd{c}\hlstd{(}\hlnum{5}\hlstd{,} \hlnum{10}\hlstd{,} \hlnum{4}\hlstd{)))}
\hlkwd{ggplot}\hlstd{(df,} \hlkwd{aes}\hlstd{(x))} \hlopt{+} \hlkwd{geom_bar}\hlstd{()}
\end{alltt}
\end{kframe}
\includegraphics[width=\maxwidth]{figure/021-ggplot2-geoms-geom_bar-5} 
\begin{kframe}\begin{alltt}
\hlcom{# cf. a histogram of the same data}
\hlkwd{ggplot}\hlstd{(df,} \hlkwd{aes}\hlstd{(x))} \hlopt{+} \hlkwd{geom_histogram}\hlstd{(}\hlkwc{binwidth} \hlstd{=} \hlnum{0.5}\hlstd{)}
\end{alltt}
\end{kframe}
\includegraphics[width=\maxwidth]{figure/021-ggplot2-geoms-geom_bar-6} 
\begin{kframe}\begin{alltt}
\hlcom{## No test: }
\hlcom{# Bar charts are automatically stacked when multiple bars are placed}
\hlcom{# at the same location}
\hlstd{g} \hlopt{+} \hlkwd{geom_bar}\hlstd{(}\hlkwd{aes}\hlstd{(}\hlkwc{fill} \hlstd{= drv))}
\end{alltt}
\end{kframe}
\includegraphics[width=\maxwidth]{figure/021-ggplot2-geoms-geom_bar-7} 
\begin{kframe}\begin{alltt}
\hlcom{# You can instead dodge, or fill them}
\hlstd{g} \hlopt{+} \hlkwd{geom_bar}\hlstd{(}\hlkwd{aes}\hlstd{(}\hlkwc{fill} \hlstd{= drv),} \hlkwc{position} \hlstd{=} \hlstr{"dodge"}\hlstd{)}
\end{alltt}
\end{kframe}
\includegraphics[width=\maxwidth]{figure/021-ggplot2-geoms-geom_bar-8} 
\begin{kframe}\begin{alltt}
\hlstd{g} \hlopt{+} \hlkwd{geom_bar}\hlstd{(}\hlkwd{aes}\hlstd{(}\hlkwc{fill} \hlstd{= drv),} \hlkwc{position} \hlstd{=} \hlstr{"fill"}\hlstd{)}
\end{alltt}
\end{kframe}
\includegraphics[width=\maxwidth]{figure/021-ggplot2-geoms-geom_bar-9} 
\begin{kframe}\begin{alltt}
\hlcom{# To change plot order of bars, change levels in underlying factor}
\hlstd{reorder_size} \hlkwb{<-} \hlkwa{function}\hlstd{(}\hlkwc{x}\hlstd{) \{}
  \hlkwd{factor}\hlstd{(x,} \hlkwc{levels} \hlstd{=} \hlkwd{names}\hlstd{(}\hlkwd{sort}\hlstd{(}\hlkwd{table}\hlstd{(x))))}
\hlstd{\}}
\hlkwd{ggplot}\hlstd{(mpg,} \hlkwd{aes}\hlstd{(}\hlkwd{reorder_size}\hlstd{(class)))} \hlopt{+} \hlkwd{geom_bar}\hlstd{()}
\end{alltt}
\end{kframe}
\includegraphics[width=\maxwidth]{figure/021-ggplot2-geoms-geom_bar-10} 
\begin{kframe}\begin{alltt}
\hlcom{## End(No test)}
\end{alltt}
\end{kframe}
\end{knitrout}


\section{geom\_bin2d}

\begin{knitrout}
\definecolor{shadecolor}{rgb}{0.969, 0.969, 0.969}\color{fgcolor}\begin{kframe}
\begin{alltt}
\hlcom{### Name: geom_bin2d}
\hlcom{### Title: Add heatmap of 2d bin counts.}
\hlcom{### Aliases: geom_bin2d stat_bin2d stat_bin_2d}

\hlcom{### ** Examples}

\hlstd{d} \hlkwb{<-} \hlkwd{ggplot}\hlstd{(diamonds,} \hlkwd{aes}\hlstd{(x, y))} \hlopt{+} \hlkwd{xlim}\hlstd{(}\hlnum{4}\hlstd{,} \hlnum{10}\hlstd{)} \hlopt{+} \hlkwd{ylim}\hlstd{(}\hlnum{4}\hlstd{,} \hlnum{10}\hlstd{)}
\hlstd{d} \hlopt{+} \hlkwd{geom_bin2d}\hlstd{()}
\end{alltt}


{\ttfamily\noindent\color{warningcolor}{\#\# Warning: Removed 478 rows containing non-finite values (stat\_bin2d).}}\end{kframe}
\includegraphics[width=\maxwidth]{figure/021-ggplot2-geoms-geom_bin2d-1} 
\begin{kframe}\begin{alltt}
\hlcom{# You can control the size of the bins by specifying the number of}
\hlcom{# bins in each direction:}
\hlstd{d} \hlopt{+} \hlkwd{geom_bin2d}\hlstd{(}\hlkwc{bins} \hlstd{=} \hlnum{10}\hlstd{)}
\end{alltt}


{\ttfamily\noindent\color{warningcolor}{\#\# Warning: Removed 478 rows containing non-finite values (stat\_bin2d).}}

{\ttfamily\noindent\color{warningcolor}{\#\# Warning: Removed 4 rows containing missing values (geom\_tile).}}\end{kframe}
\includegraphics[width=\maxwidth]{figure/021-ggplot2-geoms-geom_bin2d-2} 
\begin{kframe}\begin{alltt}
\hlstd{d} \hlopt{+} \hlkwd{geom_bin2d}\hlstd{(}\hlkwc{bins} \hlstd{=} \hlnum{30}\hlstd{)}
\end{alltt}


{\ttfamily\noindent\color{warningcolor}{\#\# Warning: Removed 478 rows containing non-finite values (stat\_bin2d).}}\end{kframe}
\includegraphics[width=\maxwidth]{figure/021-ggplot2-geoms-geom_bin2d-3} 
\begin{kframe}\begin{alltt}
\hlcom{# Or by specifying the width of the bins}
\hlstd{d} \hlopt{+} \hlkwd{geom_bin2d}\hlstd{(}\hlkwc{binwidth} \hlstd{=} \hlkwd{c}\hlstd{(}\hlnum{0.1}\hlstd{,} \hlnum{0.1}\hlstd{))}
\end{alltt}


{\ttfamily\noindent\color{warningcolor}{\#\# Warning: Removed 478 rows containing non-finite values (stat\_bin2d).}}\end{kframe}
\includegraphics[width=\maxwidth]{figure/021-ggplot2-geoms-geom_bin2d-4} 

\end{knitrout}


\section{geom\_blank}

\begin{knitrout}
\definecolor{shadecolor}{rgb}{0.969, 0.969, 0.969}\color{fgcolor}\begin{kframe}
\begin{alltt}
\hlcom{### Name: geom_blank}
\hlcom{### Title: Blank, draws nothing.}
\hlcom{### Aliases: geom_blank}

\hlcom{### ** Examples}

\hlkwd{ggplot}\hlstd{(mtcars,} \hlkwd{aes}\hlstd{(wt, mpg))} \hlopt{+} \hlkwd{geom_blank}\hlstd{()}
\end{alltt}
\end{kframe}
\includegraphics[width=\maxwidth]{figure/021-ggplot2-geoms-geom_blank-1} 
\begin{kframe}\begin{alltt}
\hlcom{# Nothing to see here!}

\hlcom{# Take the following scatter plot}
\hlstd{a} \hlkwb{<-} \hlkwd{ggplot}\hlstd{(mtcars,} \hlkwd{aes}\hlstd{(}\hlkwc{x} \hlstd{= wt,} \hlkwc{y} \hlstd{= mpg), .} \hlopt{~} \hlstd{cyl)} \hlopt{+} \hlkwd{geom_point}\hlstd{()}
\hlcom{# Add to that some lines with geom_abline()}
\hlstd{df} \hlkwb{<-} \hlkwd{data.frame}\hlstd{(}\hlkwc{a} \hlstd{=} \hlkwd{rnorm}\hlstd{(}\hlnum{10}\hlstd{,} \hlnum{25}\hlstd{),} \hlkwc{b} \hlstd{=} \hlkwd{rnorm}\hlstd{(}\hlnum{10}\hlstd{,} \hlnum{0}\hlstd{))}
\hlstd{a} \hlopt{+} \hlkwd{geom_abline}\hlstd{(}\hlkwd{aes}\hlstd{(}\hlkwc{intercept} \hlstd{= a,} \hlkwc{slope} \hlstd{= b),} \hlkwc{data} \hlstd{= df)}
\end{alltt}
\end{kframe}
\includegraphics[width=\maxwidth]{figure/021-ggplot2-geoms-geom_blank-2} 
\begin{kframe}\begin{alltt}
\hlcom{# Suppose you then wanted to remove the geom_point layer}
\hlcom{# If you just remove geom_point, you will get an error}
\hlstd{b} \hlkwb{<-} \hlkwd{ggplot}\hlstd{(mtcars,} \hlkwd{aes}\hlstd{(}\hlkwc{x} \hlstd{= wt,} \hlkwc{y} \hlstd{= mpg))}
\hlcom{## Not run: b + geom_abline(aes(intercept = a, slope = b), data = df)}
\hlcom{# Switching to geom_blank() gets the desired plot}
\hlstd{c} \hlkwb{<-} \hlkwd{ggplot}\hlstd{(mtcars,} \hlkwd{aes}\hlstd{(}\hlkwc{x} \hlstd{= wt,} \hlkwc{y} \hlstd{= mpg))} \hlopt{+} \hlkwd{geom_blank}\hlstd{()}
\hlstd{c} \hlopt{+} \hlkwd{geom_abline}\hlstd{(}\hlkwd{aes}\hlstd{(}\hlkwc{intercept} \hlstd{= a,} \hlkwc{slope} \hlstd{= b),} \hlkwc{data} \hlstd{= df)}
\end{alltt}
\end{kframe}
\includegraphics[width=\maxwidth]{figure/021-ggplot2-geoms-geom_blank-3} 

\end{knitrout}


\section{geom\_boxplot}

\begin{knitrout}
\definecolor{shadecolor}{rgb}{0.969, 0.969, 0.969}\color{fgcolor}\begin{kframe}
\begin{alltt}
\hlcom{### Name: geom_boxplot}
\hlcom{### Title: Box and whiskers plot.}
\hlcom{### Aliases: geom_boxplot stat_boxplot}

\hlcom{### ** Examples}

\hlstd{p} \hlkwb{<-} \hlkwd{ggplot}\hlstd{(mpg,} \hlkwd{aes}\hlstd{(class, hwy))}
\hlstd{p} \hlopt{+} \hlkwd{geom_boxplot}\hlstd{()}
\end{alltt}
\end{kframe}
\includegraphics[width=\maxwidth]{figure/021-ggplot2-geoms-geom_boxplot-1} 
\begin{kframe}\begin{alltt}
\hlstd{p} \hlopt{+} \hlkwd{geom_boxplot}\hlstd{()} \hlopt{+} \hlkwd{geom_jitter}\hlstd{(}\hlkwc{width} \hlstd{=} \hlnum{0.2}\hlstd{)}
\end{alltt}
\end{kframe}
\includegraphics[width=\maxwidth]{figure/021-ggplot2-geoms-geom_boxplot-2} 
\begin{kframe}\begin{alltt}
\hlstd{p} \hlopt{+} \hlkwd{geom_boxplot}\hlstd{()} \hlopt{+} \hlkwd{coord_flip}\hlstd{()}
\end{alltt}
\end{kframe}
\includegraphics[width=\maxwidth]{figure/021-ggplot2-geoms-geom_boxplot-3} 
\begin{kframe}\begin{alltt}
\hlstd{p} \hlopt{+} \hlkwd{geom_boxplot}\hlstd{(}\hlkwc{notch} \hlstd{=} \hlnum{TRUE}\hlstd{)}
\end{alltt}


{\ttfamily\noindent\itshape\color{messagecolor}{\#\# notch went outside hinges. Try setting notch=FALSE.\\\#\# notch went outside hinges. Try setting notch=FALSE.}}\end{kframe}
\includegraphics[width=\maxwidth]{figure/021-ggplot2-geoms-geom_boxplot-4} 
\begin{kframe}\begin{alltt}
\hlstd{p} \hlopt{+} \hlkwd{geom_boxplot}\hlstd{(}\hlkwc{varwidth} \hlstd{=} \hlnum{TRUE}\hlstd{)}
\end{alltt}
\end{kframe}
\includegraphics[width=\maxwidth]{figure/021-ggplot2-geoms-geom_boxplot-5} 
\begin{kframe}\begin{alltt}
\hlstd{p} \hlopt{+} \hlkwd{geom_boxplot}\hlstd{(}\hlkwc{fill} \hlstd{=} \hlstr{"white"}\hlstd{,} \hlkwc{colour} \hlstd{=} \hlstr{"#3366FF"}\hlstd{)}
\end{alltt}
\end{kframe}
\includegraphics[width=\maxwidth]{figure/021-ggplot2-geoms-geom_boxplot-6} 
\begin{kframe}\begin{alltt}
\hlcom{# By default, outlier points match the colour of the box. Use}
\hlcom{# outlier.colour to override}
\hlstd{p} \hlopt{+} \hlkwd{geom_boxplot}\hlstd{(}\hlkwc{outlier.colour} \hlstd{=} \hlstr{"red"}\hlstd{,} \hlkwc{outlier.shape} \hlstd{=} \hlnum{1}\hlstd{)}
\end{alltt}
\end{kframe}
\includegraphics[width=\maxwidth]{figure/021-ggplot2-geoms-geom_boxplot-7} 
\begin{kframe}\begin{alltt}
\hlcom{# Boxplots are automatically dodged when any aesthetic is a factor}
\hlstd{p} \hlopt{+} \hlkwd{geom_boxplot}\hlstd{(}\hlkwd{aes}\hlstd{(}\hlkwc{colour} \hlstd{= drv))}
\end{alltt}
\end{kframe}
\includegraphics[width=\maxwidth]{figure/021-ggplot2-geoms-geom_boxplot-8} 
\begin{kframe}\begin{alltt}
\hlcom{# You can also use boxplots with continuous x, as long as you supply}
\hlcom{# a grouping variable. cut_width is particularly useful}
\hlkwd{ggplot}\hlstd{(diamonds,} \hlkwd{aes}\hlstd{(carat, price))} \hlopt{+}
  \hlkwd{geom_boxplot}\hlstd{()}
\end{alltt}


{\ttfamily\noindent\color{warningcolor}{\#\# Warning: Continuous x aesthetic -- did you forget aes(group=...)?}}\end{kframe}
\includegraphics[width=\maxwidth]{figure/021-ggplot2-geoms-geom_boxplot-9} 
\begin{kframe}\begin{alltt}
\hlkwd{ggplot}\hlstd{(diamonds,} \hlkwd{aes}\hlstd{(carat, price))} \hlopt{+}
  \hlkwd{geom_boxplot}\hlstd{(}\hlkwd{aes}\hlstd{(}\hlkwc{group} \hlstd{=} \hlkwd{cut_width}\hlstd{(carat,} \hlnum{0.25}\hlstd{)))}
\end{alltt}
\end{kframe}
\includegraphics[width=\maxwidth]{figure/021-ggplot2-geoms-geom_boxplot-10} 
\begin{kframe}\begin{alltt}
\hlcom{## No test: }
\hlcom{# It's possible to draw a boxplot with your own computations if you}
\hlcom{# use stat = "identity":}
\hlstd{y} \hlkwb{<-} \hlkwd{rnorm}\hlstd{(}\hlnum{100}\hlstd{)}
\hlstd{df} \hlkwb{<-} \hlkwd{data.frame}\hlstd{(}
  \hlkwc{x} \hlstd{=} \hlnum{1}\hlstd{,}
  \hlkwc{y0} \hlstd{=} \hlkwd{min}\hlstd{(y),}
  \hlkwc{y25} \hlstd{=} \hlkwd{quantile}\hlstd{(y,} \hlnum{0.25}\hlstd{),}
  \hlkwc{y50} \hlstd{=} \hlkwd{median}\hlstd{(y),}
  \hlkwc{y75} \hlstd{=} \hlkwd{quantile}\hlstd{(y,} \hlnum{0.75}\hlstd{),}
  \hlkwc{y100} \hlstd{=} \hlkwd{max}\hlstd{(y)}
\hlstd{)}
\hlkwd{ggplot}\hlstd{(df,} \hlkwd{aes}\hlstd{(x))} \hlopt{+}
  \hlkwd{geom_boxplot}\hlstd{(}
   \hlkwd{aes}\hlstd{(}\hlkwc{ymin} \hlstd{= y0,} \hlkwc{lower} \hlstd{= y25,} \hlkwc{middle} \hlstd{= y50,} \hlkwc{upper} \hlstd{= y75,} \hlkwc{ymax} \hlstd{= y100),}
   \hlkwc{stat} \hlstd{=} \hlstr{"identity"}
 \hlstd{)}
\end{alltt}
\end{kframe}
\includegraphics[width=\maxwidth]{figure/021-ggplot2-geoms-geom_boxplot-11} 
\begin{kframe}\begin{alltt}
\hlcom{## End(No test)}
\end{alltt}
\end{kframe}
\end{knitrout}


\section{geom\_contour}

\begin{knitrout}
\definecolor{shadecolor}{rgb}{0.969, 0.969, 0.969}\color{fgcolor}\begin{kframe}
\begin{alltt}
\hlcom{### Name: geom_contour}
\hlcom{### Title: Display contours of a 3d surface in 2d.}
\hlcom{### Aliases: geom_contour stat_contour}

\hlcom{### ** Examples}

\hlcom{#' # Basic plot}
\hlstd{v} \hlkwb{<-} \hlkwd{ggplot}\hlstd{(faithfuld,} \hlkwd{aes}\hlstd{(waiting, eruptions,} \hlkwc{z} \hlstd{= density))}
\hlstd{v} \hlopt{+} \hlkwd{geom_contour}\hlstd{()}
\end{alltt}
\end{kframe}
\includegraphics[width=\maxwidth]{figure/021-ggplot2-geoms-geom_contour-1} 
\begin{kframe}\begin{alltt}
\hlcom{# Or compute from raw data}
\hlkwd{ggplot}\hlstd{(faithful,} \hlkwd{aes}\hlstd{(waiting, eruptions))} \hlopt{+}
  \hlkwd{geom_density_2d}\hlstd{()}
\end{alltt}
\end{kframe}
\includegraphics[width=\maxwidth]{figure/021-ggplot2-geoms-geom_contour-2} 
\begin{kframe}\begin{alltt}
\hlcom{## No test: }
\hlcom{# Setting bins creates evenly spaced contours in the range of the data}
\hlstd{v} \hlopt{+} \hlkwd{geom_contour}\hlstd{(}\hlkwc{bins} \hlstd{=} \hlnum{2}\hlstd{)}
\end{alltt}
\end{kframe}
\includegraphics[width=\maxwidth]{figure/021-ggplot2-geoms-geom_contour-3} 
\begin{kframe}\begin{alltt}
\hlstd{v} \hlopt{+} \hlkwd{geom_contour}\hlstd{(}\hlkwc{bins} \hlstd{=} \hlnum{10}\hlstd{)}
\end{alltt}
\end{kframe}
\includegraphics[width=\maxwidth]{figure/021-ggplot2-geoms-geom_contour-4} 
\begin{kframe}\begin{alltt}
\hlcom{# Setting binwidth does the same thing, parameterised by the distance}
\hlcom{# between contours}
\hlstd{v} \hlopt{+} \hlkwd{geom_contour}\hlstd{(}\hlkwc{binwidth} \hlstd{=} \hlnum{0.01}\hlstd{)}
\end{alltt}
\end{kframe}
\includegraphics[width=\maxwidth]{figure/021-ggplot2-geoms-geom_contour-5} 
\begin{kframe}\begin{alltt}
\hlstd{v} \hlopt{+} \hlkwd{geom_contour}\hlstd{(}\hlkwc{binwidth} \hlstd{=} \hlnum{0.001}\hlstd{)}
\end{alltt}
\end{kframe}
\includegraphics[width=\maxwidth]{figure/021-ggplot2-geoms-geom_contour-6} 
\begin{kframe}\begin{alltt}
\hlcom{# Other parameters}
\hlstd{v} \hlopt{+} \hlkwd{geom_contour}\hlstd{(}\hlkwd{aes}\hlstd{(}\hlkwc{colour} \hlstd{= ..level..))}
\end{alltt}
\end{kframe}
\includegraphics[width=\maxwidth]{figure/021-ggplot2-geoms-geom_contour-7} 
\begin{kframe}\begin{alltt}
\hlstd{v} \hlopt{+} \hlkwd{geom_contour}\hlstd{(}\hlkwc{colour} \hlstd{=} \hlstr{"red"}\hlstd{)}
\end{alltt}
\end{kframe}
\includegraphics[width=\maxwidth]{figure/021-ggplot2-geoms-geom_contour-8} 
\begin{kframe}\begin{alltt}
\hlstd{v} \hlopt{+} \hlkwd{geom_raster}\hlstd{(}\hlkwd{aes}\hlstd{(}\hlkwc{fill} \hlstd{= density))} \hlopt{+}
  \hlkwd{geom_contour}\hlstd{(}\hlkwc{colour} \hlstd{=} \hlstr{"white"}\hlstd{)}
\end{alltt}
\end{kframe}
\includegraphics[width=\maxwidth]{figure/021-ggplot2-geoms-geom_contour-9} 
\begin{kframe}\begin{alltt}
\hlcom{## End(No test)}
\end{alltt}
\end{kframe}
\end{knitrout}


\section{geom\_count}

\begin{knitrout}
\definecolor{shadecolor}{rgb}{0.969, 0.969, 0.969}\color{fgcolor}\begin{kframe}
\begin{alltt}
\hlcom{### Name: geom_count}
\hlcom{### Title: Count the number of observations at each location.}
\hlcom{### Aliases: geom_count stat_sum}

\hlcom{### ** Examples}

\hlkwd{ggplot}\hlstd{(mpg,} \hlkwd{aes}\hlstd{(cty, hwy))} \hlopt{+}
 \hlkwd{geom_point}\hlstd{()}
\end{alltt}
\end{kframe}
\includegraphics[width=\maxwidth]{figure/021-ggplot2-geoms-geom_count-1} 
\begin{kframe}\begin{alltt}
\hlkwd{ggplot}\hlstd{(mpg,} \hlkwd{aes}\hlstd{(cty, hwy))} \hlopt{+}
 \hlkwd{geom_count}\hlstd{()}
\end{alltt}
\end{kframe}
\includegraphics[width=\maxwidth]{figure/021-ggplot2-geoms-geom_count-2} 
\begin{kframe}\begin{alltt}
\hlcom{# Best used in conjunction with scale_size_area which ensures that}
\hlcom{# counts of zero would be given size 0. Doesn't make much different}
\hlcom{# here because the smallest count is already close to 0.}
\hlkwd{ggplot}\hlstd{(mpg,} \hlkwd{aes}\hlstd{(cty, hwy))} \hlopt{+}
 \hlkwd{geom_count}\hlstd{()}
\end{alltt}
\end{kframe}
\includegraphics[width=\maxwidth]{figure/021-ggplot2-geoms-geom_count-3} 
\begin{kframe}\begin{alltt}
 \hlkwd{scale_size_area}\hlstd{()}
\end{alltt}
\begin{verbatim}
## <ScaleContinuous>
##  Range:  
##  Limits:    0 --    1
\end{verbatim}
\begin{alltt}
\hlcom{# Display proportions instead of counts -------------------------------------}
\hlcom{# By default, all categorical variables in the plot form the groups.}
\hlcom{# Specifying geom_count without a group identifier leads to a plot which is}
\hlcom{# not useful:}
\hlstd{d} \hlkwb{<-} \hlkwd{ggplot}\hlstd{(diamonds,} \hlkwd{aes}\hlstd{(}\hlkwc{x} \hlstd{= cut,} \hlkwc{y} \hlstd{= clarity))}
\hlstd{d} \hlopt{+} \hlkwd{geom_count}\hlstd{(}\hlkwd{aes}\hlstd{(}\hlkwc{size} \hlstd{= ..prop..))}
\end{alltt}
\end{kframe}
\includegraphics[width=\maxwidth]{figure/021-ggplot2-geoms-geom_count-4} 
\begin{kframe}\begin{alltt}
\hlcom{# To correct this problem and achieve a more desirable plot, we need}
\hlcom{# to specify which group the proportion is to be calculated over.}
\hlstd{d} \hlopt{+} \hlkwd{geom_count}\hlstd{(}\hlkwd{aes}\hlstd{(}\hlkwc{size} \hlstd{= ..prop..,} \hlkwc{group} \hlstd{=} \hlnum{1}\hlstd{))} \hlopt{+}
  \hlkwd{scale_size_area}\hlstd{(}\hlkwc{max_size} \hlstd{=} \hlnum{10}\hlstd{)}
\end{alltt}
\end{kframe}
\includegraphics[width=\maxwidth]{figure/021-ggplot2-geoms-geom_count-5} 
\begin{kframe}\begin{alltt}
\hlcom{# Or group by x/y variables to have rows/columns sum to 1.}
\hlstd{d} \hlopt{+} \hlkwd{geom_count}\hlstd{(}\hlkwd{aes}\hlstd{(}\hlkwc{size} \hlstd{= ..prop..,} \hlkwc{group} \hlstd{= cut))} \hlopt{+}
  \hlkwd{scale_size_area}\hlstd{(}\hlkwc{max_size} \hlstd{=} \hlnum{10}\hlstd{)}
\end{alltt}
\end{kframe}
\includegraphics[width=\maxwidth]{figure/021-ggplot2-geoms-geom_count-6} 
\begin{kframe}\begin{alltt}
\hlstd{d} \hlopt{+} \hlkwd{geom_count}\hlstd{(}\hlkwd{aes}\hlstd{(}\hlkwc{size} \hlstd{= ..prop..,} \hlkwc{group} \hlstd{= clarity))} \hlopt{+}
  \hlkwd{scale_size_area}\hlstd{(}\hlkwc{max_size} \hlstd{=} \hlnum{10}\hlstd{)}
\end{alltt}
\end{kframe}
\includegraphics[width=\maxwidth]{figure/021-ggplot2-geoms-geom_count-7} 

\end{knitrout}


\section{geom\_crossbar}

\begin{knitrout}
\definecolor{shadecolor}{rgb}{0.969, 0.969, 0.969}\color{fgcolor}\begin{kframe}
\begin{alltt}
\hlcom{### Name: geom_crossbar}
\hlcom{### Title: Vertical intervals: lines, crossbars & errorbars.}
\hlcom{### Aliases: geom_crossbar geom_errorbar geom_linerange geom_pointrange}

\hlcom{### ** Examples}

\hlcom{#' # Create a simple example dataset}
\hlstd{df} \hlkwb{<-} \hlkwd{data.frame}\hlstd{(}
  \hlkwc{trt} \hlstd{=} \hlkwd{factor}\hlstd{(}\hlkwd{c}\hlstd{(}\hlnum{1}\hlstd{,} \hlnum{1}\hlstd{,} \hlnum{2}\hlstd{,} \hlnum{2}\hlstd{)),}
  \hlkwc{resp} \hlstd{=} \hlkwd{c}\hlstd{(}\hlnum{1}\hlstd{,} \hlnum{5}\hlstd{,} \hlnum{3}\hlstd{,} \hlnum{4}\hlstd{),}
  \hlkwc{group} \hlstd{=} \hlkwd{factor}\hlstd{(}\hlkwd{c}\hlstd{(}\hlnum{1}\hlstd{,} \hlnum{2}\hlstd{,} \hlnum{1}\hlstd{,} \hlnum{2}\hlstd{)),}
  \hlkwc{upper} \hlstd{=} \hlkwd{c}\hlstd{(}\hlnum{1.1}\hlstd{,} \hlnum{5.3}\hlstd{,} \hlnum{3.3}\hlstd{,} \hlnum{4.2}\hlstd{),}
  \hlkwc{lower} \hlstd{=} \hlkwd{c}\hlstd{(}\hlnum{0.8}\hlstd{,} \hlnum{4.6}\hlstd{,} \hlnum{2.4}\hlstd{,} \hlnum{3.6}\hlstd{)}
\hlstd{)}

\hlstd{p} \hlkwb{<-} \hlkwd{ggplot}\hlstd{(df,} \hlkwd{aes}\hlstd{(trt, resp,} \hlkwc{colour} \hlstd{= group))}
\hlstd{p} \hlopt{+} \hlkwd{geom_linerange}\hlstd{(}\hlkwd{aes}\hlstd{(}\hlkwc{ymin} \hlstd{= lower,} \hlkwc{ymax} \hlstd{= upper))}
\end{alltt}
\end{kframe}
\includegraphics[width=\maxwidth]{figure/021-ggplot2-geoms-geom_crossbar-1} 
\begin{kframe}\begin{alltt}
\hlstd{p} \hlopt{+} \hlkwd{geom_pointrange}\hlstd{(}\hlkwd{aes}\hlstd{(}\hlkwc{ymin} \hlstd{= lower,} \hlkwc{ymax} \hlstd{= upper))}
\end{alltt}
\end{kframe}
\includegraphics[width=\maxwidth]{figure/021-ggplot2-geoms-geom_crossbar-2} 
\begin{kframe}\begin{alltt}
\hlstd{p} \hlopt{+} \hlkwd{geom_crossbar}\hlstd{(}\hlkwd{aes}\hlstd{(}\hlkwc{ymin} \hlstd{= lower,} \hlkwc{ymax} \hlstd{= upper),} \hlkwc{width} \hlstd{=} \hlnum{0.2}\hlstd{)}
\end{alltt}
\end{kframe}
\includegraphics[width=\maxwidth]{figure/021-ggplot2-geoms-geom_crossbar-3} 
\begin{kframe}\begin{alltt}
\hlstd{p} \hlopt{+} \hlkwd{geom_errorbar}\hlstd{(}\hlkwd{aes}\hlstd{(}\hlkwc{ymin} \hlstd{= lower,} \hlkwc{ymax} \hlstd{= upper),} \hlkwc{width} \hlstd{=} \hlnum{0.2}\hlstd{)}
\end{alltt}
\end{kframe}
\includegraphics[width=\maxwidth]{figure/021-ggplot2-geoms-geom_crossbar-4} 
\begin{kframe}\begin{alltt}
\hlcom{# Draw lines connecting group means}
\hlstd{p} \hlopt{+}
  \hlkwd{geom_line}\hlstd{(}\hlkwd{aes}\hlstd{(}\hlkwc{group} \hlstd{= group))} \hlopt{+}
  \hlkwd{geom_errorbar}\hlstd{(}\hlkwd{aes}\hlstd{(}\hlkwc{ymin} \hlstd{= lower,} \hlkwc{ymax} \hlstd{= upper),} \hlkwc{width} \hlstd{=} \hlnum{0.2}\hlstd{)}
\end{alltt}
\end{kframe}
\includegraphics[width=\maxwidth]{figure/021-ggplot2-geoms-geom_crossbar-5} 
\begin{kframe}\begin{alltt}
\hlcom{# If you want to dodge bars and errorbars, you need to manually}
\hlcom{# specify the dodge width}
\hlstd{p} \hlkwb{<-} \hlkwd{ggplot}\hlstd{(df,} \hlkwd{aes}\hlstd{(trt, resp,} \hlkwc{fill} \hlstd{= group))}
\hlstd{p} \hlopt{+}
 \hlkwd{geom_bar}\hlstd{(}\hlkwc{position} \hlstd{=} \hlstr{"dodge"}\hlstd{,} \hlkwc{stat} \hlstd{=} \hlstr{"identity"}\hlstd{)} \hlopt{+}
 \hlkwd{geom_errorbar}\hlstd{(}\hlkwd{aes}\hlstd{(}\hlkwc{ymin} \hlstd{= lower,} \hlkwc{ymax} \hlstd{= upper),} \hlkwc{position} \hlstd{=} \hlstr{"dodge"}\hlstd{,} \hlkwc{width} \hlstd{=} \hlnum{0.25}\hlstd{)}
\end{alltt}
\end{kframe}
\includegraphics[width=\maxwidth]{figure/021-ggplot2-geoms-geom_crossbar-6} 
\begin{kframe}\begin{alltt}
\hlcom{# Because the bars and errorbars have different widths}
\hlcom{# we need to specify how wide the objects we are dodging are}
\hlstd{dodge} \hlkwb{<-} \hlkwd{position_dodge}\hlstd{(}\hlkwc{width}\hlstd{=}\hlnum{0.9}\hlstd{)}
\hlstd{p} \hlopt{+}
  \hlkwd{geom_bar}\hlstd{(}\hlkwc{position} \hlstd{= dodge,} \hlkwc{stat} \hlstd{=} \hlstr{"identity"}\hlstd{)} \hlopt{+}
  \hlkwd{geom_errorbar}\hlstd{(}\hlkwd{aes}\hlstd{(}\hlkwc{ymin} \hlstd{= lower,} \hlkwc{ymax} \hlstd{= upper),} \hlkwc{position} \hlstd{= dodge,} \hlkwc{width} \hlstd{=} \hlnum{0.25}\hlstd{)}
\end{alltt}
\end{kframe}
\includegraphics[width=\maxwidth]{figure/021-ggplot2-geoms-geom_crossbar-7} 

\end{knitrout}


\section{geom\_curve}

\begin{knitrout}
\definecolor{shadecolor}{rgb}{0.969, 0.969, 0.969}\color{fgcolor}\begin{kframe}
\begin{alltt}
\hlcom{### Name: geom_segment}
\hlcom{### Title: Line segments and curves.}
\hlcom{### Aliases: geom_curve geom_segment}

\hlcom{### ** Examples}

\hlstd{b} \hlkwb{<-} \hlkwd{ggplot}\hlstd{(mtcars,} \hlkwd{aes}\hlstd{(wt, mpg))} \hlopt{+}
  \hlkwd{geom_point}\hlstd{()}

\hlstd{df} \hlkwb{<-} \hlkwd{data.frame}\hlstd{(}\hlkwc{x1} \hlstd{=} \hlnum{2.62}\hlstd{,} \hlkwc{x2} \hlstd{=} \hlnum{3.57}\hlstd{,} \hlkwc{y1} \hlstd{=} \hlnum{21.0}\hlstd{,} \hlkwc{y2} \hlstd{=} \hlnum{15.0}\hlstd{)}
\hlstd{b} \hlopt{+}
 \hlkwd{geom_curve}\hlstd{(}\hlkwd{aes}\hlstd{(}\hlkwc{x} \hlstd{= x1,} \hlkwc{y} \hlstd{= y1,} \hlkwc{xend} \hlstd{= x2,} \hlkwc{yend} \hlstd{= y2,} \hlkwc{colour} \hlstd{=} \hlstr{"curve"}\hlstd{),} \hlkwc{data} \hlstd{= df)} \hlopt{+}
 \hlkwd{geom_segment}\hlstd{(}\hlkwd{aes}\hlstd{(}\hlkwc{x} \hlstd{= x1,} \hlkwc{y} \hlstd{= y1,} \hlkwc{xend} \hlstd{= x2,} \hlkwc{yend} \hlstd{= y2,} \hlkwc{colour} \hlstd{=} \hlstr{"segment"}\hlstd{),} \hlkwc{data} \hlstd{= df)}
\end{alltt}
\end{kframe}
\includegraphics[width=\maxwidth]{figure/021-ggplot2-geoms-geom_curve-1} 
\begin{kframe}\begin{alltt}
\hlstd{b} \hlopt{+} \hlkwd{geom_curve}\hlstd{(}\hlkwd{aes}\hlstd{(}\hlkwc{x} \hlstd{= x1,} \hlkwc{y} \hlstd{= y1,} \hlkwc{xend} \hlstd{= x2,} \hlkwc{yend} \hlstd{= y2),} \hlkwc{data} \hlstd{= df,} \hlkwc{curvature} \hlstd{=} \hlopt{-}\hlnum{0.2}\hlstd{)}
\end{alltt}
\end{kframe}
\includegraphics[width=\maxwidth]{figure/021-ggplot2-geoms-geom_curve-2} 
\begin{kframe}\begin{alltt}
\hlstd{b} \hlopt{+} \hlkwd{geom_curve}\hlstd{(}\hlkwd{aes}\hlstd{(}\hlkwc{x} \hlstd{= x1,} \hlkwc{y} \hlstd{= y1,} \hlkwc{xend} \hlstd{= x2,} \hlkwc{yend} \hlstd{= y2),} \hlkwc{data} \hlstd{= df,} \hlkwc{curvature} \hlstd{=} \hlnum{1}\hlstd{)}
\end{alltt}
\end{kframe}
\includegraphics[width=\maxwidth]{figure/021-ggplot2-geoms-geom_curve-3} 
\begin{kframe}\begin{alltt}
\hlstd{b} \hlopt{+} \hlkwd{geom_curve}\hlstd{(}
  \hlkwd{aes}\hlstd{(}\hlkwc{x} \hlstd{= x1,} \hlkwc{y} \hlstd{= y1,} \hlkwc{xend} \hlstd{= x2,} \hlkwc{yend} \hlstd{= y2),}
  \hlkwc{data} \hlstd{= df,}
  \hlkwc{arrow} \hlstd{=} \hlkwd{arrow}\hlstd{(}\hlkwc{length} \hlstd{=} \hlkwd{unit}\hlstd{(}\hlnum{0.03}\hlstd{,} \hlstr{"npc"}\hlstd{))}
\hlstd{)}
\end{alltt}
\end{kframe}
\includegraphics[width=\maxwidth]{figure/021-ggplot2-geoms-geom_curve-4} 
\begin{kframe}\begin{alltt}
\hlkwd{ggplot}\hlstd{(seals,} \hlkwd{aes}\hlstd{(long, lat))} \hlopt{+}
  \hlkwd{geom_segment}\hlstd{(}\hlkwd{aes}\hlstd{(}\hlkwc{xend} \hlstd{= long} \hlopt{+} \hlstd{delta_long,} \hlkwc{yend} \hlstd{= lat} \hlopt{+} \hlstd{delta_lat),}
    \hlkwc{arrow} \hlstd{=} \hlkwd{arrow}\hlstd{(}\hlkwc{length} \hlstd{=} \hlkwd{unit}\hlstd{(}\hlnum{0.1}\hlstd{,}\hlstr{"cm"}\hlstd{)))} \hlopt{+}
  \hlkwd{borders}\hlstd{(}\hlstr{"state"}\hlstd{)}
\end{alltt}


{\ttfamily\noindent\itshape\color{messagecolor}{\#\# \\\#\#\ \ \# ATTENTION: maps v3.0 has an updated 'world' map.\ \ \ \ \ \ \ \ \#\\\#\#\ \ \# Many country borders and names have changed since 1990. \#\\\#\#\ \ \# Type '?world' or 'news(package="{}maps"{})'. See README\_v3. \#}}\end{kframe}
\includegraphics[width=\maxwidth]{figure/021-ggplot2-geoms-geom_curve-5} 
\begin{kframe}\begin{alltt}
\hlcom{# You can also use geom_segment to recreate plot(type = "h") :}
\hlstd{counts} \hlkwb{<-} \hlkwd{as.data.frame}\hlstd{(}\hlkwd{table}\hlstd{(}\hlkwc{x} \hlstd{=} \hlkwd{rpois}\hlstd{(}\hlnum{100}\hlstd{,}\hlnum{5}\hlstd{)))}
\hlstd{counts}\hlopt{$}\hlstd{x} \hlkwb{<-} \hlkwd{as.numeric}\hlstd{(}\hlkwd{as.character}\hlstd{(counts}\hlopt{$}\hlstd{x))}
\hlkwd{with}\hlstd{(counts,} \hlkwd{plot}\hlstd{(x, Freq,} \hlkwc{type} \hlstd{=} \hlstr{"h"}\hlstd{,} \hlkwc{lwd} \hlstd{=} \hlnum{10}\hlstd{))}
\end{alltt}
\end{kframe}
\includegraphics[width=\maxwidth]{figure/021-ggplot2-geoms-geom_curve-6} 
\begin{kframe}\begin{alltt}
\hlkwd{ggplot}\hlstd{(counts,} \hlkwd{aes}\hlstd{(x, Freq))} \hlopt{+}
  \hlkwd{geom_segment}\hlstd{(}\hlkwd{aes}\hlstd{(}\hlkwc{xend} \hlstd{= x,} \hlkwc{yend} \hlstd{=} \hlnum{0}\hlstd{),} \hlkwc{size} \hlstd{=} \hlnum{10}\hlstd{,} \hlkwc{lineend} \hlstd{=} \hlstr{"butt"}\hlstd{)}
\end{alltt}
\end{kframe}
\includegraphics[width=\maxwidth]{figure/021-ggplot2-geoms-geom_curve-7} 

\end{knitrout}


\section{geom\_density}

\begin{knitrout}
\definecolor{shadecolor}{rgb}{0.969, 0.969, 0.969}\color{fgcolor}\begin{kframe}
\begin{alltt}
\hlcom{### Name: geom_density}
\hlcom{### Title: Display a smooth density estimate.}
\hlcom{### Aliases: geom_density stat_density}

\hlcom{### ** Examples}

\hlkwd{ggplot}\hlstd{(diamonds,} \hlkwd{aes}\hlstd{(carat))} \hlopt{+}
  \hlkwd{geom_density}\hlstd{()}
\end{alltt}
\end{kframe}
\includegraphics[width=\maxwidth]{figure/021-ggplot2-geoms-geom_density-1} 
\begin{kframe}\begin{alltt}
\hlkwd{ggplot}\hlstd{(diamonds,} \hlkwd{aes}\hlstd{(carat))} \hlopt{+}
  \hlkwd{geom_density}\hlstd{(}\hlkwc{adjust} \hlstd{=} \hlnum{1}\hlopt{/}\hlnum{5}\hlstd{)}
\end{alltt}
\end{kframe}
\includegraphics[width=\maxwidth]{figure/021-ggplot2-geoms-geom_density-2} 
\begin{kframe}\begin{alltt}
\hlkwd{ggplot}\hlstd{(diamonds,} \hlkwd{aes}\hlstd{(carat))} \hlopt{+}
  \hlkwd{geom_density}\hlstd{(}\hlkwc{adjust} \hlstd{=} \hlnum{5}\hlstd{)}
\end{alltt}
\end{kframe}
\includegraphics[width=\maxwidth]{figure/021-ggplot2-geoms-geom_density-3} 
\begin{kframe}\begin{alltt}
\hlkwd{ggplot}\hlstd{(diamonds,} \hlkwd{aes}\hlstd{(depth,} \hlkwc{colour} \hlstd{= cut))} \hlopt{+}
  \hlkwd{geom_density}\hlstd{()} \hlopt{+}
  \hlkwd{xlim}\hlstd{(}\hlnum{55}\hlstd{,} \hlnum{70}\hlstd{)}
\end{alltt}


{\ttfamily\noindent\color{warningcolor}{\#\# Warning: Removed 45 rows containing non-finite values (stat\_density).}}\end{kframe}
\includegraphics[width=\maxwidth]{figure/021-ggplot2-geoms-geom_density-4} 
\begin{kframe}\begin{alltt}
\hlkwd{ggplot}\hlstd{(diamonds,} \hlkwd{aes}\hlstd{(depth,} \hlkwc{fill} \hlstd{= cut,} \hlkwc{colour} \hlstd{= cut))} \hlopt{+}
  \hlkwd{geom_density}\hlstd{(}\hlkwc{alpha} \hlstd{=} \hlnum{0.1}\hlstd{)} \hlopt{+}
  \hlkwd{xlim}\hlstd{(}\hlnum{55}\hlstd{,} \hlnum{70}\hlstd{)}
\end{alltt}


{\ttfamily\noindent\color{warningcolor}{\#\# Warning: Removed 45 rows containing non-finite values (stat\_density).}}\end{kframe}
\includegraphics[width=\maxwidth]{figure/021-ggplot2-geoms-geom_density-5} 
\begin{kframe}\begin{alltt}
\hlcom{## No test: }
\hlcom{# Stacked density plots: if you want to create a stacked density plot, you}
\hlcom{# probably want to 'count' (density * n) variable instead of the default}
\hlcom{# density}

\hlcom{# Loses marginal densities}
\hlkwd{ggplot}\hlstd{(diamonds,} \hlkwd{aes}\hlstd{(carat,} \hlkwc{fill} \hlstd{= cut))} \hlopt{+}
  \hlkwd{geom_density}\hlstd{(}\hlkwc{position} \hlstd{=} \hlstr{"stack"}\hlstd{)}
\end{alltt}
\end{kframe}
\includegraphics[width=\maxwidth]{figure/021-ggplot2-geoms-geom_density-6} 
\begin{kframe}\begin{alltt}
\hlcom{# Preserves marginal densities}
\hlkwd{ggplot}\hlstd{(diamonds,} \hlkwd{aes}\hlstd{(carat, ..count..,} \hlkwc{fill} \hlstd{= cut))} \hlopt{+}
  \hlkwd{geom_density}\hlstd{(}\hlkwc{position} \hlstd{=} \hlstr{"stack"}\hlstd{)}
\end{alltt}
\end{kframe}
\includegraphics[width=\maxwidth]{figure/021-ggplot2-geoms-geom_density-7} 
\begin{kframe}\begin{alltt}
\hlcom{# You can use position="fill" to produce a conditional density estimate}
\hlkwd{ggplot}\hlstd{(diamonds,} \hlkwd{aes}\hlstd{(carat, ..count..,} \hlkwc{fill} \hlstd{= cut))} \hlopt{+}
  \hlkwd{geom_density}\hlstd{(}\hlkwc{position} \hlstd{=} \hlstr{"fill"}\hlstd{)}
\end{alltt}
\end{kframe}
\includegraphics[width=\maxwidth]{figure/021-ggplot2-geoms-geom_density-8} 
\begin{kframe}\begin{alltt}
\hlcom{## End(No test)}
\end{alltt}
\end{kframe}
\end{knitrout}


\section{geom\_density2d}

\begin{knitrout}
\definecolor{shadecolor}{rgb}{0.969, 0.969, 0.969}\color{fgcolor}\begin{kframe}
\begin{alltt}
\hlcom{### Name: geom_density_2d}
\hlcom{### Title: Contours from a 2d density estimate.}
\hlcom{### Aliases: geom_density2d geom_density_2d stat_density2d stat_density_2d}

\hlcom{### ** Examples}

\hlstd{m} \hlkwb{<-} \hlkwd{ggplot}\hlstd{(faithful,} \hlkwd{aes}\hlstd{(}\hlkwc{x} \hlstd{= eruptions,} \hlkwc{y} \hlstd{= waiting))} \hlopt{+}
 \hlkwd{geom_point}\hlstd{()} \hlopt{+}
 \hlkwd{xlim}\hlstd{(}\hlnum{0.5}\hlstd{,} \hlnum{6}\hlstd{)} \hlopt{+}
 \hlkwd{ylim}\hlstd{(}\hlnum{40}\hlstd{,} \hlnum{110}\hlstd{)}
\hlstd{m} \hlopt{+} \hlkwd{geom_density_2d}\hlstd{()}
\end{alltt}
\end{kframe}
\includegraphics[width=\maxwidth]{figure/021-ggplot2-geoms-geom_density2d-1} 
\begin{kframe}\begin{alltt}
\hlcom{## No test: }
\hlstd{m} \hlopt{+} \hlkwd{stat_density_2d}\hlstd{(}\hlkwd{aes}\hlstd{(}\hlkwc{fill} \hlstd{= ..level..),} \hlkwc{geom} \hlstd{=} \hlstr{"polygon"}\hlstd{)}
\end{alltt}
\end{kframe}
\includegraphics[width=\maxwidth]{figure/021-ggplot2-geoms-geom_density2d-2} 
\begin{kframe}\begin{alltt}
\hlkwd{set.seed}\hlstd{(}\hlnum{4393}\hlstd{)}
\hlstd{dsmall} \hlkwb{<-} \hlstd{diamonds[}\hlkwd{sample}\hlstd{(}\hlkwd{nrow}\hlstd{(diamonds),} \hlnum{1000}\hlstd{), ]}
\hlstd{d} \hlkwb{<-} \hlkwd{ggplot}\hlstd{(dsmall,} \hlkwd{aes}\hlstd{(x, y))}
\hlcom{# If you map an aesthetic to a categorical variable, you will get a}
\hlcom{# set of contours for each value of that variable}
\hlstd{d} \hlopt{+} \hlkwd{geom_density_2d}\hlstd{(}\hlkwd{aes}\hlstd{(}\hlkwc{colour} \hlstd{= cut))}
\end{alltt}
\end{kframe}
\includegraphics[width=\maxwidth]{figure/021-ggplot2-geoms-geom_density2d-3} 
\begin{kframe}\begin{alltt}
\hlcom{# If we turn contouring off, we can use use geoms like tiles:}
\hlstd{d} \hlopt{+} \hlkwd{stat_density_2d}\hlstd{(}\hlkwc{geom} \hlstd{=} \hlstr{"raster"}\hlstd{,} \hlkwd{aes}\hlstd{(}\hlkwc{fill} \hlstd{= ..density..),} \hlkwc{contour} \hlstd{=} \hlnum{FALSE}\hlstd{)}
\end{alltt}
\end{kframe}
\includegraphics[width=\maxwidth]{figure/021-ggplot2-geoms-geom_density2d-4} 
\begin{kframe}\begin{alltt}
\hlcom{# Or points:}
\hlstd{d} \hlopt{+} \hlkwd{stat_density_2d}\hlstd{(}\hlkwc{geom} \hlstd{=} \hlstr{"point"}\hlstd{,} \hlkwd{aes}\hlstd{(}\hlkwc{size} \hlstd{= ..density..),} \hlkwc{n} \hlstd{=} \hlnum{20}\hlstd{,} \hlkwc{contour} \hlstd{=} \hlnum{FALSE}\hlstd{)}
\end{alltt}
\end{kframe}
\includegraphics[width=\maxwidth]{figure/021-ggplot2-geoms-geom_density2d-5} 
\begin{kframe}\begin{alltt}
\hlcom{## End(No test)}
\end{alltt}
\end{kframe}
\end{knitrout}


\section{geom\_density\_2d}

\begin{knitrout}
\definecolor{shadecolor}{rgb}{0.969, 0.969, 0.969}\color{fgcolor}\begin{kframe}
\begin{alltt}
\hlcom{### Name: geom_density_2d}
\hlcom{### Title: Contours from a 2d density estimate.}
\hlcom{### Aliases: geom_density2d geom_density_2d stat_density2d stat_density_2d}

\hlcom{### ** Examples}

\hlstd{m} \hlkwb{<-} \hlkwd{ggplot}\hlstd{(faithful,} \hlkwd{aes}\hlstd{(}\hlkwc{x} \hlstd{= eruptions,} \hlkwc{y} \hlstd{= waiting))} \hlopt{+}
 \hlkwd{geom_point}\hlstd{()} \hlopt{+}
 \hlkwd{xlim}\hlstd{(}\hlnum{0.5}\hlstd{,} \hlnum{6}\hlstd{)} \hlopt{+}
 \hlkwd{ylim}\hlstd{(}\hlnum{40}\hlstd{,} \hlnum{110}\hlstd{)}
\hlstd{m} \hlopt{+} \hlkwd{geom_density_2d}\hlstd{()}
\end{alltt}
\end{kframe}
\includegraphics[width=\maxwidth]{figure/021-ggplot2-geoms-geom_density_2d-1} 
\begin{kframe}\begin{alltt}
\hlcom{## No test: }
\hlstd{m} \hlopt{+} \hlkwd{stat_density_2d}\hlstd{(}\hlkwd{aes}\hlstd{(}\hlkwc{fill} \hlstd{= ..level..),} \hlkwc{geom} \hlstd{=} \hlstr{"polygon"}\hlstd{)}
\end{alltt}
\end{kframe}
\includegraphics[width=\maxwidth]{figure/021-ggplot2-geoms-geom_density_2d-2} 
\begin{kframe}\begin{alltt}
\hlkwd{set.seed}\hlstd{(}\hlnum{4393}\hlstd{)}
\hlstd{dsmall} \hlkwb{<-} \hlstd{diamonds[}\hlkwd{sample}\hlstd{(}\hlkwd{nrow}\hlstd{(diamonds),} \hlnum{1000}\hlstd{), ]}
\hlstd{d} \hlkwb{<-} \hlkwd{ggplot}\hlstd{(dsmall,} \hlkwd{aes}\hlstd{(x, y))}
\hlcom{# If you map an aesthetic to a categorical variable, you will get a}
\hlcom{# set of contours for each value of that variable}
\hlstd{d} \hlopt{+} \hlkwd{geom_density_2d}\hlstd{(}\hlkwd{aes}\hlstd{(}\hlkwc{colour} \hlstd{= cut))}
\end{alltt}
\end{kframe}
\includegraphics[width=\maxwidth]{figure/021-ggplot2-geoms-geom_density_2d-3} 
\begin{kframe}\begin{alltt}
\hlcom{# If we turn contouring off, we can use use geoms like tiles:}
\hlstd{d} \hlopt{+} \hlkwd{stat_density_2d}\hlstd{(}\hlkwc{geom} \hlstd{=} \hlstr{"raster"}\hlstd{,} \hlkwd{aes}\hlstd{(}\hlkwc{fill} \hlstd{= ..density..),} \hlkwc{contour} \hlstd{=} \hlnum{FALSE}\hlstd{)}
\end{alltt}
\end{kframe}
\includegraphics[width=\maxwidth]{figure/021-ggplot2-geoms-geom_density_2d-4} 
\begin{kframe}\begin{alltt}
\hlcom{# Or points:}
\hlstd{d} \hlopt{+} \hlkwd{stat_density_2d}\hlstd{(}\hlkwc{geom} \hlstd{=} \hlstr{"point"}\hlstd{,} \hlkwd{aes}\hlstd{(}\hlkwc{size} \hlstd{= ..density..),} \hlkwc{n} \hlstd{=} \hlnum{20}\hlstd{,} \hlkwc{contour} \hlstd{=} \hlnum{FALSE}\hlstd{)}
\end{alltt}
\end{kframe}
\includegraphics[width=\maxwidth]{figure/021-ggplot2-geoms-geom_density_2d-5} 
\begin{kframe}\begin{alltt}
\hlcom{## End(No test)}
\end{alltt}
\end{kframe}
\end{knitrout}


\section{geom\_dotplot}

\begin{knitrout}
\definecolor{shadecolor}{rgb}{0.969, 0.969, 0.969}\color{fgcolor}\begin{kframe}
\begin{alltt}
\hlcom{### Name: geom_dotplot}
\hlcom{### Title: Dot plot}
\hlcom{### Aliases: geom_dotplot}

\hlcom{### ** Examples}

\hlkwd{ggplot}\hlstd{(mtcars,} \hlkwd{aes}\hlstd{(}\hlkwc{x} \hlstd{= mpg))} \hlopt{+} \hlkwd{geom_dotplot}\hlstd{()}
\end{alltt}


{\ttfamily\noindent\itshape\color{messagecolor}{\#\# `stat\_bindot()` using `bins = 30`. Pick better value with `binwidth`.}}\end{kframe}
\includegraphics[width=\maxwidth]{figure/021-ggplot2-geoms-geom_dotplot-1} 
\begin{kframe}\begin{alltt}
\hlkwd{ggplot}\hlstd{(mtcars,} \hlkwd{aes}\hlstd{(}\hlkwc{x} \hlstd{= mpg))} \hlopt{+} \hlkwd{geom_dotplot}\hlstd{(}\hlkwc{binwidth} \hlstd{=} \hlnum{1.5}\hlstd{)}
\end{alltt}
\end{kframe}
\includegraphics[width=\maxwidth]{figure/021-ggplot2-geoms-geom_dotplot-2} 
\begin{kframe}\begin{alltt}
\hlcom{# Use fixed-width bins}
\hlkwd{ggplot}\hlstd{(mtcars,} \hlkwd{aes}\hlstd{(}\hlkwc{x} \hlstd{= mpg))} \hlopt{+}
  \hlkwd{geom_dotplot}\hlstd{(}\hlkwc{method}\hlstd{=}\hlstr{"histodot"}\hlstd{,} \hlkwc{binwidth} \hlstd{=} \hlnum{1.5}\hlstd{)}
\end{alltt}
\end{kframe}
\includegraphics[width=\maxwidth]{figure/021-ggplot2-geoms-geom_dotplot-3} 
\begin{kframe}\begin{alltt}
\hlcom{# Some other stacking methods}
\hlkwd{ggplot}\hlstd{(mtcars,} \hlkwd{aes}\hlstd{(}\hlkwc{x} \hlstd{= mpg))} \hlopt{+}
  \hlkwd{geom_dotplot}\hlstd{(}\hlkwc{binwidth} \hlstd{=} \hlnum{1.5}\hlstd{,} \hlkwc{stackdir} \hlstd{=} \hlstr{"center"}\hlstd{)}
\end{alltt}
\end{kframe}
\includegraphics[width=\maxwidth]{figure/021-ggplot2-geoms-geom_dotplot-4} 
\begin{kframe}\begin{alltt}
\hlkwd{ggplot}\hlstd{(mtcars,} \hlkwd{aes}\hlstd{(}\hlkwc{x} \hlstd{= mpg))} \hlopt{+}
  \hlkwd{geom_dotplot}\hlstd{(}\hlkwc{binwidth} \hlstd{=} \hlnum{1.5}\hlstd{,} \hlkwc{stackdir} \hlstd{=} \hlstr{"centerwhole"}\hlstd{)}
\end{alltt}
\end{kframe}
\includegraphics[width=\maxwidth]{figure/021-ggplot2-geoms-geom_dotplot-5} 
\begin{kframe}\begin{alltt}
\hlcom{# y axis isn't really meaningful, so hide it}
\hlkwd{ggplot}\hlstd{(mtcars,} \hlkwd{aes}\hlstd{(}\hlkwc{x} \hlstd{= mpg))} \hlopt{+} \hlkwd{geom_dotplot}\hlstd{(}\hlkwc{binwidth} \hlstd{=} \hlnum{1.5}\hlstd{)} \hlopt{+}
  \hlkwd{scale_y_continuous}\hlstd{(}\hlkwa{NULL}\hlstd{,} \hlkwc{breaks} \hlstd{=} \hlkwa{NULL}\hlstd{)}
\end{alltt}
\end{kframe}
\includegraphics[width=\maxwidth]{figure/021-ggplot2-geoms-geom_dotplot-6} 
\begin{kframe}\begin{alltt}
\hlcom{# Overlap dots vertically}
\hlkwd{ggplot}\hlstd{(mtcars,} \hlkwd{aes}\hlstd{(}\hlkwc{x} \hlstd{= mpg))} \hlopt{+} \hlkwd{geom_dotplot}\hlstd{(}\hlkwc{binwidth} \hlstd{=} \hlnum{1.5}\hlstd{,} \hlkwc{stackratio} \hlstd{=} \hlnum{.7}\hlstd{)}
\end{alltt}
\end{kframe}
\includegraphics[width=\maxwidth]{figure/021-ggplot2-geoms-geom_dotplot-7} 
\begin{kframe}\begin{alltt}
\hlcom{# Expand dot diameter}
\hlkwd{ggplot}\hlstd{(mtcars,} \hlkwd{aes}\hlstd{(}\hlkwc{x} \hlstd{= mpg))} \hlopt{+} \hlkwd{geom_dotplot}\hlstd{(}\hlkwc{binwidth} \hlstd{=} \hlnum{1.5}\hlstd{,} \hlkwc{dotsize} \hlstd{=} \hlnum{1.25}\hlstd{)}
\end{alltt}
\end{kframe}
\includegraphics[width=\maxwidth]{figure/021-ggplot2-geoms-geom_dotplot-8} 
\begin{kframe}\begin{alltt}
\hlcom{## No test: }
\hlcom{# Examples with stacking along y axis instead of x}
\hlkwd{ggplot}\hlstd{(mtcars,} \hlkwd{aes}\hlstd{(}\hlkwc{x} \hlstd{=} \hlnum{1}\hlstd{,} \hlkwc{y} \hlstd{= mpg))} \hlopt{+}
  \hlkwd{geom_dotplot}\hlstd{(}\hlkwc{binaxis} \hlstd{=} \hlstr{"y"}\hlstd{,} \hlkwc{stackdir} \hlstd{=} \hlstr{"center"}\hlstd{)}
\end{alltt}


{\ttfamily\noindent\itshape\color{messagecolor}{\#\# `stat\_bindot()` using `bins = 30`. Pick better value with `binwidth`.}}\end{kframe}
\includegraphics[width=\maxwidth]{figure/021-ggplot2-geoms-geom_dotplot-9} 
\begin{kframe}\begin{alltt}
\hlkwd{ggplot}\hlstd{(mtcars,} \hlkwd{aes}\hlstd{(}\hlkwc{x} \hlstd{=} \hlkwd{factor}\hlstd{(cyl),} \hlkwc{y} \hlstd{= mpg))} \hlopt{+}
  \hlkwd{geom_dotplot}\hlstd{(}\hlkwc{binaxis} \hlstd{=} \hlstr{"y"}\hlstd{,} \hlkwc{stackdir} \hlstd{=} \hlstr{"center"}\hlstd{)}
\end{alltt}


{\ttfamily\noindent\itshape\color{messagecolor}{\#\# `stat\_bindot()` using `bins = 30`. Pick better value with `binwidth`.}}\end{kframe}
\includegraphics[width=\maxwidth]{figure/021-ggplot2-geoms-geom_dotplot-10} 
\begin{kframe}\begin{alltt}
\hlkwd{ggplot}\hlstd{(mtcars,} \hlkwd{aes}\hlstd{(}\hlkwc{x} \hlstd{=} \hlkwd{factor}\hlstd{(cyl),} \hlkwc{y} \hlstd{= mpg))} \hlopt{+}
  \hlkwd{geom_dotplot}\hlstd{(}\hlkwc{binaxis} \hlstd{=} \hlstr{"y"}\hlstd{,} \hlkwc{stackdir} \hlstd{=} \hlstr{"centerwhole"}\hlstd{)}
\end{alltt}


{\ttfamily\noindent\itshape\color{messagecolor}{\#\# `stat\_bindot()` using `bins = 30`. Pick better value with `binwidth`.}}\end{kframe}
\includegraphics[width=\maxwidth]{figure/021-ggplot2-geoms-geom_dotplot-11} 
\begin{kframe}\begin{alltt}
\hlkwd{ggplot}\hlstd{(mtcars,} \hlkwd{aes}\hlstd{(}\hlkwc{x} \hlstd{=} \hlkwd{factor}\hlstd{(vs),} \hlkwc{fill} \hlstd{=} \hlkwd{factor}\hlstd{(cyl),} \hlkwc{y} \hlstd{= mpg))} \hlopt{+}
  \hlkwd{geom_dotplot}\hlstd{(}\hlkwc{binaxis} \hlstd{=} \hlstr{"y"}\hlstd{,} \hlkwc{stackdir} \hlstd{=} \hlstr{"center"}\hlstd{,} \hlkwc{position} \hlstd{=} \hlstr{"dodge"}\hlstd{)}
\end{alltt}


{\ttfamily\noindent\itshape\color{messagecolor}{\#\# `stat\_bindot()` using `bins = 30`. Pick better value with `binwidth`.}}\end{kframe}
\includegraphics[width=\maxwidth]{figure/021-ggplot2-geoms-geom_dotplot-12} 
\begin{kframe}\begin{alltt}
\hlcom{# binpositions="all" ensures that the bins are aligned between groups}
\hlkwd{ggplot}\hlstd{(mtcars,} \hlkwd{aes}\hlstd{(}\hlkwc{x} \hlstd{=} \hlkwd{factor}\hlstd{(am),} \hlkwc{y} \hlstd{= mpg))} \hlopt{+}
  \hlkwd{geom_dotplot}\hlstd{(}\hlkwc{binaxis} \hlstd{=} \hlstr{"y"}\hlstd{,} \hlkwc{stackdir} \hlstd{=} \hlstr{"center"}\hlstd{,} \hlkwc{binpositions}\hlstd{=}\hlstr{"all"}\hlstd{)}
\end{alltt}


{\ttfamily\noindent\itshape\color{messagecolor}{\#\# `stat\_bindot()` using `bins = 30`. Pick better value with `binwidth`.}}\end{kframe}
\includegraphics[width=\maxwidth]{figure/021-ggplot2-geoms-geom_dotplot-13} 
\begin{kframe}\begin{alltt}
\hlcom{# Stacking multiple groups, with different fill}
\hlkwd{ggplot}\hlstd{(mtcars,} \hlkwd{aes}\hlstd{(}\hlkwc{x} \hlstd{= mpg,} \hlkwc{fill} \hlstd{=} \hlkwd{factor}\hlstd{(cyl)))} \hlopt{+}
  \hlkwd{geom_dotplot}\hlstd{(}\hlkwc{stackgroups} \hlstd{=} \hlnum{TRUE}\hlstd{,} \hlkwc{binwidth} \hlstd{=} \hlnum{1}\hlstd{,} \hlkwc{binpositions} \hlstd{=} \hlstr{"all"}\hlstd{)}
\end{alltt}
\end{kframe}
\includegraphics[width=\maxwidth]{figure/021-ggplot2-geoms-geom_dotplot-14} 
\begin{kframe}\begin{alltt}
\hlkwd{ggplot}\hlstd{(mtcars,} \hlkwd{aes}\hlstd{(}\hlkwc{x} \hlstd{= mpg,} \hlkwc{fill} \hlstd{=} \hlkwd{factor}\hlstd{(cyl)))} \hlopt{+}
  \hlkwd{geom_dotplot}\hlstd{(}\hlkwc{stackgroups} \hlstd{=} \hlnum{TRUE}\hlstd{,} \hlkwc{binwidth} \hlstd{=} \hlnum{1}\hlstd{,} \hlkwc{method} \hlstd{=} \hlstr{"histodot"}\hlstd{)}
\end{alltt}
\end{kframe}
\includegraphics[width=\maxwidth]{figure/021-ggplot2-geoms-geom_dotplot-15} 
\begin{kframe}\begin{alltt}
\hlkwd{ggplot}\hlstd{(mtcars,} \hlkwd{aes}\hlstd{(}\hlkwc{x} \hlstd{=} \hlnum{1}\hlstd{,} \hlkwc{y} \hlstd{= mpg,} \hlkwc{fill} \hlstd{=} \hlkwd{factor}\hlstd{(cyl)))} \hlopt{+}
  \hlkwd{geom_dotplot}\hlstd{(}\hlkwc{binaxis} \hlstd{=} \hlstr{"y"}\hlstd{,} \hlkwc{stackgroups} \hlstd{=} \hlnum{TRUE}\hlstd{,} \hlkwc{binwidth} \hlstd{=} \hlnum{1}\hlstd{,} \hlkwc{method} \hlstd{=} \hlstr{"histodot"}\hlstd{)}
\end{alltt}
\end{kframe}
\includegraphics[width=\maxwidth]{figure/021-ggplot2-geoms-geom_dotplot-16} 
\begin{kframe}\begin{alltt}
\hlcom{## End(No test)}
\end{alltt}
\end{kframe}
\end{knitrout}


\section{geom\_errorbar}

\begin{knitrout}
\definecolor{shadecolor}{rgb}{0.969, 0.969, 0.969}\color{fgcolor}\begin{kframe}
\begin{alltt}
\hlcom{### Name: geom_crossbar}
\hlcom{### Title: Vertical intervals: lines, crossbars & errorbars.}
\hlcom{### Aliases: geom_crossbar geom_errorbar geom_linerange geom_pointrange}

\hlcom{### ** Examples}

\hlcom{#' # Create a simple example dataset}
\hlstd{df} \hlkwb{<-} \hlkwd{data.frame}\hlstd{(}
  \hlkwc{trt} \hlstd{=} \hlkwd{factor}\hlstd{(}\hlkwd{c}\hlstd{(}\hlnum{1}\hlstd{,} \hlnum{1}\hlstd{,} \hlnum{2}\hlstd{,} \hlnum{2}\hlstd{)),}
  \hlkwc{resp} \hlstd{=} \hlkwd{c}\hlstd{(}\hlnum{1}\hlstd{,} \hlnum{5}\hlstd{,} \hlnum{3}\hlstd{,} \hlnum{4}\hlstd{),}
  \hlkwc{group} \hlstd{=} \hlkwd{factor}\hlstd{(}\hlkwd{c}\hlstd{(}\hlnum{1}\hlstd{,} \hlnum{2}\hlstd{,} \hlnum{1}\hlstd{,} \hlnum{2}\hlstd{)),}
  \hlkwc{upper} \hlstd{=} \hlkwd{c}\hlstd{(}\hlnum{1.1}\hlstd{,} \hlnum{5.3}\hlstd{,} \hlnum{3.3}\hlstd{,} \hlnum{4.2}\hlstd{),}
  \hlkwc{lower} \hlstd{=} \hlkwd{c}\hlstd{(}\hlnum{0.8}\hlstd{,} \hlnum{4.6}\hlstd{,} \hlnum{2.4}\hlstd{,} \hlnum{3.6}\hlstd{)}
\hlstd{)}

\hlstd{p} \hlkwb{<-} \hlkwd{ggplot}\hlstd{(df,} \hlkwd{aes}\hlstd{(trt, resp,} \hlkwc{colour} \hlstd{= group))}
\hlstd{p} \hlopt{+} \hlkwd{geom_linerange}\hlstd{(}\hlkwd{aes}\hlstd{(}\hlkwc{ymin} \hlstd{= lower,} \hlkwc{ymax} \hlstd{= upper))}
\end{alltt}
\end{kframe}
\includegraphics[width=\maxwidth]{figure/021-ggplot2-geoms-geom_errorbar-1} 
\begin{kframe}\begin{alltt}
\hlstd{p} \hlopt{+} \hlkwd{geom_pointrange}\hlstd{(}\hlkwd{aes}\hlstd{(}\hlkwc{ymin} \hlstd{= lower,} \hlkwc{ymax} \hlstd{= upper))}
\end{alltt}
\end{kframe}
\includegraphics[width=\maxwidth]{figure/021-ggplot2-geoms-geom_errorbar-2} 
\begin{kframe}\begin{alltt}
\hlstd{p} \hlopt{+} \hlkwd{geom_crossbar}\hlstd{(}\hlkwd{aes}\hlstd{(}\hlkwc{ymin} \hlstd{= lower,} \hlkwc{ymax} \hlstd{= upper),} \hlkwc{width} \hlstd{=} \hlnum{0.2}\hlstd{)}
\end{alltt}
\end{kframe}
\includegraphics[width=\maxwidth]{figure/021-ggplot2-geoms-geom_errorbar-3} 
\begin{kframe}\begin{alltt}
\hlstd{p} \hlopt{+} \hlkwd{geom_errorbar}\hlstd{(}\hlkwd{aes}\hlstd{(}\hlkwc{ymin} \hlstd{= lower,} \hlkwc{ymax} \hlstd{= upper),} \hlkwc{width} \hlstd{=} \hlnum{0.2}\hlstd{)}
\end{alltt}
\end{kframe}
\includegraphics[width=\maxwidth]{figure/021-ggplot2-geoms-geom_errorbar-4} 
\begin{kframe}\begin{alltt}
\hlcom{# Draw lines connecting group means}
\hlstd{p} \hlopt{+}
  \hlkwd{geom_line}\hlstd{(}\hlkwd{aes}\hlstd{(}\hlkwc{group} \hlstd{= group))} \hlopt{+}
  \hlkwd{geom_errorbar}\hlstd{(}\hlkwd{aes}\hlstd{(}\hlkwc{ymin} \hlstd{= lower,} \hlkwc{ymax} \hlstd{= upper),} \hlkwc{width} \hlstd{=} \hlnum{0.2}\hlstd{)}
\end{alltt}
\end{kframe}
\includegraphics[width=\maxwidth]{figure/021-ggplot2-geoms-geom_errorbar-5} 
\begin{kframe}\begin{alltt}
\hlcom{# If you want to dodge bars and errorbars, you need to manually}
\hlcom{# specify the dodge width}
\hlstd{p} \hlkwb{<-} \hlkwd{ggplot}\hlstd{(df,} \hlkwd{aes}\hlstd{(trt, resp,} \hlkwc{fill} \hlstd{= group))}
\hlstd{p} \hlopt{+}
 \hlkwd{geom_bar}\hlstd{(}\hlkwc{position} \hlstd{=} \hlstr{"dodge"}\hlstd{,} \hlkwc{stat} \hlstd{=} \hlstr{"identity"}\hlstd{)} \hlopt{+}
 \hlkwd{geom_errorbar}\hlstd{(}\hlkwd{aes}\hlstd{(}\hlkwc{ymin} \hlstd{= lower,} \hlkwc{ymax} \hlstd{= upper),} \hlkwc{position} \hlstd{=} \hlstr{"dodge"}\hlstd{,} \hlkwc{width} \hlstd{=} \hlnum{0.25}\hlstd{)}
\end{alltt}
\end{kframe}
\includegraphics[width=\maxwidth]{figure/021-ggplot2-geoms-geom_errorbar-6} 
\begin{kframe}\begin{alltt}
\hlcom{# Because the bars and errorbars have different widths}
\hlcom{# we need to specify how wide the objects we are dodging are}
\hlstd{dodge} \hlkwb{<-} \hlkwd{position_dodge}\hlstd{(}\hlkwc{width}\hlstd{=}\hlnum{0.9}\hlstd{)}
\hlstd{p} \hlopt{+}
  \hlkwd{geom_bar}\hlstd{(}\hlkwc{position} \hlstd{= dodge,} \hlkwc{stat} \hlstd{=} \hlstr{"identity"}\hlstd{)} \hlopt{+}
  \hlkwd{geom_errorbar}\hlstd{(}\hlkwd{aes}\hlstd{(}\hlkwc{ymin} \hlstd{= lower,} \hlkwc{ymax} \hlstd{= upper),} \hlkwc{position} \hlstd{= dodge,} \hlkwc{width} \hlstd{=} \hlnum{0.25}\hlstd{)}
\end{alltt}
\end{kframe}
\includegraphics[width=\maxwidth]{figure/021-ggplot2-geoms-geom_errorbar-7} 

\end{knitrout}


\section{geom\_errorbarh}

\begin{knitrout}
\definecolor{shadecolor}{rgb}{0.969, 0.969, 0.969}\color{fgcolor}\begin{kframe}
\begin{alltt}
\hlcom{### Name: geom_errorbarh}
\hlcom{### Title: Horizontal error bars}
\hlcom{### Aliases: geom_errorbarh}

\hlcom{### ** Examples}

\hlstd{df} \hlkwb{<-} \hlkwd{data.frame}\hlstd{(}
  \hlkwc{trt} \hlstd{=} \hlkwd{factor}\hlstd{(}\hlkwd{c}\hlstd{(}\hlnum{1}\hlstd{,} \hlnum{1}\hlstd{,} \hlnum{2}\hlstd{,} \hlnum{2}\hlstd{)),}
  \hlkwc{resp} \hlstd{=} \hlkwd{c}\hlstd{(}\hlnum{1}\hlstd{,} \hlnum{5}\hlstd{,} \hlnum{3}\hlstd{,} \hlnum{4}\hlstd{),}
  \hlkwc{group} \hlstd{=} \hlkwd{factor}\hlstd{(}\hlkwd{c}\hlstd{(}\hlnum{1}\hlstd{,} \hlnum{2}\hlstd{,} \hlnum{1}\hlstd{,} \hlnum{2}\hlstd{)),}
  \hlkwc{se} \hlstd{=} \hlkwd{c}\hlstd{(}\hlnum{0.1}\hlstd{,} \hlnum{0.3}\hlstd{,} \hlnum{0.3}\hlstd{,} \hlnum{0.2}\hlstd{)}
\hlstd{)}

\hlcom{# Define the top and bottom of the errorbars}

\hlstd{p} \hlkwb{<-} \hlkwd{ggplot}\hlstd{(df,} \hlkwd{aes}\hlstd{(resp, trt,} \hlkwc{colour} \hlstd{= group))}
\hlstd{p} \hlopt{+} \hlkwd{geom_point}\hlstd{()} \hlopt{+}
  \hlkwd{geom_errorbarh}\hlstd{(}\hlkwd{aes}\hlstd{(}\hlkwc{xmax} \hlstd{= resp} \hlopt{+} \hlstd{se,} \hlkwc{xmin} \hlstd{= resp} \hlopt{-} \hlstd{se))}
\end{alltt}
\end{kframe}
\includegraphics[width=\maxwidth]{figure/021-ggplot2-geoms-geom_errorbarh-1} 
\begin{kframe}\begin{alltt}
\hlstd{p} \hlopt{+} \hlkwd{geom_point}\hlstd{()} \hlopt{+}
  \hlkwd{geom_errorbarh}\hlstd{(}\hlkwd{aes}\hlstd{(}\hlkwc{xmax} \hlstd{= resp} \hlopt{+} \hlstd{se,} \hlkwc{xmin} \hlstd{= resp} \hlopt{-} \hlstd{se,} \hlkwc{height} \hlstd{=} \hlnum{.2}\hlstd{))}
\end{alltt}
\end{kframe}
\includegraphics[width=\maxwidth]{figure/021-ggplot2-geoms-geom_errorbarh-2} 

\end{knitrout}


\section{geom\_freqpoly}

\begin{knitrout}
\definecolor{shadecolor}{rgb}{0.969, 0.969, 0.969}\color{fgcolor}\begin{kframe}
\begin{alltt}
\hlcom{### Name: geom_freqpoly}
\hlcom{### Title: Histograms and frequency polygons.}
\hlcom{### Aliases: geom_freqpoly geom_histogram stat_bin}

\hlcom{### ** Examples}

\hlkwd{ggplot}\hlstd{(diamonds,} \hlkwd{aes}\hlstd{(carat))} \hlopt{+}
  \hlkwd{geom_histogram}\hlstd{()}
\end{alltt}


{\ttfamily\noindent\itshape\color{messagecolor}{\#\# `stat\_bin()` using `bins = 30`. Pick better value with `binwidth`.}}\end{kframe}
\includegraphics[width=\maxwidth]{figure/021-ggplot2-geoms-geom_freqpoly-1} 
\begin{kframe}\begin{alltt}
\hlkwd{ggplot}\hlstd{(diamonds,} \hlkwd{aes}\hlstd{(carat))} \hlopt{+}
  \hlkwd{geom_histogram}\hlstd{(}\hlkwc{binwidth} \hlstd{=} \hlnum{0.01}\hlstd{)}
\end{alltt}
\end{kframe}
\includegraphics[width=\maxwidth]{figure/021-ggplot2-geoms-geom_freqpoly-2} 
\begin{kframe}\begin{alltt}
\hlkwd{ggplot}\hlstd{(diamonds,} \hlkwd{aes}\hlstd{(carat))} \hlopt{+}
  \hlkwd{geom_histogram}\hlstd{(}\hlkwc{bins} \hlstd{=} \hlnum{200}\hlstd{)}
\end{alltt}
\end{kframe}
\includegraphics[width=\maxwidth]{figure/021-ggplot2-geoms-geom_freqpoly-3} 
\begin{kframe}\begin{alltt}
\hlcom{# Rather than stacking histograms, it's easier to compare frequency}
\hlcom{# polygons}
\hlkwd{ggplot}\hlstd{(diamonds,} \hlkwd{aes}\hlstd{(price,} \hlkwc{fill} \hlstd{= cut))} \hlopt{+}
  \hlkwd{geom_histogram}\hlstd{(}\hlkwc{binwidth} \hlstd{=} \hlnum{500}\hlstd{)}
\end{alltt}
\end{kframe}
\includegraphics[width=\maxwidth]{figure/021-ggplot2-geoms-geom_freqpoly-4} 
\begin{kframe}\begin{alltt}
\hlkwd{ggplot}\hlstd{(diamonds,} \hlkwd{aes}\hlstd{(price,} \hlkwc{colour} \hlstd{= cut))} \hlopt{+}
  \hlkwd{geom_freqpoly}\hlstd{(}\hlkwc{binwidth} \hlstd{=} \hlnum{500}\hlstd{)}
\end{alltt}
\end{kframe}
\includegraphics[width=\maxwidth]{figure/021-ggplot2-geoms-geom_freqpoly-5} 
\begin{kframe}\begin{alltt}
\hlcom{# To make it easier to compare distributions with very different counts,}
\hlcom{# put density on the y axis instead of the default count}
\hlkwd{ggplot}\hlstd{(diamonds,} \hlkwd{aes}\hlstd{(price, ..density..,} \hlkwc{colour} \hlstd{= cut))} \hlopt{+}
  \hlkwd{geom_freqpoly}\hlstd{(}\hlkwc{binwidth} \hlstd{=} \hlnum{500}\hlstd{)}
\end{alltt}
\end{kframe}
\includegraphics[width=\maxwidth]{figure/021-ggplot2-geoms-geom_freqpoly-6} 
\begin{kframe}\begin{alltt}
\hlkwa{if} \hlstd{(}\hlkwd{require}\hlstd{(}\hlstr{"ggplot2movies"}\hlstd{)) \{}
\hlcom{# Often we don't want the height of the bar to represent the}
\hlcom{# count of observations, but the sum of some other variable.}
\hlcom{# For example, the following plot shows the number of movies}
\hlcom{# in each rating.}
\hlstd{m} \hlkwb{<-} \hlkwd{ggplot}\hlstd{(movies,} \hlkwd{aes}\hlstd{(rating))}
\hlstd{m} \hlopt{+} \hlkwd{geom_histogram}\hlstd{(}\hlkwc{binwidth} \hlstd{=} \hlnum{0.1}\hlstd{)}

\hlcom{# If, however, we want to see the number of votes cast in each}
\hlcom{# category, we need to weight by the votes variable}
\hlstd{m} \hlopt{+} \hlkwd{geom_histogram}\hlstd{(}\hlkwd{aes}\hlstd{(}\hlkwc{weight} \hlstd{= votes),} \hlkwc{binwidth} \hlstd{=} \hlnum{0.1}\hlstd{)} \hlopt{+} \hlkwd{ylab}\hlstd{(}\hlstr{"votes"}\hlstd{)}

\hlcom{# For transformed scales, binwidth applies to the transformed data.}
\hlcom{# The bins have constant width on the transformed scale.}
\hlstd{m} \hlopt{+} \hlkwd{geom_histogram}\hlstd{()} \hlopt{+} \hlkwd{scale_x_log10}\hlstd{()}
\hlstd{m} \hlopt{+} \hlkwd{geom_histogram}\hlstd{(}\hlkwc{binwidth} \hlstd{=} \hlnum{0.05}\hlstd{)} \hlopt{+} \hlkwd{scale_x_log10}\hlstd{()}

\hlcom{# For transformed coordinate systems, the binwidth applies to the}
\hlcom{# raw data. The bins have constant width on the original scale.}

\hlcom{# Using log scales does not work here, because the first}
\hlcom{# bar is anchored at zero, and so when transformed becomes negative}
\hlcom{# infinity. This is not a problem when transforming the scales, because}
\hlcom{# no observations have 0 ratings.}
\hlstd{m} \hlopt{+} \hlkwd{geom_histogram}\hlstd{(}\hlkwc{origin} \hlstd{=} \hlnum{0}\hlstd{)} \hlopt{+} \hlkwd{coord_trans}\hlstd{(}\hlkwc{x} \hlstd{=} \hlstr{"log10"}\hlstd{)}
\hlcom{# Use origin = 0, to make sure we don't take sqrt of negative values}
\hlstd{m} \hlopt{+} \hlkwd{geom_histogram}\hlstd{(}\hlkwc{origin} \hlstd{=} \hlnum{0}\hlstd{)} \hlopt{+} \hlkwd{coord_trans}\hlstd{(}\hlkwc{x} \hlstd{=} \hlstr{"sqrt"}\hlstd{)}

\hlcom{# You can also transform the y axis.  Remember that the base of the bars}
\hlcom{# has value 0, so log transformations are not appropriate}
\hlstd{m} \hlkwb{<-} \hlkwd{ggplot}\hlstd{(movies,} \hlkwd{aes}\hlstd{(}\hlkwc{x} \hlstd{= rating))}
\hlstd{m} \hlopt{+} \hlkwd{geom_histogram}\hlstd{(}\hlkwc{binwidth} \hlstd{=} \hlnum{0.5}\hlstd{)} \hlopt{+} \hlkwd{scale_y_sqrt}\hlstd{()}
\hlstd{\}}
\end{alltt}


{\ttfamily\noindent\itshape\color{messagecolor}{\#\# Loading required package: ggplot2movies}}

{\ttfamily\noindent\color{warningcolor}{\#\# Warning in library(package, lib.loc = lib.loc, character.only = TRUE, logical.return = TRUE, : there is no package called 'ggplot2movies'}}\begin{alltt}
\hlkwd{rm}\hlstd{(movies)}
\end{alltt}


{\ttfamily\noindent\color{warningcolor}{\#\# Warning in rm(movies): object 'movies' not found}}\end{kframe}
\end{knitrout}


\section{geom\_hex}

\begin{knitrout}
\definecolor{shadecolor}{rgb}{0.969, 0.969, 0.969}\color{fgcolor}\begin{kframe}
\begin{alltt}
\hlcom{### Name: geom_hex}
\hlcom{### Title: Hexagon binning.}
\hlcom{### Aliases: geom_hex stat_bin_hex stat_binhex}

\hlcom{### ** Examples}

\hlstd{d} \hlkwb{<-} \hlkwd{ggplot}\hlstd{(diamonds,} \hlkwd{aes}\hlstd{(carat, price))}
\hlstd{d} \hlopt{+} \hlkwd{geom_hex}\hlstd{()}
\end{alltt}
\end{kframe}
\includegraphics[width=\maxwidth]{figure/021-ggplot2-geoms-geom_hex-1} 
\begin{kframe}\begin{alltt}
\hlcom{## No test: }
\hlcom{# You can control the size of the bins by specifying the number of}
\hlcom{# bins in each direction:}
\hlstd{d} \hlopt{+} \hlkwd{geom_hex}\hlstd{(}\hlkwc{bins} \hlstd{=} \hlnum{10}\hlstd{)}
\end{alltt}
\end{kframe}
\includegraphics[width=\maxwidth]{figure/021-ggplot2-geoms-geom_hex-2} 
\begin{kframe}\begin{alltt}
\hlstd{d} \hlopt{+} \hlkwd{geom_hex}\hlstd{(}\hlkwc{bins} \hlstd{=} \hlnum{30}\hlstd{)}
\end{alltt}
\end{kframe}
\includegraphics[width=\maxwidth]{figure/021-ggplot2-geoms-geom_hex-3} 
\begin{kframe}\begin{alltt}
\hlcom{# Or by specifying the width of the bins}
\hlstd{d} \hlopt{+} \hlkwd{geom_hex}\hlstd{(}\hlkwc{binwidth} \hlstd{=} \hlkwd{c}\hlstd{(}\hlnum{1}\hlstd{,} \hlnum{1000}\hlstd{))}
\end{alltt}
\end{kframe}
\includegraphics[width=\maxwidth]{figure/021-ggplot2-geoms-geom_hex-4} 
\begin{kframe}\begin{alltt}
\hlstd{d} \hlopt{+} \hlkwd{geom_hex}\hlstd{(}\hlkwc{binwidth} \hlstd{=} \hlkwd{c}\hlstd{(}\hlnum{.1}\hlstd{,} \hlnum{500}\hlstd{))}
\end{alltt}
\end{kframe}
\includegraphics[width=\maxwidth]{figure/021-ggplot2-geoms-geom_hex-5} 
\begin{kframe}\begin{alltt}
\hlcom{## End(No test)}
\end{alltt}
\end{kframe}
\end{knitrout}


\section{geom\_histogram}

\begin{knitrout}
\definecolor{shadecolor}{rgb}{0.969, 0.969, 0.969}\color{fgcolor}\begin{kframe}
\begin{alltt}
\hlcom{### Name: geom_freqpoly}
\hlcom{### Title: Histograms and frequency polygons.}
\hlcom{### Aliases: geom_freqpoly geom_histogram stat_bin}

\hlcom{### ** Examples}

\hlkwd{ggplot}\hlstd{(diamonds,} \hlkwd{aes}\hlstd{(carat))} \hlopt{+}
  \hlkwd{geom_histogram}\hlstd{()}
\end{alltt}


{\ttfamily\noindent\itshape\color{messagecolor}{\#\# `stat\_bin()` using `bins = 30`. Pick better value with `binwidth`.}}\end{kframe}
\includegraphics[width=\maxwidth]{figure/021-ggplot2-geoms-geom_histogram-1} 
\begin{kframe}\begin{alltt}
\hlkwd{ggplot}\hlstd{(diamonds,} \hlkwd{aes}\hlstd{(carat))} \hlopt{+}
  \hlkwd{geom_histogram}\hlstd{(}\hlkwc{binwidth} \hlstd{=} \hlnum{0.01}\hlstd{)}
\end{alltt}
\end{kframe}
\includegraphics[width=\maxwidth]{figure/021-ggplot2-geoms-geom_histogram-2} 
\begin{kframe}\begin{alltt}
\hlkwd{ggplot}\hlstd{(diamonds,} \hlkwd{aes}\hlstd{(carat))} \hlopt{+}
  \hlkwd{geom_histogram}\hlstd{(}\hlkwc{bins} \hlstd{=} \hlnum{200}\hlstd{)}
\end{alltt}
\end{kframe}
\includegraphics[width=\maxwidth]{figure/021-ggplot2-geoms-geom_histogram-3} 
\begin{kframe}\begin{alltt}
\hlcom{# Rather than stacking histograms, it's easier to compare frequency}
\hlcom{# polygons}
\hlkwd{ggplot}\hlstd{(diamonds,} \hlkwd{aes}\hlstd{(price,} \hlkwc{fill} \hlstd{= cut))} \hlopt{+}
  \hlkwd{geom_histogram}\hlstd{(}\hlkwc{binwidth} \hlstd{=} \hlnum{500}\hlstd{)}
\end{alltt}
\end{kframe}
\includegraphics[width=\maxwidth]{figure/021-ggplot2-geoms-geom_histogram-4} 
\begin{kframe}\begin{alltt}
\hlkwd{ggplot}\hlstd{(diamonds,} \hlkwd{aes}\hlstd{(price,} \hlkwc{colour} \hlstd{= cut))} \hlopt{+}
  \hlkwd{geom_freqpoly}\hlstd{(}\hlkwc{binwidth} \hlstd{=} \hlnum{500}\hlstd{)}
\end{alltt}
\end{kframe}
\includegraphics[width=\maxwidth]{figure/021-ggplot2-geoms-geom_histogram-5} 
\begin{kframe}\begin{alltt}
\hlcom{# To make it easier to compare distributions with very different counts,}
\hlcom{# put density on the y axis instead of the default count}
\hlkwd{ggplot}\hlstd{(diamonds,} \hlkwd{aes}\hlstd{(price, ..density..,} \hlkwc{colour} \hlstd{= cut))} \hlopt{+}
  \hlkwd{geom_freqpoly}\hlstd{(}\hlkwc{binwidth} \hlstd{=} \hlnum{500}\hlstd{)}
\end{alltt}
\end{kframe}
\includegraphics[width=\maxwidth]{figure/021-ggplot2-geoms-geom_histogram-6} 
\begin{kframe}\begin{alltt}
\hlkwa{if} \hlstd{(}\hlkwd{require}\hlstd{(}\hlstr{"ggplot2movies"}\hlstd{)) \{}
\hlcom{# Often we don't want the height of the bar to represent the}
\hlcom{# count of observations, but the sum of some other variable.}
\hlcom{# For example, the following plot shows the number of movies}
\hlcom{# in each rating.}
\hlstd{m} \hlkwb{<-} \hlkwd{ggplot}\hlstd{(movies,} \hlkwd{aes}\hlstd{(rating))}
\hlstd{m} \hlopt{+} \hlkwd{geom_histogram}\hlstd{(}\hlkwc{binwidth} \hlstd{=} \hlnum{0.1}\hlstd{)}

\hlcom{# If, however, we want to see the number of votes cast in each}
\hlcom{# category, we need to weight by the votes variable}
\hlstd{m} \hlopt{+} \hlkwd{geom_histogram}\hlstd{(}\hlkwd{aes}\hlstd{(}\hlkwc{weight} \hlstd{= votes),} \hlkwc{binwidth} \hlstd{=} \hlnum{0.1}\hlstd{)} \hlopt{+} \hlkwd{ylab}\hlstd{(}\hlstr{"votes"}\hlstd{)}

\hlcom{# For transformed scales, binwidth applies to the transformed data.}
\hlcom{# The bins have constant width on the transformed scale.}
\hlstd{m} \hlopt{+} \hlkwd{geom_histogram}\hlstd{()} \hlopt{+} \hlkwd{scale_x_log10}\hlstd{()}
\hlstd{m} \hlopt{+} \hlkwd{geom_histogram}\hlstd{(}\hlkwc{binwidth} \hlstd{=} \hlnum{0.05}\hlstd{)} \hlopt{+} \hlkwd{scale_x_log10}\hlstd{()}

\hlcom{# For transformed coordinate systems, the binwidth applies to the}
\hlcom{# raw data. The bins have constant width on the original scale.}

\hlcom{# Using log scales does not work here, because the first}
\hlcom{# bar is anchored at zero, and so when transformed becomes negative}
\hlcom{# infinity. This is not a problem when transforming the scales, because}
\hlcom{# no observations have 0 ratings.}
\hlstd{m} \hlopt{+} \hlkwd{geom_histogram}\hlstd{(}\hlkwc{origin} \hlstd{=} \hlnum{0}\hlstd{)} \hlopt{+} \hlkwd{coord_trans}\hlstd{(}\hlkwc{x} \hlstd{=} \hlstr{"log10"}\hlstd{)}
\hlcom{# Use origin = 0, to make sure we don't take sqrt of negative values}
\hlstd{m} \hlopt{+} \hlkwd{geom_histogram}\hlstd{(}\hlkwc{origin} \hlstd{=} \hlnum{0}\hlstd{)} \hlopt{+} \hlkwd{coord_trans}\hlstd{(}\hlkwc{x} \hlstd{=} \hlstr{"sqrt"}\hlstd{)}

\hlcom{# You can also transform the y axis.  Remember that the base of the bars}
\hlcom{# has value 0, so log transformations are not appropriate}
\hlstd{m} \hlkwb{<-} \hlkwd{ggplot}\hlstd{(movies,} \hlkwd{aes}\hlstd{(}\hlkwc{x} \hlstd{= rating))}
\hlstd{m} \hlopt{+} \hlkwd{geom_histogram}\hlstd{(}\hlkwc{binwidth} \hlstd{=} \hlnum{0.5}\hlstd{)} \hlopt{+} \hlkwd{scale_y_sqrt}\hlstd{()}
\hlstd{\}}
\end{alltt}


{\ttfamily\noindent\itshape\color{messagecolor}{\#\# Loading required package: ggplot2movies}}

{\ttfamily\noindent\color{warningcolor}{\#\# Warning in library(package, lib.loc = lib.loc, character.only = TRUE, logical.return = TRUE, : there is no package called 'ggplot2movies'}}\begin{alltt}
\hlkwd{rm}\hlstd{(movies)}
\end{alltt}


{\ttfamily\noindent\color{warningcolor}{\#\# Warning in rm(movies): object 'movies' not found}}\end{kframe}
\end{knitrout}


\section{geom\_hline}

\begin{knitrout}
\definecolor{shadecolor}{rgb}{0.969, 0.969, 0.969}\color{fgcolor}\begin{kframe}
\begin{alltt}
\hlcom{### Name: geom_abline}
\hlcom{### Title: Lines: horizontal, vertical, and specified by slope and}
\hlcom{###   intercept.}
\hlcom{### Aliases: geom_abline geom_hline geom_vline}

\hlcom{### ** Examples}

\hlstd{p} \hlkwb{<-} \hlkwd{ggplot}\hlstd{(mtcars,} \hlkwd{aes}\hlstd{(wt, mpg))} \hlopt{+} \hlkwd{geom_point}\hlstd{()}

\hlcom{# Fixed values}
\hlstd{p} \hlopt{+} \hlkwd{geom_vline}\hlstd{(}\hlkwc{xintercept} \hlstd{=} \hlnum{5}\hlstd{)}
\end{alltt}
\end{kframe}
\includegraphics[width=\maxwidth]{figure/021-ggplot2-geoms-geom_hline-1} 
\begin{kframe}\begin{alltt}
\hlstd{p} \hlopt{+} \hlkwd{geom_vline}\hlstd{(}\hlkwc{xintercept} \hlstd{=} \hlnum{1}\hlopt{:}\hlnum{5}\hlstd{)}
\end{alltt}
\end{kframe}
\includegraphics[width=\maxwidth]{figure/021-ggplot2-geoms-geom_hline-2} 
\begin{kframe}\begin{alltt}
\hlstd{p} \hlopt{+} \hlkwd{geom_hline}\hlstd{(}\hlkwc{yintercept} \hlstd{=} \hlnum{20}\hlstd{)}
\end{alltt}
\end{kframe}
\includegraphics[width=\maxwidth]{figure/021-ggplot2-geoms-geom_hline-3} 
\begin{kframe}\begin{alltt}
\hlstd{p} \hlopt{+} \hlkwd{geom_abline}\hlstd{()} \hlcom{# Can't see it - outside the range of the data}
\end{alltt}
\end{kframe}
\includegraphics[width=\maxwidth]{figure/021-ggplot2-geoms-geom_hline-4} 
\begin{kframe}\begin{alltt}
\hlstd{p} \hlopt{+} \hlkwd{geom_abline}\hlstd{(}\hlkwc{intercept} \hlstd{=} \hlnum{20}\hlstd{)}
\end{alltt}
\end{kframe}
\includegraphics[width=\maxwidth]{figure/021-ggplot2-geoms-geom_hline-5} 
\begin{kframe}\begin{alltt}
\hlcom{# Calculate slope and intercept of line of best fit}
\hlkwd{coef}\hlstd{(}\hlkwd{lm}\hlstd{(mpg} \hlopt{~} \hlstd{wt,} \hlkwc{data} \hlstd{= mtcars))}
\end{alltt}
\begin{verbatim}
## (Intercept)          wt 
##      37.285      -5.344
\end{verbatim}
\begin{alltt}
\hlstd{p} \hlopt{+} \hlkwd{geom_abline}\hlstd{(}\hlkwc{intercept} \hlstd{=} \hlnum{37}\hlstd{,} \hlkwc{slope} \hlstd{=} \hlopt{-}\hlnum{5}\hlstd{)}
\end{alltt}
\end{kframe}
\includegraphics[width=\maxwidth]{figure/021-ggplot2-geoms-geom_hline-6} 
\begin{kframe}\begin{alltt}
\hlcom{# But this is easier to do with geom_smooth:}
\hlstd{p} \hlopt{+} \hlkwd{geom_smooth}\hlstd{(}\hlkwc{method} \hlstd{=} \hlstr{"lm"}\hlstd{,} \hlkwc{se} \hlstd{=} \hlnum{FALSE}\hlstd{)}
\end{alltt}
\end{kframe}
\includegraphics[width=\maxwidth]{figure/021-ggplot2-geoms-geom_hline-7} 
\begin{kframe}\begin{alltt}
\hlcom{# To show different lines in different facets, use aesthetics}
\hlstd{p} \hlkwb{<-} \hlkwd{ggplot}\hlstd{(mtcars,} \hlkwd{aes}\hlstd{(mpg, wt))} \hlopt{+}
  \hlkwd{geom_point}\hlstd{()} \hlopt{+}
  \hlkwd{facet_wrap}\hlstd{(}\hlopt{~} \hlstd{cyl)}

\hlstd{mean_wt} \hlkwb{<-} \hlkwd{data.frame}\hlstd{(}\hlkwc{cyl} \hlstd{=} \hlkwd{c}\hlstd{(}\hlnum{4}\hlstd{,} \hlnum{6}\hlstd{,} \hlnum{8}\hlstd{),} \hlkwc{wt} \hlstd{=} \hlkwd{c}\hlstd{(}\hlnum{2.28}\hlstd{,} \hlnum{3.11}\hlstd{,} \hlnum{4.00}\hlstd{))}
\hlstd{p} \hlopt{+} \hlkwd{geom_hline}\hlstd{(}\hlkwd{aes}\hlstd{(}\hlkwc{yintercept} \hlstd{= wt), mean_wt)}
\end{alltt}
\end{kframe}
\includegraphics[width=\maxwidth]{figure/021-ggplot2-geoms-geom_hline-8} 
\begin{kframe}\begin{alltt}
\hlcom{# You can also control other aesthetics}
\hlkwd{ggplot}\hlstd{(mtcars,} \hlkwd{aes}\hlstd{(mpg, wt,} \hlkwc{colour} \hlstd{= wt))} \hlopt{+}
  \hlkwd{geom_point}\hlstd{()} \hlopt{+}
  \hlkwd{geom_hline}\hlstd{(}\hlkwd{aes}\hlstd{(}\hlkwc{yintercept} \hlstd{= wt,} \hlkwc{colour} \hlstd{= wt), mean_wt)} \hlopt{+}
  \hlkwd{facet_wrap}\hlstd{(}\hlopt{~} \hlstd{cyl)}
\end{alltt}
\end{kframe}
\includegraphics[width=\maxwidth]{figure/021-ggplot2-geoms-geom_hline-9} 

\end{knitrout}


\section{geom\_jitter}

\begin{knitrout}
\definecolor{shadecolor}{rgb}{0.969, 0.969, 0.969}\color{fgcolor}\begin{kframe}
\begin{alltt}
\hlcom{### Name: geom_jitter}
\hlcom{### Title: Points, jittered to reduce overplotting.}
\hlcom{### Aliases: geom_jitter}

\hlcom{### ** Examples}

\hlstd{p} \hlkwb{<-} \hlkwd{ggplot}\hlstd{(mpg,} \hlkwd{aes}\hlstd{(cyl, hwy))}
\hlstd{p} \hlopt{+} \hlkwd{geom_point}\hlstd{()}
\end{alltt}
\end{kframe}
\includegraphics[width=\maxwidth]{figure/021-ggplot2-geoms-geom_jitter-1} 
\begin{kframe}\begin{alltt}
\hlstd{p} \hlopt{+} \hlkwd{geom_jitter}\hlstd{()}
\end{alltt}
\end{kframe}
\includegraphics[width=\maxwidth]{figure/021-ggplot2-geoms-geom_jitter-2} 
\begin{kframe}\begin{alltt}
\hlcom{# Add aesthetic mappings}
\hlstd{p} \hlopt{+} \hlkwd{geom_jitter}\hlstd{(}\hlkwd{aes}\hlstd{(}\hlkwc{colour} \hlstd{= class))}
\end{alltt}
\end{kframe}
\includegraphics[width=\maxwidth]{figure/021-ggplot2-geoms-geom_jitter-3} 
\begin{kframe}\begin{alltt}
\hlcom{# Use smaller width/height to emphasise categories}
\hlkwd{ggplot}\hlstd{(mpg,} \hlkwd{aes}\hlstd{(cyl, hwy))} \hlopt{+} \hlkwd{geom_jitter}\hlstd{()}
\end{alltt}
\end{kframe}
\includegraphics[width=\maxwidth]{figure/021-ggplot2-geoms-geom_jitter-4} 
\begin{kframe}\begin{alltt}
\hlkwd{ggplot}\hlstd{(mpg,} \hlkwd{aes}\hlstd{(cyl, hwy))} \hlopt{+} \hlkwd{geom_jitter}\hlstd{(}\hlkwc{width} \hlstd{=} \hlnum{0.25}\hlstd{)}
\end{alltt}
\end{kframe}
\includegraphics[width=\maxwidth]{figure/021-ggplot2-geoms-geom_jitter-5} 
\begin{kframe}\begin{alltt}
\hlcom{# Use larger width/height to completely smooth away discreteness}
\hlkwd{ggplot}\hlstd{(mpg,} \hlkwd{aes}\hlstd{(cty, hwy))} \hlopt{+} \hlkwd{geom_jitter}\hlstd{()}
\end{alltt}
\end{kframe}
\includegraphics[width=\maxwidth]{figure/021-ggplot2-geoms-geom_jitter-6} 
\begin{kframe}\begin{alltt}
\hlkwd{ggplot}\hlstd{(mpg,} \hlkwd{aes}\hlstd{(cty, hwy))} \hlopt{+} \hlkwd{geom_jitter}\hlstd{(}\hlkwc{width} \hlstd{=} \hlnum{0.5}\hlstd{,} \hlkwc{height} \hlstd{=} \hlnum{0.5}\hlstd{)}
\end{alltt}
\end{kframe}
\includegraphics[width=\maxwidth]{figure/021-ggplot2-geoms-geom_jitter-7} 

\end{knitrout}


\section{geom\_label}

\begin{knitrout}
\definecolor{shadecolor}{rgb}{0.969, 0.969, 0.969}\color{fgcolor}\begin{kframe}
\begin{alltt}
\hlcom{### Name: geom_label}
\hlcom{### Title: Textual annotations.}
\hlcom{### Aliases: geom_label geom_text}

\hlcom{### ** Examples}

\hlstd{p} \hlkwb{<-} \hlkwd{ggplot}\hlstd{(mtcars,} \hlkwd{aes}\hlstd{(wt, mpg,} \hlkwc{label} \hlstd{=} \hlkwd{rownames}\hlstd{(mtcars)))}

\hlstd{p} \hlopt{+} \hlkwd{geom_text}\hlstd{()}
\end{alltt}
\end{kframe}
\includegraphics[width=\maxwidth]{figure/021-ggplot2-geoms-geom_label-1} 
\begin{kframe}\begin{alltt}
\hlcom{# Avoid overlaps}
\hlstd{p} \hlopt{+} \hlkwd{geom_text}\hlstd{(}\hlkwc{check_overlap} \hlstd{=} \hlnum{TRUE}\hlstd{)}
\end{alltt}
\end{kframe}
\includegraphics[width=\maxwidth]{figure/021-ggplot2-geoms-geom_label-2} 
\begin{kframe}\begin{alltt}
\hlcom{# Labels with background}
\hlstd{p} \hlopt{+} \hlkwd{geom_label}\hlstd{()}
\end{alltt}
\end{kframe}
\includegraphics[width=\maxwidth]{figure/021-ggplot2-geoms-geom_label-3} 
\begin{kframe}\begin{alltt}
\hlcom{# Change size of the label}
\hlstd{p} \hlopt{+} \hlkwd{geom_text}\hlstd{(}\hlkwc{size} \hlstd{=} \hlnum{10}\hlstd{)}
\end{alltt}
\end{kframe}
\includegraphics[width=\maxwidth]{figure/021-ggplot2-geoms-geom_label-4} 
\begin{kframe}\begin{alltt}
\hlcom{# Set aesthetics to fixed value}
\hlstd{p} \hlopt{+} \hlkwd{geom_point}\hlstd{()} \hlopt{+} \hlkwd{geom_text}\hlstd{(}\hlkwc{hjust} \hlstd{=} \hlnum{0}\hlstd{,} \hlkwc{nudge_x} \hlstd{=} \hlnum{0.05}\hlstd{)}
\end{alltt}
\end{kframe}
\includegraphics[width=\maxwidth]{figure/021-ggplot2-geoms-geom_label-5} 
\begin{kframe}\begin{alltt}
\hlstd{p} \hlopt{+} \hlkwd{geom_point}\hlstd{()} \hlopt{+} \hlkwd{geom_text}\hlstd{(}\hlkwc{vjust} \hlstd{=} \hlnum{0}\hlstd{,} \hlkwc{nudge_y} \hlstd{=} \hlnum{0.5}\hlstd{)}
\end{alltt}
\end{kframe}
\includegraphics[width=\maxwidth]{figure/021-ggplot2-geoms-geom_label-6} 
\begin{kframe}\begin{alltt}
\hlstd{p} \hlopt{+} \hlkwd{geom_point}\hlstd{()} \hlopt{+} \hlkwd{geom_text}\hlstd{(}\hlkwc{angle} \hlstd{=} \hlnum{45}\hlstd{)}
\end{alltt}
\end{kframe}
\includegraphics[width=\maxwidth]{figure/021-ggplot2-geoms-geom_label-7} 
\begin{kframe}\begin{alltt}
\hlcom{## Not run: }
\hlcom{##D p + geom_text(family = "Times New Roman")}
\hlcom{## End(Not run)}

\hlcom{# Add aesthetic mappings}
\hlstd{p} \hlopt{+} \hlkwd{geom_text}\hlstd{(}\hlkwd{aes}\hlstd{(}\hlkwc{colour} \hlstd{=} \hlkwd{factor}\hlstd{(cyl)))}
\end{alltt}
\end{kframe}
\includegraphics[width=\maxwidth]{figure/021-ggplot2-geoms-geom_label-8} 
\begin{kframe}\begin{alltt}
\hlstd{p} \hlopt{+} \hlkwd{geom_text}\hlstd{(}\hlkwd{aes}\hlstd{(}\hlkwc{colour} \hlstd{=} \hlkwd{factor}\hlstd{(cyl)))} \hlopt{+}
  \hlkwd{scale_colour_discrete}\hlstd{(}\hlkwc{l} \hlstd{=} \hlnum{40}\hlstd{)}
\end{alltt}
\end{kframe}
\includegraphics[width=\maxwidth]{figure/021-ggplot2-geoms-geom_label-9} 
\begin{kframe}\begin{alltt}
\hlstd{p} \hlopt{+} \hlkwd{geom_label}\hlstd{(}\hlkwd{aes}\hlstd{(}\hlkwc{fill} \hlstd{=} \hlkwd{factor}\hlstd{(cyl)),} \hlkwc{colour} \hlstd{=} \hlstr{"white"}\hlstd{,} \hlkwc{fontface} \hlstd{=} \hlstr{"bold"}\hlstd{)}
\end{alltt}
\end{kframe}
\includegraphics[width=\maxwidth]{figure/021-ggplot2-geoms-geom_label-10} 
\begin{kframe}\begin{alltt}
\hlstd{p} \hlopt{+} \hlkwd{geom_text}\hlstd{(}\hlkwd{aes}\hlstd{(}\hlkwc{size} \hlstd{= wt))}
\end{alltt}
\end{kframe}
\includegraphics[width=\maxwidth]{figure/021-ggplot2-geoms-geom_label-11} 
\begin{kframe}\begin{alltt}
\hlcom{# Scale height of text, rather than sqrt(height)}
\hlstd{p} \hlopt{+} \hlkwd{geom_text}\hlstd{(}\hlkwd{aes}\hlstd{(}\hlkwc{size} \hlstd{= wt))} \hlopt{+} \hlkwd{scale_radius}\hlstd{(}\hlkwc{range} \hlstd{=} \hlkwd{c}\hlstd{(}\hlnum{3}\hlstd{,}\hlnum{6}\hlstd{))}
\end{alltt}
\end{kframe}
\includegraphics[width=\maxwidth]{figure/021-ggplot2-geoms-geom_label-12} 
\begin{kframe}\begin{alltt}
\hlcom{# You can display expressions by setting parse = TRUE.  The}
\hlcom{# details of the display are described in ?plotmath, but note that}
\hlcom{# geom_text uses strings, not expressions.}
\hlstd{p} \hlopt{+} \hlkwd{geom_text}\hlstd{(}\hlkwd{aes}\hlstd{(}\hlkwc{label} \hlstd{=} \hlkwd{paste}\hlstd{(wt,} \hlstr{"^("}\hlstd{, cyl,} \hlstr{")"}\hlstd{,} \hlkwc{sep} \hlstd{=} \hlstr{""}\hlstd{)),}
  \hlkwc{parse} \hlstd{=} \hlnum{TRUE}\hlstd{)}
\end{alltt}
\end{kframe}
\includegraphics[width=\maxwidth]{figure/021-ggplot2-geoms-geom_label-13} 
\begin{kframe}\begin{alltt}
\hlcom{# Add a text annotation}
\hlstd{p} \hlopt{+}
  \hlkwd{geom_text}\hlstd{()} \hlopt{+}
  \hlkwd{annotate}\hlstd{(}\hlstr{"text"}\hlstd{,} \hlkwc{label} \hlstd{=} \hlstr{"plot mpg vs. wt"}\hlstd{,} \hlkwc{x} \hlstd{=} \hlnum{2}\hlstd{,} \hlkwc{y} \hlstd{=} \hlnum{15}\hlstd{,} \hlkwc{size} \hlstd{=} \hlnum{8}\hlstd{,} \hlkwc{colour} \hlstd{=} \hlstr{"red"}\hlstd{)}
\end{alltt}
\end{kframe}
\includegraphics[width=\maxwidth]{figure/021-ggplot2-geoms-geom_label-14} 
\begin{kframe}\begin{alltt}
\hlcom{## No test: }
\hlcom{# Aligning labels and bars --------------------------------------------------}
\hlstd{df} \hlkwb{<-} \hlkwd{data.frame}\hlstd{(}
  \hlkwc{x} \hlstd{=} \hlkwd{factor}\hlstd{(}\hlkwd{c}\hlstd{(}\hlnum{1}\hlstd{,} \hlnum{1}\hlstd{,} \hlnum{2}\hlstd{,} \hlnum{2}\hlstd{)),}
  \hlkwc{y} \hlstd{=} \hlkwd{c}\hlstd{(}\hlnum{1}\hlstd{,} \hlnum{3}\hlstd{,} \hlnum{2}\hlstd{,} \hlnum{1}\hlstd{),}
  \hlkwc{grp} \hlstd{=} \hlkwd{c}\hlstd{(}\hlstr{"a"}\hlstd{,} \hlstr{"b"}\hlstd{,} \hlstr{"a"}\hlstd{,} \hlstr{"b"}\hlstd{)}
\hlstd{)}

\hlcom{# ggplot2 doesn't know you want to give the labels the same virtual width}
\hlcom{# as the bars:}
\hlkwd{ggplot}\hlstd{(}\hlkwc{data} \hlstd{= df,} \hlkwd{aes}\hlstd{(x, y,} \hlkwc{fill} \hlstd{= grp,} \hlkwc{label} \hlstd{= y))} \hlopt{+}
  \hlkwd{geom_bar}\hlstd{(}\hlkwc{stat} \hlstd{=} \hlstr{"identity"}\hlstd{,} \hlkwc{position} \hlstd{=} \hlstr{"dodge"}\hlstd{)} \hlopt{+}
  \hlkwd{geom_text}\hlstd{(}\hlkwc{position} \hlstd{=} \hlstr{"dodge"}\hlstd{)}
\end{alltt}


{\ttfamily\noindent\color{warningcolor}{\#\# Warning: Width not defined. Set with `position\_dodge(width = ?)`}}\end{kframe}
\includegraphics[width=\maxwidth]{figure/021-ggplot2-geoms-geom_label-15} 
\begin{kframe}\begin{alltt}
\hlcom{# So tell it:}
\hlkwd{ggplot}\hlstd{(}\hlkwc{data} \hlstd{= df,} \hlkwd{aes}\hlstd{(x, y,} \hlkwc{fill} \hlstd{= grp,} \hlkwc{label} \hlstd{= y))} \hlopt{+}
  \hlkwd{geom_bar}\hlstd{(}\hlkwc{stat} \hlstd{=} \hlstr{"identity"}\hlstd{,} \hlkwc{position} \hlstd{=} \hlstr{"dodge"}\hlstd{)} \hlopt{+}
  \hlkwd{geom_text}\hlstd{(}\hlkwc{position} \hlstd{=} \hlkwd{position_dodge}\hlstd{(}\hlnum{0.9}\hlstd{))}
\end{alltt}
\end{kframe}
\includegraphics[width=\maxwidth]{figure/021-ggplot2-geoms-geom_label-16} 
\begin{kframe}\begin{alltt}
\hlcom{# Use you can't nudge and dodge text, so instead adjust the y postion}
\hlkwd{ggplot}\hlstd{(}\hlkwc{data} \hlstd{= df,} \hlkwd{aes}\hlstd{(x, y,} \hlkwc{fill} \hlstd{= grp,} \hlkwc{label} \hlstd{= y))} \hlopt{+}
  \hlkwd{geom_bar}\hlstd{(}\hlkwc{stat} \hlstd{=} \hlstr{"identity"}\hlstd{,} \hlkwc{position} \hlstd{=} \hlstr{"dodge"}\hlstd{)} \hlopt{+}
  \hlkwd{geom_text}\hlstd{(}\hlkwd{aes}\hlstd{(}\hlkwc{y} \hlstd{= y} \hlopt{+} \hlnum{0.05}\hlstd{),} \hlkwc{position} \hlstd{=} \hlkwd{position_dodge}\hlstd{(}\hlnum{0.9}\hlstd{),} \hlkwc{vjust} \hlstd{=} \hlnum{0}\hlstd{)}
\end{alltt}
\end{kframe}
\includegraphics[width=\maxwidth]{figure/021-ggplot2-geoms-geom_label-17} 
\begin{kframe}\begin{alltt}
\hlcom{# To place text in the middle of each bar in a stacked barplot, you}
\hlcom{# need to do the computation yourself}
\hlstd{df} \hlkwb{<-} \hlkwd{transform}\hlstd{(df,} \hlkwc{mid_y} \hlstd{=} \hlkwd{ave}\hlstd{(df}\hlopt{$}\hlstd{y, df}\hlopt{$}\hlstd{x,} \hlkwc{FUN} \hlstd{=} \hlkwa{function}\hlstd{(}\hlkwc{val}\hlstd{)} \hlkwd{cumsum}\hlstd{(val)} \hlopt{-} \hlstd{(}\hlnum{0.5} \hlopt{*} \hlstd{val)))}

\hlkwd{ggplot}\hlstd{(}\hlkwc{data} \hlstd{= df,} \hlkwd{aes}\hlstd{(x, y,} \hlkwc{fill} \hlstd{= grp,} \hlkwc{label} \hlstd{= y))} \hlopt{+}
 \hlkwd{geom_bar}\hlstd{(}\hlkwc{stat} \hlstd{=} \hlstr{"identity"}\hlstd{)} \hlopt{+}
 \hlkwd{geom_text}\hlstd{(}\hlkwd{aes}\hlstd{(}\hlkwc{y} \hlstd{= mid_y))}
\end{alltt}
\end{kframe}
\includegraphics[width=\maxwidth]{figure/021-ggplot2-geoms-geom_label-18} 
\begin{kframe}\begin{alltt}
\hlcom{# Justification -------------------------------------------------------------}
\hlstd{df} \hlkwb{<-} \hlkwd{data.frame}\hlstd{(}
  \hlkwc{x} \hlstd{=} \hlkwd{c}\hlstd{(}\hlnum{1}\hlstd{,} \hlnum{1}\hlstd{,} \hlnum{2}\hlstd{,} \hlnum{2}\hlstd{,} \hlnum{1.5}\hlstd{),}
  \hlkwc{y} \hlstd{=} \hlkwd{c}\hlstd{(}\hlnum{1}\hlstd{,} \hlnum{2}\hlstd{,} \hlnum{1}\hlstd{,} \hlnum{2}\hlstd{,} \hlnum{1.5}\hlstd{),}
  \hlkwc{text} \hlstd{=} \hlkwd{c}\hlstd{(}\hlstr{"bottom-left"}\hlstd{,} \hlstr{"bottom-right"}\hlstd{,} \hlstr{"top-left"}\hlstd{,} \hlstr{"top-right"}\hlstd{,} \hlstr{"center"}\hlstd{)}
\hlstd{)}
\hlkwd{ggplot}\hlstd{(df,} \hlkwd{aes}\hlstd{(x, y))} \hlopt{+}
  \hlkwd{geom_text}\hlstd{(}\hlkwd{aes}\hlstd{(}\hlkwc{label} \hlstd{= text))}
\end{alltt}
\end{kframe}
\includegraphics[width=\maxwidth]{figure/021-ggplot2-geoms-geom_label-19} 
\begin{kframe}\begin{alltt}
\hlkwd{ggplot}\hlstd{(df,} \hlkwd{aes}\hlstd{(x, y))} \hlopt{+}
  \hlkwd{geom_text}\hlstd{(}\hlkwd{aes}\hlstd{(}\hlkwc{label} \hlstd{= text),} \hlkwc{vjust} \hlstd{=} \hlstr{"inward"}\hlstd{,} \hlkwc{hjust} \hlstd{=} \hlstr{"inward"}\hlstd{)}
\end{alltt}
\end{kframe}
\includegraphics[width=\maxwidth]{figure/021-ggplot2-geoms-geom_label-20} 
\begin{kframe}\begin{alltt}
\hlcom{## End(No test)}
\end{alltt}
\end{kframe}
\end{knitrout}


\section{geom\_line}

\begin{knitrout}
\definecolor{shadecolor}{rgb}{0.969, 0.969, 0.969}\color{fgcolor}\begin{kframe}
\begin{alltt}
\hlcom{### Name: geom_path}
\hlcom{### Title: Connect observations.}
\hlcom{### Aliases: geom_line geom_path geom_step}

\hlcom{### ** Examples}

\hlcom{# geom_line() is suitable for time series}
\hlkwd{ggplot}\hlstd{(economics,} \hlkwd{aes}\hlstd{(date, unemploy))} \hlopt{+} \hlkwd{geom_line}\hlstd{()}
\end{alltt}
\end{kframe}
\includegraphics[width=\maxwidth]{figure/021-ggplot2-geoms-geom_line-1} 
\begin{kframe}\begin{alltt}
\hlkwd{ggplot}\hlstd{(economics_long,} \hlkwd{aes}\hlstd{(date, value01,} \hlkwc{colour} \hlstd{= variable))} \hlopt{+}
  \hlkwd{geom_line}\hlstd{()}
\end{alltt}
\end{kframe}
\includegraphics[width=\maxwidth]{figure/021-ggplot2-geoms-geom_line-2} 
\begin{kframe}\begin{alltt}
\hlcom{# geom_step() is useful when you want to highlight exactly when}
\hlcom{# the y value chanes}
\hlstd{recent} \hlkwb{<-} \hlstd{economics[economics}\hlopt{$}\hlstd{date} \hlopt{>} \hlkwd{as.Date}\hlstd{(}\hlstr{"2013-01-01"}\hlstd{), ]}
\hlkwd{ggplot}\hlstd{(recent,} \hlkwd{aes}\hlstd{(date, unemploy))} \hlopt{+} \hlkwd{geom_line}\hlstd{()}
\end{alltt}
\end{kframe}
\includegraphics[width=\maxwidth]{figure/021-ggplot2-geoms-geom_line-3} 
\begin{kframe}\begin{alltt}
\hlkwd{ggplot}\hlstd{(recent,} \hlkwd{aes}\hlstd{(date, unemploy))} \hlopt{+} \hlkwd{geom_step}\hlstd{()}
\end{alltt}
\end{kframe}
\includegraphics[width=\maxwidth]{figure/021-ggplot2-geoms-geom_line-4} 
\begin{kframe}\begin{alltt}
\hlcom{# geom_path lets you explore how two variables are related over time,}
\hlcom{# e.g. unemployment and personal savings rate}
\hlstd{m} \hlkwb{<-} \hlkwd{ggplot}\hlstd{(economics,} \hlkwd{aes}\hlstd{(unemploy}\hlopt{/}\hlstd{pop, psavert))}
\hlstd{m} \hlopt{+} \hlkwd{geom_path}\hlstd{()}
\end{alltt}
\end{kframe}
\includegraphics[width=\maxwidth]{figure/021-ggplot2-geoms-geom_line-5} 
\begin{kframe}\begin{alltt}
\hlstd{m} \hlopt{+} \hlkwd{geom_path}\hlstd{(}\hlkwd{aes}\hlstd{(}\hlkwc{colour} \hlstd{=} \hlkwd{as.numeric}\hlstd{(date)))}
\end{alltt}
\end{kframe}
\includegraphics[width=\maxwidth]{figure/021-ggplot2-geoms-geom_line-6} 
\begin{kframe}\begin{alltt}
\hlcom{# Changing parameters ----------------------------------------------}
\hlkwd{ggplot}\hlstd{(economics,} \hlkwd{aes}\hlstd{(date, unemploy))} \hlopt{+}
  \hlkwd{geom_line}\hlstd{(}\hlkwc{colour} \hlstd{=} \hlstr{"red"}\hlstd{)}
\end{alltt}
\end{kframe}
\includegraphics[width=\maxwidth]{figure/021-ggplot2-geoms-geom_line-7} 
\begin{kframe}\begin{alltt}
\hlcom{# Use the arrow parameter to add an arrow to the line}
\hlcom{# See ?arrow for more details}
\hlstd{c} \hlkwb{<-} \hlkwd{ggplot}\hlstd{(economics,} \hlkwd{aes}\hlstd{(}\hlkwc{x} \hlstd{= date,} \hlkwc{y} \hlstd{= pop))}
\hlstd{c} \hlopt{+} \hlkwd{geom_line}\hlstd{(}\hlkwc{arrow} \hlstd{=} \hlkwd{arrow}\hlstd{())}
\end{alltt}
\end{kframe}
\includegraphics[width=\maxwidth]{figure/021-ggplot2-geoms-geom_line-8} 
\begin{kframe}\begin{alltt}
\hlstd{c} \hlopt{+} \hlkwd{geom_line}\hlstd{(}
  \hlkwc{arrow} \hlstd{=} \hlkwd{arrow}\hlstd{(}\hlkwc{angle} \hlstd{=} \hlnum{15}\hlstd{,} \hlkwc{ends} \hlstd{=} \hlstr{"both"}\hlstd{,} \hlkwc{type} \hlstd{=} \hlstr{"closed"}\hlstd{)}
\hlstd{)}
\end{alltt}
\end{kframe}
\includegraphics[width=\maxwidth]{figure/021-ggplot2-geoms-geom_line-9} 
\begin{kframe}\begin{alltt}
\hlcom{# Control line join parameters}
\hlstd{df} \hlkwb{<-} \hlkwd{data.frame}\hlstd{(}\hlkwc{x} \hlstd{=} \hlnum{1}\hlopt{:}\hlnum{3}\hlstd{,} \hlkwc{y} \hlstd{=} \hlkwd{c}\hlstd{(}\hlnum{4}\hlstd{,} \hlnum{1}\hlstd{,} \hlnum{9}\hlstd{))}
\hlstd{base} \hlkwb{<-} \hlkwd{ggplot}\hlstd{(df,} \hlkwd{aes}\hlstd{(x, y))}
\hlstd{base} \hlopt{+} \hlkwd{geom_path}\hlstd{(}\hlkwc{size} \hlstd{=} \hlnum{10}\hlstd{)}
\end{alltt}
\end{kframe}
\includegraphics[width=\maxwidth]{figure/021-ggplot2-geoms-geom_line-10} 
\begin{kframe}\begin{alltt}
\hlstd{base} \hlopt{+} \hlkwd{geom_path}\hlstd{(}\hlkwc{size} \hlstd{=} \hlnum{10}\hlstd{,} \hlkwc{lineend} \hlstd{=} \hlstr{"round"}\hlstd{)}
\end{alltt}
\end{kframe}
\includegraphics[width=\maxwidth]{figure/021-ggplot2-geoms-geom_line-11} 
\begin{kframe}\begin{alltt}
\hlstd{base} \hlopt{+} \hlkwd{geom_path}\hlstd{(}\hlkwc{size} \hlstd{=} \hlnum{10}\hlstd{,} \hlkwc{linejoin} \hlstd{=} \hlstr{"mitre"}\hlstd{,} \hlkwc{lineend} \hlstd{=} \hlstr{"butt"}\hlstd{)}
\end{alltt}
\end{kframe}
\includegraphics[width=\maxwidth]{figure/021-ggplot2-geoms-geom_line-12} 
\begin{kframe}\begin{alltt}
\hlcom{# NAs break the line. Use na.rm = T to suppress the warning message}
\hlstd{df} \hlkwb{<-} \hlkwd{data.frame}\hlstd{(}
  \hlkwc{x} \hlstd{=} \hlnum{1}\hlopt{:}\hlnum{5}\hlstd{,}
  \hlkwc{y1} \hlstd{=} \hlkwd{c}\hlstd{(}\hlnum{1}\hlstd{,} \hlnum{2}\hlstd{,} \hlnum{3}\hlstd{,} \hlnum{4}\hlstd{,} \hlnum{NA}\hlstd{),}
  \hlkwc{y2} \hlstd{=} \hlkwd{c}\hlstd{(}\hlnum{NA}\hlstd{,} \hlnum{2}\hlstd{,} \hlnum{3}\hlstd{,} \hlnum{4}\hlstd{,} \hlnum{5}\hlstd{),}
  \hlkwc{y3} \hlstd{=} \hlkwd{c}\hlstd{(}\hlnum{1}\hlstd{,} \hlnum{2}\hlstd{,} \hlnum{NA}\hlstd{,} \hlnum{4}\hlstd{,} \hlnum{5}\hlstd{)}
\hlstd{)}
\hlkwd{ggplot}\hlstd{(df,} \hlkwd{aes}\hlstd{(x, y1))} \hlopt{+} \hlkwd{geom_point}\hlstd{()} \hlopt{+} \hlkwd{geom_line}\hlstd{()}
\end{alltt}


{\ttfamily\noindent\color{warningcolor}{\#\# Warning: Removed 1 rows containing missing values (geom\_point).}}

{\ttfamily\noindent\color{warningcolor}{\#\# Warning: Removed 1 rows containing missing values (geom\_path).}}\end{kframe}
\includegraphics[width=\maxwidth]{figure/021-ggplot2-geoms-geom_line-13} 
\begin{kframe}\begin{alltt}
\hlkwd{ggplot}\hlstd{(df,} \hlkwd{aes}\hlstd{(x, y2))} \hlopt{+} \hlkwd{geom_point}\hlstd{()} \hlopt{+} \hlkwd{geom_line}\hlstd{()}
\end{alltt}


{\ttfamily\noindent\color{warningcolor}{\#\# Warning: Removed 1 rows containing missing values (geom\_point).}}

{\ttfamily\noindent\color{warningcolor}{\#\# Warning: Removed 1 rows containing missing values (geom\_path).}}\end{kframe}
\includegraphics[width=\maxwidth]{figure/021-ggplot2-geoms-geom_line-14} 
\begin{kframe}\begin{alltt}
\hlkwd{ggplot}\hlstd{(df,} \hlkwd{aes}\hlstd{(x, y3))} \hlopt{+} \hlkwd{geom_point}\hlstd{()} \hlopt{+} \hlkwd{geom_line}\hlstd{()}
\end{alltt}


{\ttfamily\noindent\color{warningcolor}{\#\# Warning: Removed 1 rows containing missing values (geom\_point).}}\end{kframe}
\includegraphics[width=\maxwidth]{figure/021-ggplot2-geoms-geom_line-15} 
\begin{kframe}\begin{alltt}
\hlcom{## No test: }
\hlcom{# Setting line type vs colour/size}
\hlcom{# Line type needs to be applied to a line as a whole, so it can}
\hlcom{# not be used with colour or size that vary across a line}
\hlstd{x} \hlkwb{<-} \hlkwd{seq}\hlstd{(}\hlnum{0.01}\hlstd{,} \hlnum{.99}\hlstd{,} \hlkwc{length.out} \hlstd{=} \hlnum{100}\hlstd{)}
\hlstd{df} \hlkwb{<-} \hlkwd{data.frame}\hlstd{(}
  \hlkwc{x} \hlstd{=} \hlkwd{rep}\hlstd{(x,} \hlnum{2}\hlstd{),}
  \hlkwc{y} \hlstd{=} \hlkwd{c}\hlstd{(}\hlkwd{qlogis}\hlstd{(x),} \hlnum{2} \hlopt{*} \hlkwd{qlogis}\hlstd{(x)),}
  \hlkwc{group} \hlstd{=} \hlkwd{rep}\hlstd{(}\hlkwd{c}\hlstd{(}\hlstr{"a"}\hlstd{,}\hlstr{"b"}\hlstd{),}
  \hlkwc{each} \hlstd{=} \hlnum{100}\hlstd{)}
\hlstd{)}
\hlstd{p} \hlkwb{<-} \hlkwd{ggplot}\hlstd{(df,} \hlkwd{aes}\hlstd{(}\hlkwc{x}\hlstd{=x,} \hlkwc{y}\hlstd{=y,} \hlkwc{group}\hlstd{=group))}
\hlcom{# These work}
\hlstd{p} \hlopt{+} \hlkwd{geom_line}\hlstd{(}\hlkwc{linetype} \hlstd{=} \hlnum{2}\hlstd{)}
\end{alltt}
\end{kframe}
\includegraphics[width=\maxwidth]{figure/021-ggplot2-geoms-geom_line-16} 
\begin{kframe}\begin{alltt}
\hlstd{p} \hlopt{+} \hlkwd{geom_line}\hlstd{(}\hlkwd{aes}\hlstd{(}\hlkwc{colour} \hlstd{= group),} \hlkwc{linetype} \hlstd{=} \hlnum{2}\hlstd{)}
\end{alltt}
\end{kframe}
\includegraphics[width=\maxwidth]{figure/021-ggplot2-geoms-geom_line-17} 
\begin{kframe}\begin{alltt}
\hlstd{p} \hlopt{+} \hlkwd{geom_line}\hlstd{(}\hlkwd{aes}\hlstd{(}\hlkwc{colour} \hlstd{= x))}
\end{alltt}
\end{kframe}
\includegraphics[width=\maxwidth]{figure/021-ggplot2-geoms-geom_line-18} 
\begin{kframe}\begin{alltt}
\hlcom{# But this doesn't}
\hlkwd{should_stop}\hlstd{(p} \hlopt{+} \hlkwd{geom_line}\hlstd{(}\hlkwd{aes}\hlstd{(}\hlkwc{colour} \hlstd{= x),} \hlkwc{linetype}\hlstd{=}\hlnum{2}\hlstd{))}
\end{alltt}
\end{kframe}
\includegraphics[width=\maxwidth]{figure/021-ggplot2-geoms-geom_line-19} 
\begin{kframe}\begin{alltt}
\hlcom{## End(No test)}
\end{alltt}
\end{kframe}
\end{knitrout}


\section{geom\_linerange}

\begin{knitrout}
\definecolor{shadecolor}{rgb}{0.969, 0.969, 0.969}\color{fgcolor}\begin{kframe}
\begin{alltt}
\hlcom{### Name: geom_crossbar}
\hlcom{### Title: Vertical intervals: lines, crossbars & errorbars.}
\hlcom{### Aliases: geom_crossbar geom_errorbar geom_linerange geom_pointrange}

\hlcom{### ** Examples}

\hlcom{#' # Create a simple example dataset}
\hlstd{df} \hlkwb{<-} \hlkwd{data.frame}\hlstd{(}
  \hlkwc{trt} \hlstd{=} \hlkwd{factor}\hlstd{(}\hlkwd{c}\hlstd{(}\hlnum{1}\hlstd{,} \hlnum{1}\hlstd{,} \hlnum{2}\hlstd{,} \hlnum{2}\hlstd{)),}
  \hlkwc{resp} \hlstd{=} \hlkwd{c}\hlstd{(}\hlnum{1}\hlstd{,} \hlnum{5}\hlstd{,} \hlnum{3}\hlstd{,} \hlnum{4}\hlstd{),}
  \hlkwc{group} \hlstd{=} \hlkwd{factor}\hlstd{(}\hlkwd{c}\hlstd{(}\hlnum{1}\hlstd{,} \hlnum{2}\hlstd{,} \hlnum{1}\hlstd{,} \hlnum{2}\hlstd{)),}
  \hlkwc{upper} \hlstd{=} \hlkwd{c}\hlstd{(}\hlnum{1.1}\hlstd{,} \hlnum{5.3}\hlstd{,} \hlnum{3.3}\hlstd{,} \hlnum{4.2}\hlstd{),}
  \hlkwc{lower} \hlstd{=} \hlkwd{c}\hlstd{(}\hlnum{0.8}\hlstd{,} \hlnum{4.6}\hlstd{,} \hlnum{2.4}\hlstd{,} \hlnum{3.6}\hlstd{)}
\hlstd{)}

\hlstd{p} \hlkwb{<-} \hlkwd{ggplot}\hlstd{(df,} \hlkwd{aes}\hlstd{(trt, resp,} \hlkwc{colour} \hlstd{= group))}
\hlstd{p} \hlopt{+} \hlkwd{geom_linerange}\hlstd{(}\hlkwd{aes}\hlstd{(}\hlkwc{ymin} \hlstd{= lower,} \hlkwc{ymax} \hlstd{= upper))}
\end{alltt}
\end{kframe}
\includegraphics[width=\maxwidth]{figure/021-ggplot2-geoms-geom_linerange-1} 
\begin{kframe}\begin{alltt}
\hlstd{p} \hlopt{+} \hlkwd{geom_pointrange}\hlstd{(}\hlkwd{aes}\hlstd{(}\hlkwc{ymin} \hlstd{= lower,} \hlkwc{ymax} \hlstd{= upper))}
\end{alltt}
\end{kframe}
\includegraphics[width=\maxwidth]{figure/021-ggplot2-geoms-geom_linerange-2} 
\begin{kframe}\begin{alltt}
\hlstd{p} \hlopt{+} \hlkwd{geom_crossbar}\hlstd{(}\hlkwd{aes}\hlstd{(}\hlkwc{ymin} \hlstd{= lower,} \hlkwc{ymax} \hlstd{= upper),} \hlkwc{width} \hlstd{=} \hlnum{0.2}\hlstd{)}
\end{alltt}
\end{kframe}
\includegraphics[width=\maxwidth]{figure/021-ggplot2-geoms-geom_linerange-3} 
\begin{kframe}\begin{alltt}
\hlstd{p} \hlopt{+} \hlkwd{geom_errorbar}\hlstd{(}\hlkwd{aes}\hlstd{(}\hlkwc{ymin} \hlstd{= lower,} \hlkwc{ymax} \hlstd{= upper),} \hlkwc{width} \hlstd{=} \hlnum{0.2}\hlstd{)}
\end{alltt}
\end{kframe}
\includegraphics[width=\maxwidth]{figure/021-ggplot2-geoms-geom_linerange-4} 
\begin{kframe}\begin{alltt}
\hlcom{# Draw lines connecting group means}
\hlstd{p} \hlopt{+}
  \hlkwd{geom_line}\hlstd{(}\hlkwd{aes}\hlstd{(}\hlkwc{group} \hlstd{= group))} \hlopt{+}
  \hlkwd{geom_errorbar}\hlstd{(}\hlkwd{aes}\hlstd{(}\hlkwc{ymin} \hlstd{= lower,} \hlkwc{ymax} \hlstd{= upper),} \hlkwc{width} \hlstd{=} \hlnum{0.2}\hlstd{)}
\end{alltt}
\end{kframe}
\includegraphics[width=\maxwidth]{figure/021-ggplot2-geoms-geom_linerange-5} 
\begin{kframe}\begin{alltt}
\hlcom{# If you want to dodge bars and errorbars, you need to manually}
\hlcom{# specify the dodge width}
\hlstd{p} \hlkwb{<-} \hlkwd{ggplot}\hlstd{(df,} \hlkwd{aes}\hlstd{(trt, resp,} \hlkwc{fill} \hlstd{= group))}
\hlstd{p} \hlopt{+}
 \hlkwd{geom_bar}\hlstd{(}\hlkwc{position} \hlstd{=} \hlstr{"dodge"}\hlstd{,} \hlkwc{stat} \hlstd{=} \hlstr{"identity"}\hlstd{)} \hlopt{+}
 \hlkwd{geom_errorbar}\hlstd{(}\hlkwd{aes}\hlstd{(}\hlkwc{ymin} \hlstd{= lower,} \hlkwc{ymax} \hlstd{= upper),} \hlkwc{position} \hlstd{=} \hlstr{"dodge"}\hlstd{,} \hlkwc{width} \hlstd{=} \hlnum{0.25}\hlstd{)}
\end{alltt}
\end{kframe}
\includegraphics[width=\maxwidth]{figure/021-ggplot2-geoms-geom_linerange-6} 
\begin{kframe}\begin{alltt}
\hlcom{# Because the bars and errorbars have different widths}
\hlcom{# we need to specify how wide the objects we are dodging are}
\hlstd{dodge} \hlkwb{<-} \hlkwd{position_dodge}\hlstd{(}\hlkwc{width}\hlstd{=}\hlnum{0.9}\hlstd{)}
\hlstd{p} \hlopt{+}
  \hlkwd{geom_bar}\hlstd{(}\hlkwc{position} \hlstd{= dodge,} \hlkwc{stat} \hlstd{=} \hlstr{"identity"}\hlstd{)} \hlopt{+}
  \hlkwd{geom_errorbar}\hlstd{(}\hlkwd{aes}\hlstd{(}\hlkwc{ymin} \hlstd{= lower,} \hlkwc{ymax} \hlstd{= upper),} \hlkwc{position} \hlstd{= dodge,} \hlkwc{width} \hlstd{=} \hlnum{0.25}\hlstd{)}
\end{alltt}
\end{kframe}
\includegraphics[width=\maxwidth]{figure/021-ggplot2-geoms-geom_linerange-7} 

\end{knitrout}


\section{geom\_map}

\begin{knitrout}
\definecolor{shadecolor}{rgb}{0.969, 0.969, 0.969}\color{fgcolor}\begin{kframe}
\begin{alltt}
\hlcom{### Name: geom_map}
\hlcom{### Title: Polygons from a reference map.}
\hlcom{### Aliases: geom_map}

\hlcom{### ** Examples}

\hlcom{# When using geom_polygon, you will typically need two data frames:}
\hlcom{# one contains the coordinates of each polygon (positions),  and the}
\hlcom{# other the values associated with each polygon (values).  An id}
\hlcom{# variable links the two together}

\hlstd{ids} \hlkwb{<-} \hlkwd{factor}\hlstd{(}\hlkwd{c}\hlstd{(}\hlstr{"1.1"}\hlstd{,} \hlstr{"2.1"}\hlstd{,} \hlstr{"1.2"}\hlstd{,} \hlstr{"2.2"}\hlstd{,} \hlstr{"1.3"}\hlstd{,} \hlstr{"2.3"}\hlstd{))}

\hlstd{values} \hlkwb{<-} \hlkwd{data.frame}\hlstd{(}
  \hlkwc{id} \hlstd{= ids,}
  \hlkwc{value} \hlstd{=} \hlkwd{c}\hlstd{(}\hlnum{3}\hlstd{,} \hlnum{3.1}\hlstd{,} \hlnum{3.1}\hlstd{,} \hlnum{3.2}\hlstd{,} \hlnum{3.15}\hlstd{,} \hlnum{3.5}\hlstd{)}
\hlstd{)}

\hlstd{positions} \hlkwb{<-} \hlkwd{data.frame}\hlstd{(}
  \hlkwc{id} \hlstd{=} \hlkwd{rep}\hlstd{(ids,} \hlkwc{each} \hlstd{=} \hlnum{4}\hlstd{),}
  \hlkwc{x} \hlstd{=} \hlkwd{c}\hlstd{(}\hlnum{2}\hlstd{,} \hlnum{1}\hlstd{,} \hlnum{1.1}\hlstd{,} \hlnum{2.2}\hlstd{,} \hlnum{1}\hlstd{,} \hlnum{0}\hlstd{,} \hlnum{0.3}\hlstd{,} \hlnum{1.1}\hlstd{,} \hlnum{2.2}\hlstd{,} \hlnum{1.1}\hlstd{,} \hlnum{1.2}\hlstd{,} \hlnum{2.5}\hlstd{,} \hlnum{1.1}\hlstd{,} \hlnum{0.3}\hlstd{,}
  \hlnum{0.5}\hlstd{,} \hlnum{1.2}\hlstd{,} \hlnum{2.5}\hlstd{,} \hlnum{1.2}\hlstd{,} \hlnum{1.3}\hlstd{,} \hlnum{2.7}\hlstd{,} \hlnum{1.2}\hlstd{,} \hlnum{0.5}\hlstd{,} \hlnum{0.6}\hlstd{,} \hlnum{1.3}\hlstd{),}
  \hlkwc{y} \hlstd{=} \hlkwd{c}\hlstd{(}\hlopt{-}\hlnum{0.5}\hlstd{,} \hlnum{0}\hlstd{,} \hlnum{1}\hlstd{,} \hlnum{0.5}\hlstd{,} \hlnum{0}\hlstd{,} \hlnum{0.5}\hlstd{,} \hlnum{1.5}\hlstd{,} \hlnum{1}\hlstd{,} \hlnum{0.5}\hlstd{,} \hlnum{1}\hlstd{,} \hlnum{2.1}\hlstd{,} \hlnum{1.7}\hlstd{,} \hlnum{1}\hlstd{,} \hlnum{1.5}\hlstd{,}
  \hlnum{2.2}\hlstd{,} \hlnum{2.1}\hlstd{,} \hlnum{1.7}\hlstd{,} \hlnum{2.1}\hlstd{,} \hlnum{3.2}\hlstd{,} \hlnum{2.8}\hlstd{,} \hlnum{2.1}\hlstd{,} \hlnum{2.2}\hlstd{,} \hlnum{3.3}\hlstd{,} \hlnum{3.2}\hlstd{)}
\hlstd{)}

\hlkwd{ggplot}\hlstd{(values)} \hlopt{+} \hlkwd{geom_map}\hlstd{(}\hlkwd{aes}\hlstd{(}\hlkwc{map_id} \hlstd{= id),} \hlkwc{map} \hlstd{= positions)} \hlopt{+}
  \hlkwd{expand_limits}\hlstd{(positions)}
\end{alltt}
\end{kframe}
\includegraphics[width=\maxwidth]{figure/021-ggplot2-geoms-geom_map-1} 
\begin{kframe}\begin{alltt}
\hlkwd{ggplot}\hlstd{(values,} \hlkwd{aes}\hlstd{(}\hlkwc{fill} \hlstd{= value))} \hlopt{+}
  \hlkwd{geom_map}\hlstd{(}\hlkwd{aes}\hlstd{(}\hlkwc{map_id} \hlstd{= id),} \hlkwc{map} \hlstd{= positions)} \hlopt{+}
  \hlkwd{expand_limits}\hlstd{(positions)}
\end{alltt}
\end{kframe}
\includegraphics[width=\maxwidth]{figure/021-ggplot2-geoms-geom_map-2} 
\begin{kframe}\begin{alltt}
\hlkwd{ggplot}\hlstd{(values,} \hlkwd{aes}\hlstd{(}\hlkwc{fill} \hlstd{= value))} \hlopt{+}
  \hlkwd{geom_map}\hlstd{(}\hlkwd{aes}\hlstd{(}\hlkwc{map_id} \hlstd{= id),} \hlkwc{map} \hlstd{= positions)} \hlopt{+}
  \hlkwd{expand_limits}\hlstd{(positions)} \hlopt{+} \hlkwd{ylim}\hlstd{(}\hlnum{0}\hlstd{,} \hlnum{3}\hlstd{)}
\end{alltt}
\end{kframe}
\includegraphics[width=\maxwidth]{figure/021-ggplot2-geoms-geom_map-3} 
\begin{kframe}\begin{alltt}
\hlcom{# Better example}
\hlstd{crimes} \hlkwb{<-} \hlkwd{data.frame}\hlstd{(}\hlkwc{state} \hlstd{=} \hlkwd{tolower}\hlstd{(}\hlkwd{rownames}\hlstd{(USArrests)), USArrests)}
\hlstd{crimesm} \hlkwb{<-} \hlstd{reshape2}\hlopt{::}\hlkwd{melt}\hlstd{(crimes,} \hlkwc{id} \hlstd{=} \hlnum{1}\hlstd{)}
\hlkwa{if} \hlstd{(}\hlkwd{require}\hlstd{(maps)) \{}
  \hlstd{states_map} \hlkwb{<-} \hlkwd{map_data}\hlstd{(}\hlstr{"state"}\hlstd{)}
  \hlkwd{ggplot}\hlstd{(crimes,} \hlkwd{aes}\hlstd{(}\hlkwc{map_id} \hlstd{= state))} \hlopt{+}
    \hlkwd{geom_map}\hlstd{(}\hlkwd{aes}\hlstd{(}\hlkwc{fill} \hlstd{= Murder),} \hlkwc{map} \hlstd{= states_map)} \hlopt{+}
    \hlkwd{expand_limits}\hlstd{(}\hlkwc{x} \hlstd{= states_map}\hlopt{$}\hlstd{long,} \hlkwc{y} \hlstd{= states_map}\hlopt{$}\hlstd{lat)}

  \hlkwd{last_plot}\hlstd{()} \hlopt{+} \hlkwd{coord_map}\hlstd{()}
  \hlkwd{ggplot}\hlstd{(crimesm,} \hlkwd{aes}\hlstd{(}\hlkwc{map_id} \hlstd{= state))} \hlopt{+}
    \hlkwd{geom_map}\hlstd{(}\hlkwd{aes}\hlstd{(}\hlkwc{fill} \hlstd{= value),} \hlkwc{map} \hlstd{= states_map)} \hlopt{+}
    \hlkwd{expand_limits}\hlstd{(}\hlkwc{x} \hlstd{= states_map}\hlopt{$}\hlstd{long,} \hlkwc{y} \hlstd{= states_map}\hlopt{$}\hlstd{lat)} \hlopt{+}
    \hlkwd{facet_wrap}\hlstd{(} \hlopt{~} \hlstd{variable)}
\hlstd{\}}
\end{alltt}
\end{kframe}
\includegraphics[width=\maxwidth]{figure/021-ggplot2-geoms-geom_map-4} 

\end{knitrout}


\section{geom\_path}

\begin{knitrout}
\definecolor{shadecolor}{rgb}{0.969, 0.969, 0.969}\color{fgcolor}\begin{kframe}
\begin{alltt}
\hlcom{### Name: geom_path}
\hlcom{### Title: Connect observations.}
\hlcom{### Aliases: geom_line geom_path geom_step}

\hlcom{### ** Examples}

\hlcom{# geom_line() is suitable for time series}
\hlkwd{ggplot}\hlstd{(economics,} \hlkwd{aes}\hlstd{(date, unemploy))} \hlopt{+} \hlkwd{geom_line}\hlstd{()}
\end{alltt}
\end{kframe}
\includegraphics[width=\maxwidth]{figure/021-ggplot2-geoms-geom_path-1} 
\begin{kframe}\begin{alltt}
\hlkwd{ggplot}\hlstd{(economics_long,} \hlkwd{aes}\hlstd{(date, value01,} \hlkwc{colour} \hlstd{= variable))} \hlopt{+}
  \hlkwd{geom_line}\hlstd{()}
\end{alltt}
\end{kframe}
\includegraphics[width=\maxwidth]{figure/021-ggplot2-geoms-geom_path-2} 
\begin{kframe}\begin{alltt}
\hlcom{# geom_step() is useful when you want to highlight exactly when}
\hlcom{# the y value chanes}
\hlstd{recent} \hlkwb{<-} \hlstd{economics[economics}\hlopt{$}\hlstd{date} \hlopt{>} \hlkwd{as.Date}\hlstd{(}\hlstr{"2013-01-01"}\hlstd{), ]}
\hlkwd{ggplot}\hlstd{(recent,} \hlkwd{aes}\hlstd{(date, unemploy))} \hlopt{+} \hlkwd{geom_line}\hlstd{()}
\end{alltt}
\end{kframe}
\includegraphics[width=\maxwidth]{figure/021-ggplot2-geoms-geom_path-3} 
\begin{kframe}\begin{alltt}
\hlkwd{ggplot}\hlstd{(recent,} \hlkwd{aes}\hlstd{(date, unemploy))} \hlopt{+} \hlkwd{geom_step}\hlstd{()}
\end{alltt}
\end{kframe}
\includegraphics[width=\maxwidth]{figure/021-ggplot2-geoms-geom_path-4} 
\begin{kframe}\begin{alltt}
\hlcom{# geom_path lets you explore how two variables are related over time,}
\hlcom{# e.g. unemployment and personal savings rate}
\hlstd{m} \hlkwb{<-} \hlkwd{ggplot}\hlstd{(economics,} \hlkwd{aes}\hlstd{(unemploy}\hlopt{/}\hlstd{pop, psavert))}
\hlstd{m} \hlopt{+} \hlkwd{geom_path}\hlstd{()}
\end{alltt}
\end{kframe}
\includegraphics[width=\maxwidth]{figure/021-ggplot2-geoms-geom_path-5} 
\begin{kframe}\begin{alltt}
\hlstd{m} \hlopt{+} \hlkwd{geom_path}\hlstd{(}\hlkwd{aes}\hlstd{(}\hlkwc{colour} \hlstd{=} \hlkwd{as.numeric}\hlstd{(date)))}
\end{alltt}
\end{kframe}
\includegraphics[width=\maxwidth]{figure/021-ggplot2-geoms-geom_path-6} 
\begin{kframe}\begin{alltt}
\hlcom{# Changing parameters ----------------------------------------------}
\hlkwd{ggplot}\hlstd{(economics,} \hlkwd{aes}\hlstd{(date, unemploy))} \hlopt{+}
  \hlkwd{geom_line}\hlstd{(}\hlkwc{colour} \hlstd{=} \hlstr{"red"}\hlstd{)}
\end{alltt}
\end{kframe}
\includegraphics[width=\maxwidth]{figure/021-ggplot2-geoms-geom_path-7} 
\begin{kframe}\begin{alltt}
\hlcom{# Use the arrow parameter to add an arrow to the line}
\hlcom{# See ?arrow for more details}
\hlstd{c} \hlkwb{<-} \hlkwd{ggplot}\hlstd{(economics,} \hlkwd{aes}\hlstd{(}\hlkwc{x} \hlstd{= date,} \hlkwc{y} \hlstd{= pop))}
\hlstd{c} \hlopt{+} \hlkwd{geom_line}\hlstd{(}\hlkwc{arrow} \hlstd{=} \hlkwd{arrow}\hlstd{())}
\end{alltt}
\end{kframe}
\includegraphics[width=\maxwidth]{figure/021-ggplot2-geoms-geom_path-8} 
\begin{kframe}\begin{alltt}
\hlstd{c} \hlopt{+} \hlkwd{geom_line}\hlstd{(}
  \hlkwc{arrow} \hlstd{=} \hlkwd{arrow}\hlstd{(}\hlkwc{angle} \hlstd{=} \hlnum{15}\hlstd{,} \hlkwc{ends} \hlstd{=} \hlstr{"both"}\hlstd{,} \hlkwc{type} \hlstd{=} \hlstr{"closed"}\hlstd{)}
\hlstd{)}
\end{alltt}
\end{kframe}
\includegraphics[width=\maxwidth]{figure/021-ggplot2-geoms-geom_path-9} 
\begin{kframe}\begin{alltt}
\hlcom{# Control line join parameters}
\hlstd{df} \hlkwb{<-} \hlkwd{data.frame}\hlstd{(}\hlkwc{x} \hlstd{=} \hlnum{1}\hlopt{:}\hlnum{3}\hlstd{,} \hlkwc{y} \hlstd{=} \hlkwd{c}\hlstd{(}\hlnum{4}\hlstd{,} \hlnum{1}\hlstd{,} \hlnum{9}\hlstd{))}
\hlstd{base} \hlkwb{<-} \hlkwd{ggplot}\hlstd{(df,} \hlkwd{aes}\hlstd{(x, y))}
\hlstd{base} \hlopt{+} \hlkwd{geom_path}\hlstd{(}\hlkwc{size} \hlstd{=} \hlnum{10}\hlstd{)}
\end{alltt}
\end{kframe}
\includegraphics[width=\maxwidth]{figure/021-ggplot2-geoms-geom_path-10} 
\begin{kframe}\begin{alltt}
\hlstd{base} \hlopt{+} \hlkwd{geom_path}\hlstd{(}\hlkwc{size} \hlstd{=} \hlnum{10}\hlstd{,} \hlkwc{lineend} \hlstd{=} \hlstr{"round"}\hlstd{)}
\end{alltt}
\end{kframe}
\includegraphics[width=\maxwidth]{figure/021-ggplot2-geoms-geom_path-11} 
\begin{kframe}\begin{alltt}
\hlstd{base} \hlopt{+} \hlkwd{geom_path}\hlstd{(}\hlkwc{size} \hlstd{=} \hlnum{10}\hlstd{,} \hlkwc{linejoin} \hlstd{=} \hlstr{"mitre"}\hlstd{,} \hlkwc{lineend} \hlstd{=} \hlstr{"butt"}\hlstd{)}
\end{alltt}
\end{kframe}
\includegraphics[width=\maxwidth]{figure/021-ggplot2-geoms-geom_path-12} 
\begin{kframe}\begin{alltt}
\hlcom{# NAs break the line. Use na.rm = T to suppress the warning message}
\hlstd{df} \hlkwb{<-} \hlkwd{data.frame}\hlstd{(}
  \hlkwc{x} \hlstd{=} \hlnum{1}\hlopt{:}\hlnum{5}\hlstd{,}
  \hlkwc{y1} \hlstd{=} \hlkwd{c}\hlstd{(}\hlnum{1}\hlstd{,} \hlnum{2}\hlstd{,} \hlnum{3}\hlstd{,} \hlnum{4}\hlstd{,} \hlnum{NA}\hlstd{),}
  \hlkwc{y2} \hlstd{=} \hlkwd{c}\hlstd{(}\hlnum{NA}\hlstd{,} \hlnum{2}\hlstd{,} \hlnum{3}\hlstd{,} \hlnum{4}\hlstd{,} \hlnum{5}\hlstd{),}
  \hlkwc{y3} \hlstd{=} \hlkwd{c}\hlstd{(}\hlnum{1}\hlstd{,} \hlnum{2}\hlstd{,} \hlnum{NA}\hlstd{,} \hlnum{4}\hlstd{,} \hlnum{5}\hlstd{)}
\hlstd{)}
\hlkwd{ggplot}\hlstd{(df,} \hlkwd{aes}\hlstd{(x, y1))} \hlopt{+} \hlkwd{geom_point}\hlstd{()} \hlopt{+} \hlkwd{geom_line}\hlstd{()}
\end{alltt}


{\ttfamily\noindent\color{warningcolor}{\#\# Warning: Removed 1 rows containing missing values (geom\_point).}}

{\ttfamily\noindent\color{warningcolor}{\#\# Warning: Removed 1 rows containing missing values (geom\_path).}}\end{kframe}
\includegraphics[width=\maxwidth]{figure/021-ggplot2-geoms-geom_path-13} 
\begin{kframe}\begin{alltt}
\hlkwd{ggplot}\hlstd{(df,} \hlkwd{aes}\hlstd{(x, y2))} \hlopt{+} \hlkwd{geom_point}\hlstd{()} \hlopt{+} \hlkwd{geom_line}\hlstd{()}
\end{alltt}


{\ttfamily\noindent\color{warningcolor}{\#\# Warning: Removed 1 rows containing missing values (geom\_point).}}

{\ttfamily\noindent\color{warningcolor}{\#\# Warning: Removed 1 rows containing missing values (geom\_path).}}\end{kframe}
\includegraphics[width=\maxwidth]{figure/021-ggplot2-geoms-geom_path-14} 
\begin{kframe}\begin{alltt}
\hlkwd{ggplot}\hlstd{(df,} \hlkwd{aes}\hlstd{(x, y3))} \hlopt{+} \hlkwd{geom_point}\hlstd{()} \hlopt{+} \hlkwd{geom_line}\hlstd{()}
\end{alltt}


{\ttfamily\noindent\color{warningcolor}{\#\# Warning: Removed 1 rows containing missing values (geom\_point).}}\end{kframe}
\includegraphics[width=\maxwidth]{figure/021-ggplot2-geoms-geom_path-15} 
\begin{kframe}\begin{alltt}
\hlcom{## No test: }
\hlcom{# Setting line type vs colour/size}
\hlcom{# Line type needs to be applied to a line as a whole, so it can}
\hlcom{# not be used with colour or size that vary across a line}
\hlstd{x} \hlkwb{<-} \hlkwd{seq}\hlstd{(}\hlnum{0.01}\hlstd{,} \hlnum{.99}\hlstd{,} \hlkwc{length.out} \hlstd{=} \hlnum{100}\hlstd{)}
\hlstd{df} \hlkwb{<-} \hlkwd{data.frame}\hlstd{(}
  \hlkwc{x} \hlstd{=} \hlkwd{rep}\hlstd{(x,} \hlnum{2}\hlstd{),}
  \hlkwc{y} \hlstd{=} \hlkwd{c}\hlstd{(}\hlkwd{qlogis}\hlstd{(x),} \hlnum{2} \hlopt{*} \hlkwd{qlogis}\hlstd{(x)),}
  \hlkwc{group} \hlstd{=} \hlkwd{rep}\hlstd{(}\hlkwd{c}\hlstd{(}\hlstr{"a"}\hlstd{,}\hlstr{"b"}\hlstd{),}
  \hlkwc{each} \hlstd{=} \hlnum{100}\hlstd{)}
\hlstd{)}
\hlstd{p} \hlkwb{<-} \hlkwd{ggplot}\hlstd{(df,} \hlkwd{aes}\hlstd{(}\hlkwc{x}\hlstd{=x,} \hlkwc{y}\hlstd{=y,} \hlkwc{group}\hlstd{=group))}
\hlcom{# These work}
\hlstd{p} \hlopt{+} \hlkwd{geom_line}\hlstd{(}\hlkwc{linetype} \hlstd{=} \hlnum{2}\hlstd{)}
\end{alltt}
\end{kframe}
\includegraphics[width=\maxwidth]{figure/021-ggplot2-geoms-geom_path-16} 
\begin{kframe}\begin{alltt}
\hlstd{p} \hlopt{+} \hlkwd{geom_line}\hlstd{(}\hlkwd{aes}\hlstd{(}\hlkwc{colour} \hlstd{= group),} \hlkwc{linetype} \hlstd{=} \hlnum{2}\hlstd{)}
\end{alltt}
\end{kframe}
\includegraphics[width=\maxwidth]{figure/021-ggplot2-geoms-geom_path-17} 
\begin{kframe}\begin{alltt}
\hlstd{p} \hlopt{+} \hlkwd{geom_line}\hlstd{(}\hlkwd{aes}\hlstd{(}\hlkwc{colour} \hlstd{= x))}
\end{alltt}
\end{kframe}
\includegraphics[width=\maxwidth]{figure/021-ggplot2-geoms-geom_path-18} 
\begin{kframe}\begin{alltt}
\hlcom{# But this doesn't}
\hlkwd{should_stop}\hlstd{(p} \hlopt{+} \hlkwd{geom_line}\hlstd{(}\hlkwd{aes}\hlstd{(}\hlkwc{colour} \hlstd{= x),} \hlkwc{linetype}\hlstd{=}\hlnum{2}\hlstd{))}
\end{alltt}
\end{kframe}
\includegraphics[width=\maxwidth]{figure/021-ggplot2-geoms-geom_path-19} 
\begin{kframe}\begin{alltt}
\hlcom{## End(No test)}
\end{alltt}
\end{kframe}
\end{knitrout}


\section{geom\_point}

\begin{knitrout}
\definecolor{shadecolor}{rgb}{0.969, 0.969, 0.969}\color{fgcolor}\begin{kframe}
\begin{alltt}
\hlcom{### Name: geom_point}
\hlcom{### Title: Points, as for a scatterplot}
\hlcom{### Aliases: geom_point}

\hlcom{### ** Examples}

\hlstd{p} \hlkwb{<-} \hlkwd{ggplot}\hlstd{(mtcars,} \hlkwd{aes}\hlstd{(wt, mpg))}
\hlstd{p} \hlopt{+} \hlkwd{geom_point}\hlstd{()}
\end{alltt}
\end{kframe}
\includegraphics[width=\maxwidth]{figure/021-ggplot2-geoms-geom_point-1} 
\begin{kframe}\begin{alltt}
\hlcom{# Add aesthetic mappings}
\hlstd{p} \hlopt{+} \hlkwd{geom_point}\hlstd{(}\hlkwd{aes}\hlstd{(}\hlkwc{colour} \hlstd{=} \hlkwd{factor}\hlstd{(cyl)))}
\end{alltt}
\end{kframe}
\includegraphics[width=\maxwidth]{figure/021-ggplot2-geoms-geom_point-2} 
\begin{kframe}\begin{alltt}
\hlstd{p} \hlopt{+} \hlkwd{geom_point}\hlstd{(}\hlkwd{aes}\hlstd{(}\hlkwc{shape} \hlstd{=} \hlkwd{factor}\hlstd{(cyl)))}
\end{alltt}
\end{kframe}
\includegraphics[width=\maxwidth]{figure/021-ggplot2-geoms-geom_point-3} 
\begin{kframe}\begin{alltt}
\hlstd{p} \hlopt{+} \hlkwd{geom_point}\hlstd{(}\hlkwd{aes}\hlstd{(}\hlkwc{size} \hlstd{= qsec))}
\end{alltt}
\end{kframe}
\includegraphics[width=\maxwidth]{figure/021-ggplot2-geoms-geom_point-4} 
\begin{kframe}\begin{alltt}
\hlcom{# Change scales}
\hlstd{p} \hlopt{+} \hlkwd{geom_point}\hlstd{(}\hlkwd{aes}\hlstd{(}\hlkwc{colour} \hlstd{= cyl))} \hlopt{+} \hlkwd{scale_colour_gradient}\hlstd{(}\hlkwc{low} \hlstd{=} \hlstr{"blue"}\hlstd{)}
\end{alltt}
\end{kframe}
\includegraphics[width=\maxwidth]{figure/021-ggplot2-geoms-geom_point-5} 
\begin{kframe}\begin{alltt}
\hlstd{p} \hlopt{+} \hlkwd{geom_point}\hlstd{(}\hlkwd{aes}\hlstd{(}\hlkwc{shape} \hlstd{=} \hlkwd{factor}\hlstd{(cyl)))} \hlopt{+} \hlkwd{scale_shape}\hlstd{(}\hlkwc{solid} \hlstd{=} \hlnum{FALSE}\hlstd{)}
\end{alltt}
\end{kframe}
\includegraphics[width=\maxwidth]{figure/021-ggplot2-geoms-geom_point-6} 
\begin{kframe}\begin{alltt}
\hlcom{# Set aesthetics to fixed value}
\hlkwd{ggplot}\hlstd{(mtcars,} \hlkwd{aes}\hlstd{(wt, mpg))} \hlopt{+} \hlkwd{geom_point}\hlstd{(}\hlkwc{colour} \hlstd{=} \hlstr{"red"}\hlstd{,} \hlkwc{size} \hlstd{=} \hlnum{3}\hlstd{)}
\end{alltt}
\end{kframe}
\includegraphics[width=\maxwidth]{figure/021-ggplot2-geoms-geom_point-7} 
\begin{kframe}\begin{alltt}
\hlcom{## No test: }
\hlcom{# Varying alpha is useful for large datasets}
\hlstd{d} \hlkwb{<-} \hlkwd{ggplot}\hlstd{(diamonds,} \hlkwd{aes}\hlstd{(carat, price))}
\hlstd{d} \hlopt{+} \hlkwd{geom_point}\hlstd{(}\hlkwc{alpha} \hlstd{=} \hlnum{1}\hlopt{/}\hlnum{10}\hlstd{)}
\end{alltt}
\end{kframe}
\includegraphics[width=\maxwidth]{figure/021-ggplot2-geoms-geom_point-8} 
\begin{kframe}\begin{alltt}
\hlstd{d} \hlopt{+} \hlkwd{geom_point}\hlstd{(}\hlkwc{alpha} \hlstd{=} \hlnum{1}\hlopt{/}\hlnum{20}\hlstd{)}
\end{alltt}
\end{kframe}
\includegraphics[width=\maxwidth]{figure/021-ggplot2-geoms-geom_point-9} 
\begin{kframe}\begin{alltt}
\hlstd{d} \hlopt{+} \hlkwd{geom_point}\hlstd{(}\hlkwc{alpha} \hlstd{=} \hlnum{1}\hlopt{/}\hlnum{100}\hlstd{)}
\end{alltt}
\end{kframe}
\includegraphics[width=\maxwidth]{figure/021-ggplot2-geoms-geom_point-10} 
\begin{kframe}\begin{alltt}
\hlcom{## End(No test)}

\hlcom{# For shapes that have a border (like 21), you can colour the inside and}
\hlcom{# outside separately. Use the stroke aesthetic to modify the width of the}
\hlcom{# border}
\hlkwd{ggplot}\hlstd{(mtcars,} \hlkwd{aes}\hlstd{(wt, mpg))} \hlopt{+}
  \hlkwd{geom_point}\hlstd{(}\hlkwc{shape} \hlstd{=} \hlnum{21}\hlstd{,} \hlkwc{colour} \hlstd{=} \hlstr{"black"}\hlstd{,} \hlkwc{fill} \hlstd{=} \hlstr{"white"}\hlstd{,} \hlkwc{size} \hlstd{=} \hlnum{5}\hlstd{,} \hlkwc{stroke} \hlstd{=} \hlnum{5}\hlstd{)}
\end{alltt}
\end{kframe}
\includegraphics[width=\maxwidth]{figure/021-ggplot2-geoms-geom_point-11} 
\begin{kframe}\begin{alltt}
\hlcom{## No test: }
\hlcom{# You can create interesting shapes by layering multiple points of}
\hlcom{# different sizes}
\hlstd{p} \hlkwb{<-} \hlkwd{ggplot}\hlstd{(mtcars,} \hlkwd{aes}\hlstd{(mpg, wt,} \hlkwc{shape} \hlstd{=} \hlkwd{factor}\hlstd{(cyl)))}
\hlstd{p} \hlopt{+} \hlkwd{geom_point}\hlstd{(}\hlkwd{aes}\hlstd{(}\hlkwc{colour} \hlstd{=} \hlkwd{factor}\hlstd{(cyl)),} \hlkwc{size} \hlstd{=} \hlnum{4}\hlstd{)} \hlopt{+}
  \hlkwd{geom_point}\hlstd{(}\hlkwc{colour} \hlstd{=} \hlstr{"grey90"}\hlstd{,} \hlkwc{size} \hlstd{=} \hlnum{1.5}\hlstd{)}
\end{alltt}
\end{kframe}
\includegraphics[width=\maxwidth]{figure/021-ggplot2-geoms-geom_point-12} 
\begin{kframe}\begin{alltt}
\hlstd{p} \hlopt{+} \hlkwd{geom_point}\hlstd{(}\hlkwc{colour} \hlstd{=} \hlstr{"black"}\hlstd{,} \hlkwc{size} \hlstd{=} \hlnum{4.5}\hlstd{)} \hlopt{+}
  \hlkwd{geom_point}\hlstd{(}\hlkwc{colour} \hlstd{=} \hlstr{"pink"}\hlstd{,} \hlkwc{size} \hlstd{=} \hlnum{4}\hlstd{)} \hlopt{+}
  \hlkwd{geom_point}\hlstd{(}\hlkwd{aes}\hlstd{(}\hlkwc{shape} \hlstd{=} \hlkwd{factor}\hlstd{(cyl)))}
\end{alltt}
\end{kframe}
\includegraphics[width=\maxwidth]{figure/021-ggplot2-geoms-geom_point-13} 
\begin{kframe}\begin{alltt}
\hlcom{# These extra layers don't usually appear in the legend, but we can}
\hlcom{# force their inclusion}
\hlstd{p} \hlopt{+} \hlkwd{geom_point}\hlstd{(}\hlkwc{colour} \hlstd{=} \hlstr{"black"}\hlstd{,} \hlkwc{size} \hlstd{=} \hlnum{4.5}\hlstd{,} \hlkwc{show.legend} \hlstd{=} \hlnum{TRUE}\hlstd{)} \hlopt{+}
  \hlkwd{geom_point}\hlstd{(}\hlkwc{colour} \hlstd{=} \hlstr{"pink"}\hlstd{,} \hlkwc{size} \hlstd{=} \hlnum{4}\hlstd{,} \hlkwc{show.legend} \hlstd{=} \hlnum{TRUE}\hlstd{)} \hlopt{+}
  \hlkwd{geom_point}\hlstd{(}\hlkwd{aes}\hlstd{(}\hlkwc{shape} \hlstd{=} \hlkwd{factor}\hlstd{(cyl)))}
\end{alltt}
\end{kframe}
\includegraphics[width=\maxwidth]{figure/021-ggplot2-geoms-geom_point-14} 
\begin{kframe}\begin{alltt}
\hlcom{# geom_point warns when missing values have been dropped from the data set}
\hlcom{# and not plotted, you can turn this off by setting na.rm = TRUE}
\hlstd{mtcars2} \hlkwb{<-} \hlkwd{transform}\hlstd{(mtcars,} \hlkwc{mpg} \hlstd{=} \hlkwd{ifelse}\hlstd{(}\hlkwd{runif}\hlstd{(}\hlnum{32}\hlstd{)} \hlopt{<} \hlnum{0.2}\hlstd{,} \hlnum{NA}\hlstd{, mpg))}
\hlkwd{ggplot}\hlstd{(mtcars2,} \hlkwd{aes}\hlstd{(wt, mpg))} \hlopt{+} \hlkwd{geom_point}\hlstd{()}
\end{alltt}


{\ttfamily\noindent\color{warningcolor}{\#\# Warning: Removed 6 rows containing missing values (geom\_point).}}\end{kframe}
\includegraphics[width=\maxwidth]{figure/021-ggplot2-geoms-geom_point-15} 
\begin{kframe}\begin{alltt}
\hlkwd{ggplot}\hlstd{(mtcars2,} \hlkwd{aes}\hlstd{(wt, mpg))} \hlopt{+} \hlkwd{geom_point}\hlstd{(}\hlkwc{na.rm} \hlstd{=} \hlnum{TRUE}\hlstd{)}
\end{alltt}
\end{kframe}
\includegraphics[width=\maxwidth]{figure/021-ggplot2-geoms-geom_point-16} 
\begin{kframe}\begin{alltt}
\hlcom{## End(No test)}
\end{alltt}
\end{kframe}
\end{knitrout}


\section{geom\_pointrange}

\begin{knitrout}
\definecolor{shadecolor}{rgb}{0.969, 0.969, 0.969}\color{fgcolor}\begin{kframe}
\begin{alltt}
\hlcom{### Name: geom_crossbar}
\hlcom{### Title: Vertical intervals: lines, crossbars & errorbars.}
\hlcom{### Aliases: geom_crossbar geom_errorbar geom_linerange geom_pointrange}

\hlcom{### ** Examples}

\hlcom{#' # Create a simple example dataset}
\hlstd{df} \hlkwb{<-} \hlkwd{data.frame}\hlstd{(}
  \hlkwc{trt} \hlstd{=} \hlkwd{factor}\hlstd{(}\hlkwd{c}\hlstd{(}\hlnum{1}\hlstd{,} \hlnum{1}\hlstd{,} \hlnum{2}\hlstd{,} \hlnum{2}\hlstd{)),}
  \hlkwc{resp} \hlstd{=} \hlkwd{c}\hlstd{(}\hlnum{1}\hlstd{,} \hlnum{5}\hlstd{,} \hlnum{3}\hlstd{,} \hlnum{4}\hlstd{),}
  \hlkwc{group} \hlstd{=} \hlkwd{factor}\hlstd{(}\hlkwd{c}\hlstd{(}\hlnum{1}\hlstd{,} \hlnum{2}\hlstd{,} \hlnum{1}\hlstd{,} \hlnum{2}\hlstd{)),}
  \hlkwc{upper} \hlstd{=} \hlkwd{c}\hlstd{(}\hlnum{1.1}\hlstd{,} \hlnum{5.3}\hlstd{,} \hlnum{3.3}\hlstd{,} \hlnum{4.2}\hlstd{),}
  \hlkwc{lower} \hlstd{=} \hlkwd{c}\hlstd{(}\hlnum{0.8}\hlstd{,} \hlnum{4.6}\hlstd{,} \hlnum{2.4}\hlstd{,} \hlnum{3.6}\hlstd{)}
\hlstd{)}

\hlstd{p} \hlkwb{<-} \hlkwd{ggplot}\hlstd{(df,} \hlkwd{aes}\hlstd{(trt, resp,} \hlkwc{colour} \hlstd{= group))}
\hlstd{p} \hlopt{+} \hlkwd{geom_linerange}\hlstd{(}\hlkwd{aes}\hlstd{(}\hlkwc{ymin} \hlstd{= lower,} \hlkwc{ymax} \hlstd{= upper))}
\end{alltt}
\end{kframe}
\includegraphics[width=\maxwidth]{figure/021-ggplot2-geoms-geom_pointrange-1} 
\begin{kframe}\begin{alltt}
\hlstd{p} \hlopt{+} \hlkwd{geom_pointrange}\hlstd{(}\hlkwd{aes}\hlstd{(}\hlkwc{ymin} \hlstd{= lower,} \hlkwc{ymax} \hlstd{= upper))}
\end{alltt}
\end{kframe}
\includegraphics[width=\maxwidth]{figure/021-ggplot2-geoms-geom_pointrange-2} 
\begin{kframe}\begin{alltt}
\hlstd{p} \hlopt{+} \hlkwd{geom_crossbar}\hlstd{(}\hlkwd{aes}\hlstd{(}\hlkwc{ymin} \hlstd{= lower,} \hlkwc{ymax} \hlstd{= upper),} \hlkwc{width} \hlstd{=} \hlnum{0.2}\hlstd{)}
\end{alltt}
\end{kframe}
\includegraphics[width=\maxwidth]{figure/021-ggplot2-geoms-geom_pointrange-3} 
\begin{kframe}\begin{alltt}
\hlstd{p} \hlopt{+} \hlkwd{geom_errorbar}\hlstd{(}\hlkwd{aes}\hlstd{(}\hlkwc{ymin} \hlstd{= lower,} \hlkwc{ymax} \hlstd{= upper),} \hlkwc{width} \hlstd{=} \hlnum{0.2}\hlstd{)}
\end{alltt}
\end{kframe}
\includegraphics[width=\maxwidth]{figure/021-ggplot2-geoms-geom_pointrange-4} 
\begin{kframe}\begin{alltt}
\hlcom{# Draw lines connecting group means}
\hlstd{p} \hlopt{+}
  \hlkwd{geom_line}\hlstd{(}\hlkwd{aes}\hlstd{(}\hlkwc{group} \hlstd{= group))} \hlopt{+}
  \hlkwd{geom_errorbar}\hlstd{(}\hlkwd{aes}\hlstd{(}\hlkwc{ymin} \hlstd{= lower,} \hlkwc{ymax} \hlstd{= upper),} \hlkwc{width} \hlstd{=} \hlnum{0.2}\hlstd{)}
\end{alltt}
\end{kframe}
\includegraphics[width=\maxwidth]{figure/021-ggplot2-geoms-geom_pointrange-5} 
\begin{kframe}\begin{alltt}
\hlcom{# If you want to dodge bars and errorbars, you need to manually}
\hlcom{# specify the dodge width}
\hlstd{p} \hlkwb{<-} \hlkwd{ggplot}\hlstd{(df,} \hlkwd{aes}\hlstd{(trt, resp,} \hlkwc{fill} \hlstd{= group))}
\hlstd{p} \hlopt{+}
 \hlkwd{geom_bar}\hlstd{(}\hlkwc{position} \hlstd{=} \hlstr{"dodge"}\hlstd{,} \hlkwc{stat} \hlstd{=} \hlstr{"identity"}\hlstd{)} \hlopt{+}
 \hlkwd{geom_errorbar}\hlstd{(}\hlkwd{aes}\hlstd{(}\hlkwc{ymin} \hlstd{= lower,} \hlkwc{ymax} \hlstd{= upper),} \hlkwc{position} \hlstd{=} \hlstr{"dodge"}\hlstd{,} \hlkwc{width} \hlstd{=} \hlnum{0.25}\hlstd{)}
\end{alltt}
\end{kframe}
\includegraphics[width=\maxwidth]{figure/021-ggplot2-geoms-geom_pointrange-6} 
\begin{kframe}\begin{alltt}
\hlcom{# Because the bars and errorbars have different widths}
\hlcom{# we need to specify how wide the objects we are dodging are}
\hlstd{dodge} \hlkwb{<-} \hlkwd{position_dodge}\hlstd{(}\hlkwc{width}\hlstd{=}\hlnum{0.9}\hlstd{)}
\hlstd{p} \hlopt{+}
  \hlkwd{geom_bar}\hlstd{(}\hlkwc{position} \hlstd{= dodge,} \hlkwc{stat} \hlstd{=} \hlstr{"identity"}\hlstd{)} \hlopt{+}
  \hlkwd{geom_errorbar}\hlstd{(}\hlkwd{aes}\hlstd{(}\hlkwc{ymin} \hlstd{= lower,} \hlkwc{ymax} \hlstd{= upper),} \hlkwc{position} \hlstd{= dodge,} \hlkwc{width} \hlstd{=} \hlnum{0.25}\hlstd{)}
\end{alltt}
\end{kframe}
\includegraphics[width=\maxwidth]{figure/021-ggplot2-geoms-geom_pointrange-7} 

\end{knitrout}


\section{geom\_polygon}

\begin{knitrout}
\definecolor{shadecolor}{rgb}{0.969, 0.969, 0.969}\color{fgcolor}\begin{kframe}
\begin{alltt}
\hlcom{### Name: geom_polygon}
\hlcom{### Title: Polygon, a filled path.}
\hlcom{### Aliases: geom_polygon}

\hlcom{### ** Examples}

\hlcom{# When using geom_polygon, you will typically need two data frames:}
\hlcom{# one contains the coordinates of each polygon (positions),  and the}
\hlcom{# other the values associated with each polygon (values).  An id}
\hlcom{# variable links the two together}

\hlstd{ids} \hlkwb{<-} \hlkwd{factor}\hlstd{(}\hlkwd{c}\hlstd{(}\hlstr{"1.1"}\hlstd{,} \hlstr{"2.1"}\hlstd{,} \hlstr{"1.2"}\hlstd{,} \hlstr{"2.2"}\hlstd{,} \hlstr{"1.3"}\hlstd{,} \hlstr{"2.3"}\hlstd{))}

\hlstd{values} \hlkwb{<-} \hlkwd{data.frame}\hlstd{(}
  \hlkwc{id} \hlstd{= ids,}
  \hlkwc{value} \hlstd{=} \hlkwd{c}\hlstd{(}\hlnum{3}\hlstd{,} \hlnum{3.1}\hlstd{,} \hlnum{3.1}\hlstd{,} \hlnum{3.2}\hlstd{,} \hlnum{3.15}\hlstd{,} \hlnum{3.5}\hlstd{)}
\hlstd{)}

\hlstd{positions} \hlkwb{<-} \hlkwd{data.frame}\hlstd{(}
  \hlkwc{id} \hlstd{=} \hlkwd{rep}\hlstd{(ids,} \hlkwc{each} \hlstd{=} \hlnum{4}\hlstd{),}
  \hlkwc{x} \hlstd{=} \hlkwd{c}\hlstd{(}\hlnum{2}\hlstd{,} \hlnum{1}\hlstd{,} \hlnum{1.1}\hlstd{,} \hlnum{2.2}\hlstd{,} \hlnum{1}\hlstd{,} \hlnum{0}\hlstd{,} \hlnum{0.3}\hlstd{,} \hlnum{1.1}\hlstd{,} \hlnum{2.2}\hlstd{,} \hlnum{1.1}\hlstd{,} \hlnum{1.2}\hlstd{,} \hlnum{2.5}\hlstd{,} \hlnum{1.1}\hlstd{,} \hlnum{0.3}\hlstd{,}
  \hlnum{0.5}\hlstd{,} \hlnum{1.2}\hlstd{,} \hlnum{2.5}\hlstd{,} \hlnum{1.2}\hlstd{,} \hlnum{1.3}\hlstd{,} \hlnum{2.7}\hlstd{,} \hlnum{1.2}\hlstd{,} \hlnum{0.5}\hlstd{,} \hlnum{0.6}\hlstd{,} \hlnum{1.3}\hlstd{),}
  \hlkwc{y} \hlstd{=} \hlkwd{c}\hlstd{(}\hlopt{-}\hlnum{0.5}\hlstd{,} \hlnum{0}\hlstd{,} \hlnum{1}\hlstd{,} \hlnum{0.5}\hlstd{,} \hlnum{0}\hlstd{,} \hlnum{0.5}\hlstd{,} \hlnum{1.5}\hlstd{,} \hlnum{1}\hlstd{,} \hlnum{0.5}\hlstd{,} \hlnum{1}\hlstd{,} \hlnum{2.1}\hlstd{,} \hlnum{1.7}\hlstd{,} \hlnum{1}\hlstd{,} \hlnum{1.5}\hlstd{,}
  \hlnum{2.2}\hlstd{,} \hlnum{2.1}\hlstd{,} \hlnum{1.7}\hlstd{,} \hlnum{2.1}\hlstd{,} \hlnum{3.2}\hlstd{,} \hlnum{2.8}\hlstd{,} \hlnum{2.1}\hlstd{,} \hlnum{2.2}\hlstd{,} \hlnum{3.3}\hlstd{,} \hlnum{3.2}\hlstd{)}
\hlstd{)}

\hlcom{# Currently we need to manually merge the two together}
\hlstd{datapoly} \hlkwb{<-} \hlkwd{merge}\hlstd{(values, positions,} \hlkwc{by}\hlstd{=}\hlkwd{c}\hlstd{(}\hlstr{"id"}\hlstd{))}

\hlstd{(p} \hlkwb{<-} \hlkwd{ggplot}\hlstd{(datapoly,} \hlkwd{aes}\hlstd{(}\hlkwc{x}\hlstd{=x,} \hlkwc{y}\hlstd{=y))} \hlopt{+} \hlkwd{geom_polygon}\hlstd{(}\hlkwd{aes}\hlstd{(}\hlkwc{fill}\hlstd{=value,} \hlkwc{group}\hlstd{=id)))}
\end{alltt}
\end{kframe}
\includegraphics[width=\maxwidth]{figure/021-ggplot2-geoms-geom_polygon-1} 
\begin{kframe}\begin{alltt}
\hlcom{# Which seems like a lot of work, but then it's easy to add on}
\hlcom{# other features in this coordinate system, e.g.:}

\hlstd{stream} \hlkwb{<-} \hlkwd{data.frame}\hlstd{(}
  \hlkwc{x} \hlstd{=} \hlkwd{cumsum}\hlstd{(}\hlkwd{runif}\hlstd{(}\hlnum{50}\hlstd{,} \hlkwc{max} \hlstd{=} \hlnum{0.1}\hlstd{)),}
  \hlkwc{y} \hlstd{=} \hlkwd{cumsum}\hlstd{(}\hlkwd{runif}\hlstd{(}\hlnum{50}\hlstd{,}\hlkwc{max} \hlstd{=} \hlnum{0.1}\hlstd{))}
\hlstd{)}

\hlstd{p} \hlopt{+} \hlkwd{geom_line}\hlstd{(}\hlkwc{data} \hlstd{= stream,} \hlkwc{colour}\hlstd{=}\hlstr{"grey30"}\hlstd{,} \hlkwc{size} \hlstd{=} \hlnum{5}\hlstd{)}
\end{alltt}
\end{kframe}
\includegraphics[width=\maxwidth]{figure/021-ggplot2-geoms-geom_polygon-2} 
\begin{kframe}\begin{alltt}
\hlcom{# And if the positions are in longitude and latitude, you can use}
\hlcom{# coord_map to produce different map projections.}
\end{alltt}
\end{kframe}
\end{knitrout}


\section{geom\_qq}

\begin{knitrout}
\definecolor{shadecolor}{rgb}{0.969, 0.969, 0.969}\color{fgcolor}\begin{kframe}
\begin{alltt}
\hlcom{### Name: stat_qq}
\hlcom{### Title: Calculation for quantile-quantile plot.}
\hlcom{### Aliases: geom_qq stat_qq}

\hlcom{### ** Examples}

\hlcom{## No test: }
\hlstd{df} \hlkwb{<-} \hlkwd{data.frame}\hlstd{(}\hlkwc{y} \hlstd{=} \hlkwd{rt}\hlstd{(}\hlnum{200}\hlstd{,} \hlkwc{df} \hlstd{=} \hlnum{5}\hlstd{))}
\hlstd{p} \hlkwb{<-} \hlkwd{ggplot}\hlstd{(df,} \hlkwd{aes}\hlstd{(}\hlkwc{sample} \hlstd{= y))}
\hlstd{p} \hlopt{+} \hlkwd{stat_qq}\hlstd{()}
\end{alltt}
\end{kframe}
\includegraphics[width=\maxwidth]{figure/021-ggplot2-geoms-geom_qq-1} 
\begin{kframe}\begin{alltt}
\hlstd{p} \hlopt{+} \hlkwd{geom_point}\hlstd{(}\hlkwc{stat} \hlstd{=} \hlstr{"qq"}\hlstd{)}
\end{alltt}
\end{kframe}
\includegraphics[width=\maxwidth]{figure/021-ggplot2-geoms-geom_qq-2} 
\begin{kframe}\begin{alltt}
\hlcom{# Use fitdistr from MASS to estimate distribution params}
\hlstd{params} \hlkwb{<-} \hlkwd{as.list}\hlstd{(MASS}\hlopt{::}\hlkwd{fitdistr}\hlstd{(df}\hlopt{$}\hlstd{y,} \hlstr{"t"}\hlstd{)}\hlopt{$}\hlstd{estimate)}
\end{alltt}


{\ttfamily\noindent\color{warningcolor}{\#\# Warning in log(s): NaNs produced}}

{\ttfamily\noindent\color{warningcolor}{\#\# Warning in log(s): NaNs produced}}\begin{alltt}
\hlkwd{ggplot}\hlstd{(df,} \hlkwd{aes}\hlstd{(}\hlkwc{sample} \hlstd{= y))} \hlopt{+}
  \hlkwd{stat_qq}\hlstd{(}\hlkwc{distribution} \hlstd{= qt,} \hlkwc{dparams} \hlstd{= params[}\hlstr{"df"}\hlstd{])}
\end{alltt}
\end{kframe}
\includegraphics[width=\maxwidth]{figure/021-ggplot2-geoms-geom_qq-3} 
\begin{kframe}\begin{alltt}
\hlcom{# Using to explore the distribution of a variable}
\hlkwd{ggplot}\hlstd{(mtcars)} \hlopt{+}
  \hlkwd{stat_qq}\hlstd{(}\hlkwd{aes}\hlstd{(}\hlkwc{sample} \hlstd{= mpg))}
\end{alltt}
\end{kframe}
\includegraphics[width=\maxwidth]{figure/021-ggplot2-geoms-geom_qq-4} 
\begin{kframe}\begin{alltt}
\hlkwd{ggplot}\hlstd{(mtcars)} \hlopt{+}
  \hlkwd{stat_qq}\hlstd{(}\hlkwd{aes}\hlstd{(}\hlkwc{sample} \hlstd{= mpg,} \hlkwc{colour} \hlstd{=} \hlkwd{factor}\hlstd{(cyl)))}
\end{alltt}
\end{kframe}
\includegraphics[width=\maxwidth]{figure/021-ggplot2-geoms-geom_qq-5} 
\begin{kframe}\begin{alltt}
\hlcom{## End(No test)}
\end{alltt}
\end{kframe}
\end{knitrout}


\section{geom\_quantile}

\begin{knitrout}
\definecolor{shadecolor}{rgb}{0.969, 0.969, 0.969}\color{fgcolor}\begin{kframe}
\begin{alltt}
\hlcom{### Name: geom_quantile}
\hlcom{### Title: Add quantile lines from a quantile regression.}
\hlcom{### Aliases: geom_quantile stat_quantile}

\hlcom{### ** Examples}

\hlstd{m} \hlkwb{<-} \hlkwd{ggplot}\hlstd{(mpg,} \hlkwd{aes}\hlstd{(displ,} \hlnum{1} \hlopt{/} \hlstd{hwy))} \hlopt{+} \hlkwd{geom_point}\hlstd{()}
\hlstd{m} \hlopt{+} \hlkwd{geom_quantile}\hlstd{()}
\end{alltt}


{\ttfamily\noindent\color{warningcolor}{\#\# Warning: Computation failed in `stat\_quantile()`:\\\#\# Package `quantreg` required for `stat\_quantile`.\\\#\# Please install and try again.}}\end{kframe}
\includegraphics[width=\maxwidth]{figure/021-ggplot2-geoms-geom_quantile-1} 
\begin{kframe}\begin{alltt}
\hlstd{m} \hlopt{+} \hlkwd{geom_quantile}\hlstd{(}\hlkwc{quantiles} \hlstd{=} \hlnum{0.5}\hlstd{)}
\end{alltt}


{\ttfamily\noindent\color{warningcolor}{\#\# Warning: Computation failed in `stat\_quantile()`:\\\#\# Package `quantreg` required for `stat\_quantile`.\\\#\# Please install and try again.}}\end{kframe}
\includegraphics[width=\maxwidth]{figure/021-ggplot2-geoms-geom_quantile-2} 
\begin{kframe}\begin{alltt}
\hlstd{q10} \hlkwb{<-} \hlkwd{seq}\hlstd{(}\hlnum{0.05}\hlstd{,} \hlnum{0.95}\hlstd{,} \hlkwc{by} \hlstd{=} \hlnum{0.05}\hlstd{)}
\hlstd{m} \hlopt{+} \hlkwd{geom_quantile}\hlstd{(}\hlkwc{quantiles} \hlstd{= q10)}
\end{alltt}


{\ttfamily\noindent\color{warningcolor}{\#\# Warning: Computation failed in `stat\_quantile()`:\\\#\# Package `quantreg` required for `stat\_quantile`.\\\#\# Please install and try again.}}\end{kframe}
\includegraphics[width=\maxwidth]{figure/021-ggplot2-geoms-geom_quantile-3} 
\begin{kframe}\begin{alltt}
\hlcom{# You can also use rqss to fit smooth quantiles}
\hlstd{m} \hlopt{+} \hlkwd{geom_quantile}\hlstd{(}\hlkwc{method} \hlstd{=} \hlstr{"rqss"}\hlstd{)}
\end{alltt}


{\ttfamily\noindent\color{warningcolor}{\#\# Warning: Computation failed in `stat\_quantile()`:\\\#\# Package `quantreg` required for `stat\_quantile`.\\\#\# Please install and try again.}}\end{kframe}
\includegraphics[width=\maxwidth]{figure/021-ggplot2-geoms-geom_quantile-4} 
\begin{kframe}\begin{alltt}
\hlcom{# Note that rqss doesn't pick a smoothing constant automatically, so}
\hlcom{# you'll need to tweak lambda yourself}
\hlstd{m} \hlopt{+} \hlkwd{geom_quantile}\hlstd{(}\hlkwc{method} \hlstd{=} \hlstr{"rqss"}\hlstd{,} \hlkwc{lambda} \hlstd{=} \hlnum{0.1}\hlstd{)}
\end{alltt}


{\ttfamily\noindent\color{warningcolor}{\#\# Warning: Computation failed in `stat\_quantile()`:\\\#\# Package `quantreg` required for `stat\_quantile`.\\\#\# Please install and try again.}}\end{kframe}
\includegraphics[width=\maxwidth]{figure/021-ggplot2-geoms-geom_quantile-5} 
\begin{kframe}\begin{alltt}
\hlcom{# Set aesthetics to fixed value}
\hlstd{m} \hlopt{+} \hlkwd{geom_quantile}\hlstd{(}\hlkwc{colour} \hlstd{=} \hlstr{"red"}\hlstd{,} \hlkwc{size} \hlstd{=} \hlnum{2}\hlstd{,} \hlkwc{alpha} \hlstd{=} \hlnum{0.5}\hlstd{)}
\end{alltt}


{\ttfamily\noindent\color{warningcolor}{\#\# Warning: Computation failed in `stat\_quantile()`:\\\#\# Package `quantreg` required for `stat\_quantile`.\\\#\# Please install and try again.}}\end{kframe}
\includegraphics[width=\maxwidth]{figure/021-ggplot2-geoms-geom_quantile-6} 

\end{knitrout}


\section{geom\_raster}

\begin{knitrout}
\definecolor{shadecolor}{rgb}{0.969, 0.969, 0.969}\color{fgcolor}\begin{kframe}
\begin{alltt}
\hlcom{### Name: geom_raster}
\hlcom{### Title: Draw rectangles.}
\hlcom{### Aliases: geom_raster geom_rect geom_tile}

\hlcom{### ** Examples}

\hlcom{# The most common use for rectangles is to draw a surface. You always want}
\hlcom{# to use geom_raster here because it's so much faster, and produces}
\hlcom{# smaller output when saving to PDF}
\hlkwd{ggplot}\hlstd{(faithfuld,} \hlkwd{aes}\hlstd{(waiting, eruptions))} \hlopt{+}
 \hlkwd{geom_raster}\hlstd{(}\hlkwd{aes}\hlstd{(}\hlkwc{fill} \hlstd{= density))}
\end{alltt}
\end{kframe}
\includegraphics[width=\maxwidth]{figure/021-ggplot2-geoms-geom_raster-1} 
\begin{kframe}\begin{alltt}
\hlcom{# Interpolation smooths the surface & is most helpful when rendering images.}
\hlkwd{ggplot}\hlstd{(faithfuld,} \hlkwd{aes}\hlstd{(waiting, eruptions))} \hlopt{+}
 \hlkwd{geom_raster}\hlstd{(}\hlkwd{aes}\hlstd{(}\hlkwc{fill} \hlstd{= density),} \hlkwc{interpolate} \hlstd{=} \hlnum{TRUE}\hlstd{)}
\end{alltt}
\end{kframe}
\includegraphics[width=\maxwidth]{figure/021-ggplot2-geoms-geom_raster-2} 
\begin{kframe}\begin{alltt}
\hlcom{# If you want to draw arbitrary rectangles, use geom_tile() or geom_rect()}
\hlstd{df} \hlkwb{<-} \hlkwd{data.frame}\hlstd{(}
  \hlkwc{x} \hlstd{=} \hlkwd{rep}\hlstd{(}\hlkwd{c}\hlstd{(}\hlnum{2}\hlstd{,} \hlnum{5}\hlstd{,} \hlnum{7}\hlstd{,} \hlnum{9}\hlstd{,} \hlnum{12}\hlstd{),} \hlnum{2}\hlstd{),}
  \hlkwc{y} \hlstd{=} \hlkwd{rep}\hlstd{(}\hlkwd{c}\hlstd{(}\hlnum{1}\hlstd{,} \hlnum{2}\hlstd{),} \hlkwc{each} \hlstd{=} \hlnum{5}\hlstd{),}
  \hlkwc{z} \hlstd{=} \hlkwd{factor}\hlstd{(}\hlkwd{rep}\hlstd{(}\hlnum{1}\hlopt{:}\hlnum{5}\hlstd{,} \hlkwc{each} \hlstd{=} \hlnum{2}\hlstd{)),}
  \hlkwc{w} \hlstd{=} \hlkwd{rep}\hlstd{(}\hlkwd{diff}\hlstd{(}\hlkwd{c}\hlstd{(}\hlnum{0}\hlstd{,} \hlnum{4}\hlstd{,} \hlnum{6}\hlstd{,} \hlnum{8}\hlstd{,} \hlnum{10}\hlstd{,} \hlnum{14}\hlstd{)),} \hlnum{2}\hlstd{)}
\hlstd{)}
\hlkwd{ggplot}\hlstd{(df,} \hlkwd{aes}\hlstd{(x, y))} \hlopt{+}
  \hlkwd{geom_tile}\hlstd{(}\hlkwd{aes}\hlstd{(}\hlkwc{fill} \hlstd{= z))}
\end{alltt}
\end{kframe}
\includegraphics[width=\maxwidth]{figure/021-ggplot2-geoms-geom_raster-3} 
\begin{kframe}\begin{alltt}
\hlkwd{ggplot}\hlstd{(df,} \hlkwd{aes}\hlstd{(x, y))} \hlopt{+}
  \hlkwd{geom_tile}\hlstd{(}\hlkwd{aes}\hlstd{(}\hlkwc{fill} \hlstd{= z,} \hlkwc{width} \hlstd{= w),} \hlkwc{colour} \hlstd{=} \hlstr{"grey50"}\hlstd{)}
\end{alltt}
\end{kframe}
\includegraphics[width=\maxwidth]{figure/021-ggplot2-geoms-geom_raster-4} 
\begin{kframe}\begin{alltt}
\hlkwd{ggplot}\hlstd{(df,} \hlkwd{aes}\hlstd{(}\hlkwc{xmin} \hlstd{= x} \hlopt{-} \hlstd{w} \hlopt{/} \hlnum{2}\hlstd{,} \hlkwc{xmax} \hlstd{= x} \hlopt{+} \hlstd{w} \hlopt{/} \hlnum{2}\hlstd{,} \hlkwc{ymin} \hlstd{= y,} \hlkwc{ymax} \hlstd{= y} \hlopt{+} \hlnum{1}\hlstd{))} \hlopt{+}
  \hlkwd{geom_rect}\hlstd{(}\hlkwd{aes}\hlstd{(}\hlkwc{fill} \hlstd{= z,} \hlkwc{width} \hlstd{= w),} \hlkwc{colour} \hlstd{=} \hlstr{"grey50"}\hlstd{)}
\end{alltt}
\end{kframe}
\includegraphics[width=\maxwidth]{figure/021-ggplot2-geoms-geom_raster-5} 
\begin{kframe}\begin{alltt}
\hlcom{## No test: }
\hlcom{# Justification controls where the cells are anchored}
\hlstd{df} \hlkwb{<-} \hlkwd{expand.grid}\hlstd{(}\hlkwc{x} \hlstd{=} \hlnum{0}\hlopt{:}\hlnum{5}\hlstd{,} \hlkwc{y} \hlstd{=} \hlnum{0}\hlopt{:}\hlnum{5}\hlstd{)}
\hlstd{df}\hlopt{$}\hlstd{z} \hlkwb{<-} \hlkwd{runif}\hlstd{(}\hlkwd{nrow}\hlstd{(df))}
\hlcom{# default is compatible with geom_tile()}
\hlkwd{ggplot}\hlstd{(df,} \hlkwd{aes}\hlstd{(x, y,} \hlkwc{fill} \hlstd{= z))} \hlopt{+} \hlkwd{geom_raster}\hlstd{()}
\end{alltt}
\end{kframe}
\includegraphics[width=\maxwidth]{figure/021-ggplot2-geoms-geom_raster-6} 
\begin{kframe}\begin{alltt}
\hlcom{# zero padding}
\hlkwd{ggplot}\hlstd{(df,} \hlkwd{aes}\hlstd{(x, y,} \hlkwc{fill} \hlstd{= z))} \hlopt{+} \hlkwd{geom_raster}\hlstd{(}\hlkwc{hjust} \hlstd{=} \hlnum{0}\hlstd{,} \hlkwc{vjust} \hlstd{=} \hlnum{0}\hlstd{)}
\end{alltt}
\end{kframe}
\includegraphics[width=\maxwidth]{figure/021-ggplot2-geoms-geom_raster-7} 
\begin{kframe}\begin{alltt}
\hlcom{# Inspired by the image-density plots of Ken Knoblauch}
\hlstd{cars} \hlkwb{<-} \hlkwd{ggplot}\hlstd{(mtcars,} \hlkwd{aes}\hlstd{(mpg,} \hlkwd{factor}\hlstd{(cyl)))}
\hlstd{cars} \hlopt{+} \hlkwd{geom_point}\hlstd{()}
\end{alltt}
\end{kframe}
\includegraphics[width=\maxwidth]{figure/021-ggplot2-geoms-geom_raster-8} 
\begin{kframe}\begin{alltt}
\hlstd{cars} \hlopt{+} \hlkwd{stat_bin2d}\hlstd{(}\hlkwd{aes}\hlstd{(}\hlkwc{fill} \hlstd{= ..count..),} \hlkwc{binwidth} \hlstd{=} \hlkwd{c}\hlstd{(}\hlnum{3}\hlstd{,}\hlnum{1}\hlstd{))}
\end{alltt}
\end{kframe}
\includegraphics[width=\maxwidth]{figure/021-ggplot2-geoms-geom_raster-9} 
\begin{kframe}\begin{alltt}
\hlstd{cars} \hlopt{+} \hlkwd{stat_bin2d}\hlstd{(}\hlkwd{aes}\hlstd{(}\hlkwc{fill} \hlstd{= ..density..),} \hlkwc{binwidth} \hlstd{=} \hlkwd{c}\hlstd{(}\hlnum{3}\hlstd{,}\hlnum{1}\hlstd{))}
\end{alltt}
\end{kframe}
\includegraphics[width=\maxwidth]{figure/021-ggplot2-geoms-geom_raster-10} 
\begin{kframe}\begin{alltt}
\hlstd{cars} \hlopt{+} \hlkwd{stat_density}\hlstd{(}\hlkwd{aes}\hlstd{(}\hlkwc{fill} \hlstd{= ..density..),} \hlkwc{geom} \hlstd{=} \hlstr{"raster"}\hlstd{,} \hlkwc{position} \hlstd{=} \hlstr{"identity"}\hlstd{)}
\end{alltt}
\end{kframe}
\includegraphics[width=\maxwidth]{figure/021-ggplot2-geoms-geom_raster-11} 
\begin{kframe}\begin{alltt}
\hlstd{cars} \hlopt{+} \hlkwd{stat_density}\hlstd{(}\hlkwd{aes}\hlstd{(}\hlkwc{fill} \hlstd{= ..count..),} \hlkwc{geom} \hlstd{=} \hlstr{"raster"}\hlstd{,} \hlkwc{position} \hlstd{=} \hlstr{"identity"}\hlstd{)}
\end{alltt}
\end{kframe}
\includegraphics[width=\maxwidth]{figure/021-ggplot2-geoms-geom_raster-12} 
\begin{kframe}\begin{alltt}
\hlcom{## End(No test)}
\end{alltt}
\end{kframe}
\end{knitrout}


\section{geom\_rect}

\begin{knitrout}
\definecolor{shadecolor}{rgb}{0.969, 0.969, 0.969}\color{fgcolor}\begin{kframe}
\begin{alltt}
\hlcom{### Name: geom_raster}
\hlcom{### Title: Draw rectangles.}
\hlcom{### Aliases: geom_raster geom_rect geom_tile}

\hlcom{### ** Examples}

\hlcom{# The most common use for rectangles is to draw a surface. You always want}
\hlcom{# to use geom_raster here because it's so much faster, and produces}
\hlcom{# smaller output when saving to PDF}
\hlkwd{ggplot}\hlstd{(faithfuld,} \hlkwd{aes}\hlstd{(waiting, eruptions))} \hlopt{+}
 \hlkwd{geom_raster}\hlstd{(}\hlkwd{aes}\hlstd{(}\hlkwc{fill} \hlstd{= density))}
\end{alltt}
\end{kframe}
\includegraphics[width=\maxwidth]{figure/021-ggplot2-geoms-geom_rect-1} 
\begin{kframe}\begin{alltt}
\hlcom{# Interpolation smooths the surface & is most helpful when rendering images.}
\hlkwd{ggplot}\hlstd{(faithfuld,} \hlkwd{aes}\hlstd{(waiting, eruptions))} \hlopt{+}
 \hlkwd{geom_raster}\hlstd{(}\hlkwd{aes}\hlstd{(}\hlkwc{fill} \hlstd{= density),} \hlkwc{interpolate} \hlstd{=} \hlnum{TRUE}\hlstd{)}
\end{alltt}
\end{kframe}
\includegraphics[width=\maxwidth]{figure/021-ggplot2-geoms-geom_rect-2} 
\begin{kframe}\begin{alltt}
\hlcom{# If you want to draw arbitrary rectangles, use geom_tile() or geom_rect()}
\hlstd{df} \hlkwb{<-} \hlkwd{data.frame}\hlstd{(}
  \hlkwc{x} \hlstd{=} \hlkwd{rep}\hlstd{(}\hlkwd{c}\hlstd{(}\hlnum{2}\hlstd{,} \hlnum{5}\hlstd{,} \hlnum{7}\hlstd{,} \hlnum{9}\hlstd{,} \hlnum{12}\hlstd{),} \hlnum{2}\hlstd{),}
  \hlkwc{y} \hlstd{=} \hlkwd{rep}\hlstd{(}\hlkwd{c}\hlstd{(}\hlnum{1}\hlstd{,} \hlnum{2}\hlstd{),} \hlkwc{each} \hlstd{=} \hlnum{5}\hlstd{),}
  \hlkwc{z} \hlstd{=} \hlkwd{factor}\hlstd{(}\hlkwd{rep}\hlstd{(}\hlnum{1}\hlopt{:}\hlnum{5}\hlstd{,} \hlkwc{each} \hlstd{=} \hlnum{2}\hlstd{)),}
  \hlkwc{w} \hlstd{=} \hlkwd{rep}\hlstd{(}\hlkwd{diff}\hlstd{(}\hlkwd{c}\hlstd{(}\hlnum{0}\hlstd{,} \hlnum{4}\hlstd{,} \hlnum{6}\hlstd{,} \hlnum{8}\hlstd{,} \hlnum{10}\hlstd{,} \hlnum{14}\hlstd{)),} \hlnum{2}\hlstd{)}
\hlstd{)}
\hlkwd{ggplot}\hlstd{(df,} \hlkwd{aes}\hlstd{(x, y))} \hlopt{+}
  \hlkwd{geom_tile}\hlstd{(}\hlkwd{aes}\hlstd{(}\hlkwc{fill} \hlstd{= z))}
\end{alltt}
\end{kframe}
\includegraphics[width=\maxwidth]{figure/021-ggplot2-geoms-geom_rect-3} 
\begin{kframe}\begin{alltt}
\hlkwd{ggplot}\hlstd{(df,} \hlkwd{aes}\hlstd{(x, y))} \hlopt{+}
  \hlkwd{geom_tile}\hlstd{(}\hlkwd{aes}\hlstd{(}\hlkwc{fill} \hlstd{= z,} \hlkwc{width} \hlstd{= w),} \hlkwc{colour} \hlstd{=} \hlstr{"grey50"}\hlstd{)}
\end{alltt}
\end{kframe}
\includegraphics[width=\maxwidth]{figure/021-ggplot2-geoms-geom_rect-4} 
\begin{kframe}\begin{alltt}
\hlkwd{ggplot}\hlstd{(df,} \hlkwd{aes}\hlstd{(}\hlkwc{xmin} \hlstd{= x} \hlopt{-} \hlstd{w} \hlopt{/} \hlnum{2}\hlstd{,} \hlkwc{xmax} \hlstd{= x} \hlopt{+} \hlstd{w} \hlopt{/} \hlnum{2}\hlstd{,} \hlkwc{ymin} \hlstd{= y,} \hlkwc{ymax} \hlstd{= y} \hlopt{+} \hlnum{1}\hlstd{))} \hlopt{+}
  \hlkwd{geom_rect}\hlstd{(}\hlkwd{aes}\hlstd{(}\hlkwc{fill} \hlstd{= z,} \hlkwc{width} \hlstd{= w),} \hlkwc{colour} \hlstd{=} \hlstr{"grey50"}\hlstd{)}
\end{alltt}
\end{kframe}
\includegraphics[width=\maxwidth]{figure/021-ggplot2-geoms-geom_rect-5} 
\begin{kframe}\begin{alltt}
\hlcom{## No test: }
\hlcom{# Justification controls where the cells are anchored}
\hlstd{df} \hlkwb{<-} \hlkwd{expand.grid}\hlstd{(}\hlkwc{x} \hlstd{=} \hlnum{0}\hlopt{:}\hlnum{5}\hlstd{,} \hlkwc{y} \hlstd{=} \hlnum{0}\hlopt{:}\hlnum{5}\hlstd{)}
\hlstd{df}\hlopt{$}\hlstd{z} \hlkwb{<-} \hlkwd{runif}\hlstd{(}\hlkwd{nrow}\hlstd{(df))}
\hlcom{# default is compatible with geom_tile()}
\hlkwd{ggplot}\hlstd{(df,} \hlkwd{aes}\hlstd{(x, y,} \hlkwc{fill} \hlstd{= z))} \hlopt{+} \hlkwd{geom_raster}\hlstd{()}
\end{alltt}
\end{kframe}
\includegraphics[width=\maxwidth]{figure/021-ggplot2-geoms-geom_rect-6} 
\begin{kframe}\begin{alltt}
\hlcom{# zero padding}
\hlkwd{ggplot}\hlstd{(df,} \hlkwd{aes}\hlstd{(x, y,} \hlkwc{fill} \hlstd{= z))} \hlopt{+} \hlkwd{geom_raster}\hlstd{(}\hlkwc{hjust} \hlstd{=} \hlnum{0}\hlstd{,} \hlkwc{vjust} \hlstd{=} \hlnum{0}\hlstd{)}
\end{alltt}
\end{kframe}
\includegraphics[width=\maxwidth]{figure/021-ggplot2-geoms-geom_rect-7} 
\begin{kframe}\begin{alltt}
\hlcom{# Inspired by the image-density plots of Ken Knoblauch}
\hlstd{cars} \hlkwb{<-} \hlkwd{ggplot}\hlstd{(mtcars,} \hlkwd{aes}\hlstd{(mpg,} \hlkwd{factor}\hlstd{(cyl)))}
\hlstd{cars} \hlopt{+} \hlkwd{geom_point}\hlstd{()}
\end{alltt}
\end{kframe}
\includegraphics[width=\maxwidth]{figure/021-ggplot2-geoms-geom_rect-8} 
\begin{kframe}\begin{alltt}
\hlstd{cars} \hlopt{+} \hlkwd{stat_bin2d}\hlstd{(}\hlkwd{aes}\hlstd{(}\hlkwc{fill} \hlstd{= ..count..),} \hlkwc{binwidth} \hlstd{=} \hlkwd{c}\hlstd{(}\hlnum{3}\hlstd{,}\hlnum{1}\hlstd{))}
\end{alltt}
\end{kframe}
\includegraphics[width=\maxwidth]{figure/021-ggplot2-geoms-geom_rect-9} 
\begin{kframe}\begin{alltt}
\hlstd{cars} \hlopt{+} \hlkwd{stat_bin2d}\hlstd{(}\hlkwd{aes}\hlstd{(}\hlkwc{fill} \hlstd{= ..density..),} \hlkwc{binwidth} \hlstd{=} \hlkwd{c}\hlstd{(}\hlnum{3}\hlstd{,}\hlnum{1}\hlstd{))}
\end{alltt}
\end{kframe}
\includegraphics[width=\maxwidth]{figure/021-ggplot2-geoms-geom_rect-10} 
\begin{kframe}\begin{alltt}
\hlstd{cars} \hlopt{+} \hlkwd{stat_density}\hlstd{(}\hlkwd{aes}\hlstd{(}\hlkwc{fill} \hlstd{= ..density..),} \hlkwc{geom} \hlstd{=} \hlstr{"raster"}\hlstd{,} \hlkwc{position} \hlstd{=} \hlstr{"identity"}\hlstd{)}
\end{alltt}
\end{kframe}
\includegraphics[width=\maxwidth]{figure/021-ggplot2-geoms-geom_rect-11} 
\begin{kframe}\begin{alltt}
\hlstd{cars} \hlopt{+} \hlkwd{stat_density}\hlstd{(}\hlkwd{aes}\hlstd{(}\hlkwc{fill} \hlstd{= ..count..),} \hlkwc{geom} \hlstd{=} \hlstr{"raster"}\hlstd{,} \hlkwc{position} \hlstd{=} \hlstr{"identity"}\hlstd{)}
\end{alltt}
\end{kframe}
\includegraphics[width=\maxwidth]{figure/021-ggplot2-geoms-geom_rect-12} 
\begin{kframe}\begin{alltt}
\hlcom{## End(No test)}
\end{alltt}
\end{kframe}
\end{knitrout}


\section{geom\_ribbon}

\begin{knitrout}
\definecolor{shadecolor}{rgb}{0.969, 0.969, 0.969}\color{fgcolor}\begin{kframe}
\begin{alltt}
\hlcom{### Name: geom_ribbon}
\hlcom{### Title: Ribbons and area plots.}
\hlcom{### Aliases: geom_area geom_ribbon}

\hlcom{### ** Examples}

\hlcom{# Generate data}
\hlstd{huron} \hlkwb{<-} \hlkwd{data.frame}\hlstd{(}\hlkwc{year} \hlstd{=} \hlnum{1875}\hlopt{:}\hlnum{1972}\hlstd{,} \hlkwc{level} \hlstd{=} \hlkwd{as.vector}\hlstd{(LakeHuron))}
\hlstd{h} \hlkwb{<-} \hlkwd{ggplot}\hlstd{(huron,} \hlkwd{aes}\hlstd{(year))}

\hlstd{h} \hlopt{+} \hlkwd{geom_ribbon}\hlstd{(}\hlkwd{aes}\hlstd{(}\hlkwc{ymin}\hlstd{=}\hlnum{0}\hlstd{,} \hlkwc{ymax}\hlstd{=level))}
\end{alltt}
\end{kframe}
\includegraphics[width=\maxwidth]{figure/021-ggplot2-geoms-geom_ribbon-1} 
\begin{kframe}\begin{alltt}
\hlstd{h} \hlopt{+} \hlkwd{geom_area}\hlstd{(}\hlkwd{aes}\hlstd{(}\hlkwc{y} \hlstd{= level))}
\end{alltt}
\end{kframe}
\includegraphics[width=\maxwidth]{figure/021-ggplot2-geoms-geom_ribbon-2} 
\begin{kframe}\begin{alltt}
\hlcom{# Add aesthetic mappings}
\hlstd{h} \hlopt{+}
  \hlkwd{geom_ribbon}\hlstd{(}\hlkwd{aes}\hlstd{(}\hlkwc{ymin} \hlstd{= level} \hlopt{-} \hlnum{1}\hlstd{,} \hlkwc{ymax} \hlstd{= level} \hlopt{+} \hlnum{1}\hlstd{),} \hlkwc{fill} \hlstd{=} \hlstr{"grey70"}\hlstd{)} \hlopt{+}
  \hlkwd{geom_line}\hlstd{(}\hlkwd{aes}\hlstd{(}\hlkwc{y} \hlstd{= level))}
\end{alltt}
\end{kframe}
\includegraphics[width=\maxwidth]{figure/021-ggplot2-geoms-geom_ribbon-3} 

\end{knitrout}


\section{geom\_rug}

\begin{knitrout}
\definecolor{shadecolor}{rgb}{0.969, 0.969, 0.969}\color{fgcolor}\begin{kframe}
\begin{alltt}
\hlcom{### Name: geom_rug}
\hlcom{### Title: Marginal rug plots.}
\hlcom{### Aliases: geom_rug}

\hlcom{### ** Examples}

\hlstd{p} \hlkwb{<-} \hlkwd{ggplot}\hlstd{(mtcars,} \hlkwd{aes}\hlstd{(wt, mpg))}
\hlstd{p} \hlopt{+} \hlkwd{geom_point}\hlstd{()}
\end{alltt}
\end{kframe}
\includegraphics[width=\maxwidth]{figure/021-ggplot2-geoms-geom_rug-1} 
\begin{kframe}\begin{alltt}
\hlstd{p} \hlopt{+} \hlkwd{geom_point}\hlstd{()} \hlopt{+} \hlkwd{geom_rug}\hlstd{()}
\end{alltt}
\end{kframe}
\includegraphics[width=\maxwidth]{figure/021-ggplot2-geoms-geom_rug-2} 
\begin{kframe}\begin{alltt}
\hlstd{p} \hlopt{+} \hlkwd{geom_point}\hlstd{()} \hlopt{+} \hlkwd{geom_rug}\hlstd{(}\hlkwc{sides}\hlstd{=}\hlstr{"b"}\hlstd{)}    \hlcom{# Rug on bottom only}
\end{alltt}
\end{kframe}
\includegraphics[width=\maxwidth]{figure/021-ggplot2-geoms-geom_rug-3} 
\begin{kframe}\begin{alltt}
\hlstd{p} \hlopt{+} \hlkwd{geom_point}\hlstd{()} \hlopt{+} \hlkwd{geom_rug}\hlstd{(}\hlkwc{sides}\hlstd{=}\hlstr{"trbl"}\hlstd{)} \hlcom{# All four sides}
\end{alltt}
\end{kframe}
\includegraphics[width=\maxwidth]{figure/021-ggplot2-geoms-geom_rug-4} 
\begin{kframe}\begin{alltt}
\hlstd{p} \hlopt{+} \hlkwd{geom_point}\hlstd{()} \hlopt{+} \hlkwd{geom_rug}\hlstd{(}\hlkwc{position}\hlstd{=}\hlstr{'jitter'}\hlstd{)}
\end{alltt}
\end{kframe}
\includegraphics[width=\maxwidth]{figure/021-ggplot2-geoms-geom_rug-5} 

\end{knitrout}


\section{geom\_segment}

\begin{knitrout}
\definecolor{shadecolor}{rgb}{0.969, 0.969, 0.969}\color{fgcolor}\begin{kframe}
\begin{alltt}
\hlcom{### Name: geom_segment}
\hlcom{### Title: Line segments and curves.}
\hlcom{### Aliases: geom_curve geom_segment}

\hlcom{### ** Examples}

\hlstd{b} \hlkwb{<-} \hlkwd{ggplot}\hlstd{(mtcars,} \hlkwd{aes}\hlstd{(wt, mpg))} \hlopt{+}
  \hlkwd{geom_point}\hlstd{()}

\hlstd{df} \hlkwb{<-} \hlkwd{data.frame}\hlstd{(}\hlkwc{x1} \hlstd{=} \hlnum{2.62}\hlstd{,} \hlkwc{x2} \hlstd{=} \hlnum{3.57}\hlstd{,} \hlkwc{y1} \hlstd{=} \hlnum{21.0}\hlstd{,} \hlkwc{y2} \hlstd{=} \hlnum{15.0}\hlstd{)}
\hlstd{b} \hlopt{+}
 \hlkwd{geom_curve}\hlstd{(}\hlkwd{aes}\hlstd{(}\hlkwc{x} \hlstd{= x1,} \hlkwc{y} \hlstd{= y1,} \hlkwc{xend} \hlstd{= x2,} \hlkwc{yend} \hlstd{= y2,} \hlkwc{colour} \hlstd{=} \hlstr{"curve"}\hlstd{),} \hlkwc{data} \hlstd{= df)} \hlopt{+}
 \hlkwd{geom_segment}\hlstd{(}\hlkwd{aes}\hlstd{(}\hlkwc{x} \hlstd{= x1,} \hlkwc{y} \hlstd{= y1,} \hlkwc{xend} \hlstd{= x2,} \hlkwc{yend} \hlstd{= y2,} \hlkwc{colour} \hlstd{=} \hlstr{"segment"}\hlstd{),} \hlkwc{data} \hlstd{= df)}
\end{alltt}
\end{kframe}
\includegraphics[width=\maxwidth]{figure/021-ggplot2-geoms-geom_segment-1} 
\begin{kframe}\begin{alltt}
\hlstd{b} \hlopt{+} \hlkwd{geom_curve}\hlstd{(}\hlkwd{aes}\hlstd{(}\hlkwc{x} \hlstd{= x1,} \hlkwc{y} \hlstd{= y1,} \hlkwc{xend} \hlstd{= x2,} \hlkwc{yend} \hlstd{= y2),} \hlkwc{data} \hlstd{= df,} \hlkwc{curvature} \hlstd{=} \hlopt{-}\hlnum{0.2}\hlstd{)}
\end{alltt}
\end{kframe}
\includegraphics[width=\maxwidth]{figure/021-ggplot2-geoms-geom_segment-2} 
\begin{kframe}\begin{alltt}
\hlstd{b} \hlopt{+} \hlkwd{geom_curve}\hlstd{(}\hlkwd{aes}\hlstd{(}\hlkwc{x} \hlstd{= x1,} \hlkwc{y} \hlstd{= y1,} \hlkwc{xend} \hlstd{= x2,} \hlkwc{yend} \hlstd{= y2),} \hlkwc{data} \hlstd{= df,} \hlkwc{curvature} \hlstd{=} \hlnum{1}\hlstd{)}
\end{alltt}
\end{kframe}
\includegraphics[width=\maxwidth]{figure/021-ggplot2-geoms-geom_segment-3} 
\begin{kframe}\begin{alltt}
\hlstd{b} \hlopt{+} \hlkwd{geom_curve}\hlstd{(}
  \hlkwd{aes}\hlstd{(}\hlkwc{x} \hlstd{= x1,} \hlkwc{y} \hlstd{= y1,} \hlkwc{xend} \hlstd{= x2,} \hlkwc{yend} \hlstd{= y2),}
  \hlkwc{data} \hlstd{= df,}
  \hlkwc{arrow} \hlstd{=} \hlkwd{arrow}\hlstd{(}\hlkwc{length} \hlstd{=} \hlkwd{unit}\hlstd{(}\hlnum{0.03}\hlstd{,} \hlstr{"npc"}\hlstd{))}
\hlstd{)}
\end{alltt}
\end{kframe}
\includegraphics[width=\maxwidth]{figure/021-ggplot2-geoms-geom_segment-4} 
\begin{kframe}\begin{alltt}
\hlkwd{ggplot}\hlstd{(seals,} \hlkwd{aes}\hlstd{(long, lat))} \hlopt{+}
  \hlkwd{geom_segment}\hlstd{(}\hlkwd{aes}\hlstd{(}\hlkwc{xend} \hlstd{= long} \hlopt{+} \hlstd{delta_long,} \hlkwc{yend} \hlstd{= lat} \hlopt{+} \hlstd{delta_lat),}
    \hlkwc{arrow} \hlstd{=} \hlkwd{arrow}\hlstd{(}\hlkwc{length} \hlstd{=} \hlkwd{unit}\hlstd{(}\hlnum{0.1}\hlstd{,}\hlstr{"cm"}\hlstd{)))} \hlopt{+}
  \hlkwd{borders}\hlstd{(}\hlstr{"state"}\hlstd{)}
\end{alltt}
\end{kframe}
\includegraphics[width=\maxwidth]{figure/021-ggplot2-geoms-geom_segment-5} 
\begin{kframe}\begin{alltt}
\hlcom{# You can also use geom_segment to recreate plot(type = "h") :}
\hlstd{counts} \hlkwb{<-} \hlkwd{as.data.frame}\hlstd{(}\hlkwd{table}\hlstd{(}\hlkwc{x} \hlstd{=} \hlkwd{rpois}\hlstd{(}\hlnum{100}\hlstd{,}\hlnum{5}\hlstd{)))}
\hlstd{counts}\hlopt{$}\hlstd{x} \hlkwb{<-} \hlkwd{as.numeric}\hlstd{(}\hlkwd{as.character}\hlstd{(counts}\hlopt{$}\hlstd{x))}
\hlkwd{with}\hlstd{(counts,} \hlkwd{plot}\hlstd{(x, Freq,} \hlkwc{type} \hlstd{=} \hlstr{"h"}\hlstd{,} \hlkwc{lwd} \hlstd{=} \hlnum{10}\hlstd{))}
\end{alltt}
\end{kframe}
\includegraphics[width=\maxwidth]{figure/021-ggplot2-geoms-geom_segment-6} 
\begin{kframe}\begin{alltt}
\hlkwd{ggplot}\hlstd{(counts,} \hlkwd{aes}\hlstd{(x, Freq))} \hlopt{+}
  \hlkwd{geom_segment}\hlstd{(}\hlkwd{aes}\hlstd{(}\hlkwc{xend} \hlstd{= x,} \hlkwc{yend} \hlstd{=} \hlnum{0}\hlstd{),} \hlkwc{size} \hlstd{=} \hlnum{10}\hlstd{,} \hlkwc{lineend} \hlstd{=} \hlstr{"butt"}\hlstd{)}
\end{alltt}
\end{kframe}
\includegraphics[width=\maxwidth]{figure/021-ggplot2-geoms-geom_segment-7} 

\end{knitrout}


\section{geom\_smooth}

\begin{knitrout}
\definecolor{shadecolor}{rgb}{0.969, 0.969, 0.969}\color{fgcolor}\begin{kframe}
\begin{alltt}
\hlcom{### Name: geom_smooth}
\hlcom{### Title: Add a smoothed conditional mean.}
\hlcom{### Aliases: geom_smooth stat_smooth}

\hlcom{### ** Examples}

\hlkwd{ggplot}\hlstd{(mpg,} \hlkwd{aes}\hlstd{(displ, hwy))} \hlopt{+}
  \hlkwd{geom_point}\hlstd{()} \hlopt{+}
  \hlkwd{geom_smooth}\hlstd{()}
\end{alltt}
\end{kframe}
\includegraphics[width=\maxwidth]{figure/021-ggplot2-geoms-geom_smooth-1} 
\begin{kframe}\begin{alltt}
\hlcom{# Use span to control the "wiggliness" of the default loess smoother}
\hlcom{# The span is the fraction of points used to fit each local regression:}
\hlcom{# small numbers make a wigglier curve, larger numbers make a smoother curve.}
\hlkwd{ggplot}\hlstd{(mpg,} \hlkwd{aes}\hlstd{(displ, hwy))} \hlopt{+}
  \hlkwd{geom_point}\hlstd{()} \hlopt{+}
  \hlkwd{geom_smooth}\hlstd{(}\hlkwc{span} \hlstd{=} \hlnum{0.3}\hlstd{)}
\end{alltt}
\end{kframe}
\includegraphics[width=\maxwidth]{figure/021-ggplot2-geoms-geom_smooth-2} 
\begin{kframe}\begin{alltt}
\hlcom{# Instead of a loess smooth, you can use any other modelling function:}
\hlkwd{ggplot}\hlstd{(mpg,} \hlkwd{aes}\hlstd{(displ, hwy))} \hlopt{+}
  \hlkwd{geom_point}\hlstd{()} \hlopt{+}
  \hlkwd{geom_smooth}\hlstd{(}\hlkwc{method} \hlstd{=} \hlstr{"lm"}\hlstd{,} \hlkwc{se} \hlstd{=} \hlnum{FALSE}\hlstd{)}
\end{alltt}
\end{kframe}
\includegraphics[width=\maxwidth]{figure/021-ggplot2-geoms-geom_smooth-3} 
\begin{kframe}\begin{alltt}
\hlkwd{ggplot}\hlstd{(mpg,} \hlkwd{aes}\hlstd{(displ, hwy))} \hlopt{+}
  \hlkwd{geom_point}\hlstd{()} \hlopt{+}
  \hlkwd{geom_smooth}\hlstd{(}\hlkwc{method} \hlstd{=} \hlstr{"lm"}\hlstd{,} \hlkwc{formula} \hlstd{= y} \hlopt{~} \hlstd{splines}\hlopt{::}\hlkwd{bs}\hlstd{(x,} \hlnum{3}\hlstd{),} \hlkwc{se} \hlstd{=} \hlnum{FALSE}\hlstd{)}
\end{alltt}
\end{kframe}
\includegraphics[width=\maxwidth]{figure/021-ggplot2-geoms-geom_smooth-4} 
\begin{kframe}\begin{alltt}
\hlcom{# Smoothes are automatically fit to each group (defined by categorical}
\hlcom{# aesthetics or the group aesthetic) and for each facet}

\hlkwd{ggplot}\hlstd{(mpg,} \hlkwd{aes}\hlstd{(displ, hwy,} \hlkwc{colour} \hlstd{= class))} \hlopt{+}
  \hlkwd{geom_point}\hlstd{()} \hlopt{+}
  \hlkwd{geom_smooth}\hlstd{(}\hlkwc{se} \hlstd{=} \hlnum{FALSE}\hlstd{,} \hlkwc{method} \hlstd{=} \hlstr{"lm"}\hlstd{)}
\end{alltt}
\end{kframe}
\includegraphics[width=\maxwidth]{figure/021-ggplot2-geoms-geom_smooth-5} 
\begin{kframe}\begin{alltt}
\hlkwd{ggplot}\hlstd{(mpg,} \hlkwd{aes}\hlstd{(displ, hwy))} \hlopt{+}
  \hlkwd{geom_point}\hlstd{()} \hlopt{+}
  \hlkwd{geom_smooth}\hlstd{(}\hlkwc{span} \hlstd{=} \hlnum{0.8}\hlstd{)} \hlopt{+}
  \hlkwd{facet_wrap}\hlstd{(}\hlopt{~}\hlstd{drv)}
\end{alltt}
\end{kframe}
\includegraphics[width=\maxwidth]{figure/021-ggplot2-geoms-geom_smooth-6} 
\begin{kframe}\begin{alltt}
\hlcom{## Not run: }
\hlcom{##D # To fit a logistic regression, you need to coerce the values to}
\hlcom{##D # a numeric vector lying between 0 and 1.}
\hlcom{##D binomial_smooth <- function(...) \{}
\hlcom{##D   geom_smooth(method = "glm", method.args = list(family = "binomial"), ...)}
\hlcom{##D \}}
\hlcom{##D }
\hlcom{##D ggplot(rpart::kyphosis, aes(Age, Kyphosis)) +}
\hlcom{##D   geom_jitter(height = 0.05) +}
\hlcom{##D   binomial_smooth()}
\hlcom{##D }
\hlcom{##D ggplot(rpart::kyphosis, aes(Age, as.numeric(Kyphosis) - 1)) +}
\hlcom{##D   geom_jitter(height = 0.05) +}
\hlcom{##D   binomial_smooth()}
\hlcom{##D }
\hlcom{##D ggplot(rpart::kyphosis, aes(Age, as.numeric(Kyphosis) - 1)) +}
\hlcom{##D   geom_jitter(height = 0.05) +}
\hlcom{##D   binomial_smooth(formula = y ~ splines::ns(x, 2))}
\hlcom{##D }
\hlcom{##D # But in this case, it's probably better to fit the model yourself}
\hlcom{##D # so you can exercise more control and see whether or not it's a good model}
\hlcom{## End(Not run)}
\end{alltt}
\end{kframe}
\end{knitrout}


\section{geom\_spoke}

\begin{knitrout}
\definecolor{shadecolor}{rgb}{0.969, 0.969, 0.969}\color{fgcolor}\begin{kframe}
\begin{alltt}
\hlcom{### Name: geom_spoke}
\hlcom{### Title: A line segment parameterised by location, direction and}
\hlcom{###   distance.}
\hlcom{### Aliases: geom_spoke stat_spoke}

\hlcom{### ** Examples}

\hlstd{df} \hlkwb{<-} \hlkwd{expand.grid}\hlstd{(}\hlkwc{x} \hlstd{=} \hlnum{1}\hlopt{:}\hlnum{10}\hlstd{,} \hlkwc{y}\hlstd{=}\hlnum{1}\hlopt{:}\hlnum{10}\hlstd{)}
\hlstd{df}\hlopt{$}\hlstd{angle} \hlkwb{<-} \hlkwd{runif}\hlstd{(}\hlnum{100}\hlstd{,} \hlnum{0}\hlstd{,} \hlnum{2}\hlopt{*}\hlstd{pi)}
\hlstd{df}\hlopt{$}\hlstd{speed} \hlkwb{<-} \hlkwd{runif}\hlstd{(}\hlnum{100}\hlstd{,} \hlnum{0}\hlstd{,} \hlkwd{sqrt}\hlstd{(}\hlnum{0.1} \hlopt{*} \hlstd{df}\hlopt{$}\hlstd{x))}

\hlkwd{ggplot}\hlstd{(df,} \hlkwd{aes}\hlstd{(x, y))} \hlopt{+}
  \hlkwd{geom_point}\hlstd{()} \hlopt{+}
  \hlkwd{geom_spoke}\hlstd{(}\hlkwd{aes}\hlstd{(}\hlkwc{angle} \hlstd{= angle),} \hlkwc{radius} \hlstd{=} \hlnum{0.5}\hlstd{)}
\end{alltt}
\end{kframe}
\includegraphics[width=\maxwidth]{figure/021-ggplot2-geoms-geom_spoke-1} 
\begin{kframe}\begin{alltt}
\hlkwd{ggplot}\hlstd{(df,} \hlkwd{aes}\hlstd{(x, y))} \hlopt{+}
  \hlkwd{geom_point}\hlstd{()} \hlopt{+}
  \hlkwd{geom_spoke}\hlstd{(}\hlkwd{aes}\hlstd{(}\hlkwc{angle} \hlstd{= angle,} \hlkwc{radius} \hlstd{= speed))}
\end{alltt}
\end{kframe}
\includegraphics[width=\maxwidth]{figure/021-ggplot2-geoms-geom_spoke-2} 

\end{knitrout}


\section{geom\_step}

\begin{knitrout}
\definecolor{shadecolor}{rgb}{0.969, 0.969, 0.969}\color{fgcolor}\begin{kframe}
\begin{alltt}
\hlcom{### Name: geom_path}
\hlcom{### Title: Connect observations.}
\hlcom{### Aliases: geom_line geom_path geom_step}

\hlcom{### ** Examples}

\hlcom{# geom_line() is suitable for time series}
\hlkwd{ggplot}\hlstd{(economics,} \hlkwd{aes}\hlstd{(date, unemploy))} \hlopt{+} \hlkwd{geom_line}\hlstd{()}
\end{alltt}
\end{kframe}
\includegraphics[width=\maxwidth]{figure/021-ggplot2-geoms-geom_step-1} 
\begin{kframe}\begin{alltt}
\hlkwd{ggplot}\hlstd{(economics_long,} \hlkwd{aes}\hlstd{(date, value01,} \hlkwc{colour} \hlstd{= variable))} \hlopt{+}
  \hlkwd{geom_line}\hlstd{()}
\end{alltt}
\end{kframe}
\includegraphics[width=\maxwidth]{figure/021-ggplot2-geoms-geom_step-2} 
\begin{kframe}\begin{alltt}
\hlcom{# geom_step() is useful when you want to highlight exactly when}
\hlcom{# the y value chanes}
\hlstd{recent} \hlkwb{<-} \hlstd{economics[economics}\hlopt{$}\hlstd{date} \hlopt{>} \hlkwd{as.Date}\hlstd{(}\hlstr{"2013-01-01"}\hlstd{), ]}
\hlkwd{ggplot}\hlstd{(recent,} \hlkwd{aes}\hlstd{(date, unemploy))} \hlopt{+} \hlkwd{geom_line}\hlstd{()}
\end{alltt}
\end{kframe}
\includegraphics[width=\maxwidth]{figure/021-ggplot2-geoms-geom_step-3} 
\begin{kframe}\begin{alltt}
\hlkwd{ggplot}\hlstd{(recent,} \hlkwd{aes}\hlstd{(date, unemploy))} \hlopt{+} \hlkwd{geom_step}\hlstd{()}
\end{alltt}
\end{kframe}
\includegraphics[width=\maxwidth]{figure/021-ggplot2-geoms-geom_step-4} 
\begin{kframe}\begin{alltt}
\hlcom{# geom_path lets you explore how two variables are related over time,}
\hlcom{# e.g. unemployment and personal savings rate}
\hlstd{m} \hlkwb{<-} \hlkwd{ggplot}\hlstd{(economics,} \hlkwd{aes}\hlstd{(unemploy}\hlopt{/}\hlstd{pop, psavert))}
\hlstd{m} \hlopt{+} \hlkwd{geom_path}\hlstd{()}
\end{alltt}
\end{kframe}
\includegraphics[width=\maxwidth]{figure/021-ggplot2-geoms-geom_step-5} 
\begin{kframe}\begin{alltt}
\hlstd{m} \hlopt{+} \hlkwd{geom_path}\hlstd{(}\hlkwd{aes}\hlstd{(}\hlkwc{colour} \hlstd{=} \hlkwd{as.numeric}\hlstd{(date)))}
\end{alltt}
\end{kframe}
\includegraphics[width=\maxwidth]{figure/021-ggplot2-geoms-geom_step-6} 
\begin{kframe}\begin{alltt}
\hlcom{# Changing parameters ----------------------------------------------}
\hlkwd{ggplot}\hlstd{(economics,} \hlkwd{aes}\hlstd{(date, unemploy))} \hlopt{+}
  \hlkwd{geom_line}\hlstd{(}\hlkwc{colour} \hlstd{=} \hlstr{"red"}\hlstd{)}
\end{alltt}
\end{kframe}
\includegraphics[width=\maxwidth]{figure/021-ggplot2-geoms-geom_step-7} 
\begin{kframe}\begin{alltt}
\hlcom{# Use the arrow parameter to add an arrow to the line}
\hlcom{# See ?arrow for more details}
\hlstd{c} \hlkwb{<-} \hlkwd{ggplot}\hlstd{(economics,} \hlkwd{aes}\hlstd{(}\hlkwc{x} \hlstd{= date,} \hlkwc{y} \hlstd{= pop))}
\hlstd{c} \hlopt{+} \hlkwd{geom_line}\hlstd{(}\hlkwc{arrow} \hlstd{=} \hlkwd{arrow}\hlstd{())}
\end{alltt}
\end{kframe}
\includegraphics[width=\maxwidth]{figure/021-ggplot2-geoms-geom_step-8} 
\begin{kframe}\begin{alltt}
\hlstd{c} \hlopt{+} \hlkwd{geom_line}\hlstd{(}
  \hlkwc{arrow} \hlstd{=} \hlkwd{arrow}\hlstd{(}\hlkwc{angle} \hlstd{=} \hlnum{15}\hlstd{,} \hlkwc{ends} \hlstd{=} \hlstr{"both"}\hlstd{,} \hlkwc{type} \hlstd{=} \hlstr{"closed"}\hlstd{)}
\hlstd{)}
\end{alltt}
\end{kframe}
\includegraphics[width=\maxwidth]{figure/021-ggplot2-geoms-geom_step-9} 
\begin{kframe}\begin{alltt}
\hlcom{# Control line join parameters}
\hlstd{df} \hlkwb{<-} \hlkwd{data.frame}\hlstd{(}\hlkwc{x} \hlstd{=} \hlnum{1}\hlopt{:}\hlnum{3}\hlstd{,} \hlkwc{y} \hlstd{=} \hlkwd{c}\hlstd{(}\hlnum{4}\hlstd{,} \hlnum{1}\hlstd{,} \hlnum{9}\hlstd{))}
\hlstd{base} \hlkwb{<-} \hlkwd{ggplot}\hlstd{(df,} \hlkwd{aes}\hlstd{(x, y))}
\hlstd{base} \hlopt{+} \hlkwd{geom_path}\hlstd{(}\hlkwc{size} \hlstd{=} \hlnum{10}\hlstd{)}
\end{alltt}
\end{kframe}
\includegraphics[width=\maxwidth]{figure/021-ggplot2-geoms-geom_step-10} 
\begin{kframe}\begin{alltt}
\hlstd{base} \hlopt{+} \hlkwd{geom_path}\hlstd{(}\hlkwc{size} \hlstd{=} \hlnum{10}\hlstd{,} \hlkwc{lineend} \hlstd{=} \hlstr{"round"}\hlstd{)}
\end{alltt}
\end{kframe}
\includegraphics[width=\maxwidth]{figure/021-ggplot2-geoms-geom_step-11} 
\begin{kframe}\begin{alltt}
\hlstd{base} \hlopt{+} \hlkwd{geom_path}\hlstd{(}\hlkwc{size} \hlstd{=} \hlnum{10}\hlstd{,} \hlkwc{linejoin} \hlstd{=} \hlstr{"mitre"}\hlstd{,} \hlkwc{lineend} \hlstd{=} \hlstr{"butt"}\hlstd{)}
\end{alltt}
\end{kframe}
\includegraphics[width=\maxwidth]{figure/021-ggplot2-geoms-geom_step-12} 
\begin{kframe}\begin{alltt}
\hlcom{# NAs break the line. Use na.rm = T to suppress the warning message}
\hlstd{df} \hlkwb{<-} \hlkwd{data.frame}\hlstd{(}
  \hlkwc{x} \hlstd{=} \hlnum{1}\hlopt{:}\hlnum{5}\hlstd{,}
  \hlkwc{y1} \hlstd{=} \hlkwd{c}\hlstd{(}\hlnum{1}\hlstd{,} \hlnum{2}\hlstd{,} \hlnum{3}\hlstd{,} \hlnum{4}\hlstd{,} \hlnum{NA}\hlstd{),}
  \hlkwc{y2} \hlstd{=} \hlkwd{c}\hlstd{(}\hlnum{NA}\hlstd{,} \hlnum{2}\hlstd{,} \hlnum{3}\hlstd{,} \hlnum{4}\hlstd{,} \hlnum{5}\hlstd{),}
  \hlkwc{y3} \hlstd{=} \hlkwd{c}\hlstd{(}\hlnum{1}\hlstd{,} \hlnum{2}\hlstd{,} \hlnum{NA}\hlstd{,} \hlnum{4}\hlstd{,} \hlnum{5}\hlstd{)}
\hlstd{)}
\hlkwd{ggplot}\hlstd{(df,} \hlkwd{aes}\hlstd{(x, y1))} \hlopt{+} \hlkwd{geom_point}\hlstd{()} \hlopt{+} \hlkwd{geom_line}\hlstd{()}
\end{alltt}


{\ttfamily\noindent\color{warningcolor}{\#\# Warning: Removed 1 rows containing missing values (geom\_point).}}

{\ttfamily\noindent\color{warningcolor}{\#\# Warning: Removed 1 rows containing missing values (geom\_path).}}\end{kframe}
\includegraphics[width=\maxwidth]{figure/021-ggplot2-geoms-geom_step-13} 
\begin{kframe}\begin{alltt}
\hlkwd{ggplot}\hlstd{(df,} \hlkwd{aes}\hlstd{(x, y2))} \hlopt{+} \hlkwd{geom_point}\hlstd{()} \hlopt{+} \hlkwd{geom_line}\hlstd{()}
\end{alltt}


{\ttfamily\noindent\color{warningcolor}{\#\# Warning: Removed 1 rows containing missing values (geom\_point).}}

{\ttfamily\noindent\color{warningcolor}{\#\# Warning: Removed 1 rows containing missing values (geom\_path).}}\end{kframe}
\includegraphics[width=\maxwidth]{figure/021-ggplot2-geoms-geom_step-14} 
\begin{kframe}\begin{alltt}
\hlkwd{ggplot}\hlstd{(df,} \hlkwd{aes}\hlstd{(x, y3))} \hlopt{+} \hlkwd{geom_point}\hlstd{()} \hlopt{+} \hlkwd{geom_line}\hlstd{()}
\end{alltt}


{\ttfamily\noindent\color{warningcolor}{\#\# Warning: Removed 1 rows containing missing values (geom\_point).}}\end{kframe}
\includegraphics[width=\maxwidth]{figure/021-ggplot2-geoms-geom_step-15} 
\begin{kframe}\begin{alltt}
\hlcom{## No test: }
\hlcom{# Setting line type vs colour/size}
\hlcom{# Line type needs to be applied to a line as a whole, so it can}
\hlcom{# not be used with colour or size that vary across a line}
\hlstd{x} \hlkwb{<-} \hlkwd{seq}\hlstd{(}\hlnum{0.01}\hlstd{,} \hlnum{.99}\hlstd{,} \hlkwc{length.out} \hlstd{=} \hlnum{100}\hlstd{)}
\hlstd{df} \hlkwb{<-} \hlkwd{data.frame}\hlstd{(}
  \hlkwc{x} \hlstd{=} \hlkwd{rep}\hlstd{(x,} \hlnum{2}\hlstd{),}
  \hlkwc{y} \hlstd{=} \hlkwd{c}\hlstd{(}\hlkwd{qlogis}\hlstd{(x),} \hlnum{2} \hlopt{*} \hlkwd{qlogis}\hlstd{(x)),}
  \hlkwc{group} \hlstd{=} \hlkwd{rep}\hlstd{(}\hlkwd{c}\hlstd{(}\hlstr{"a"}\hlstd{,}\hlstr{"b"}\hlstd{),}
  \hlkwc{each} \hlstd{=} \hlnum{100}\hlstd{)}
\hlstd{)}
\hlstd{p} \hlkwb{<-} \hlkwd{ggplot}\hlstd{(df,} \hlkwd{aes}\hlstd{(}\hlkwc{x}\hlstd{=x,} \hlkwc{y}\hlstd{=y,} \hlkwc{group}\hlstd{=group))}
\hlcom{# These work}
\hlstd{p} \hlopt{+} \hlkwd{geom_line}\hlstd{(}\hlkwc{linetype} \hlstd{=} \hlnum{2}\hlstd{)}
\end{alltt}
\end{kframe}
\includegraphics[width=\maxwidth]{figure/021-ggplot2-geoms-geom_step-16} 
\begin{kframe}\begin{alltt}
\hlstd{p} \hlopt{+} \hlkwd{geom_line}\hlstd{(}\hlkwd{aes}\hlstd{(}\hlkwc{colour} \hlstd{= group),} \hlkwc{linetype} \hlstd{=} \hlnum{2}\hlstd{)}
\end{alltt}
\end{kframe}
\includegraphics[width=\maxwidth]{figure/021-ggplot2-geoms-geom_step-17} 
\begin{kframe}\begin{alltt}
\hlstd{p} \hlopt{+} \hlkwd{geom_line}\hlstd{(}\hlkwd{aes}\hlstd{(}\hlkwc{colour} \hlstd{= x))}
\end{alltt}
\end{kframe}
\includegraphics[width=\maxwidth]{figure/021-ggplot2-geoms-geom_step-18} 
\begin{kframe}\begin{alltt}
\hlcom{# But this doesn't}
\hlkwd{should_stop}\hlstd{(p} \hlopt{+} \hlkwd{geom_line}\hlstd{(}\hlkwd{aes}\hlstd{(}\hlkwc{colour} \hlstd{= x),} \hlkwc{linetype}\hlstd{=}\hlnum{2}\hlstd{))}
\end{alltt}
\end{kframe}
\includegraphics[width=\maxwidth]{figure/021-ggplot2-geoms-geom_step-19} 
\begin{kframe}\begin{alltt}
\hlcom{## End(No test)}
\end{alltt}
\end{kframe}
\end{knitrout}


\section{geom\_text}

\begin{knitrout}
\definecolor{shadecolor}{rgb}{0.969, 0.969, 0.969}\color{fgcolor}\begin{kframe}
\begin{alltt}
\hlcom{### Name: geom_label}
\hlcom{### Title: Textual annotations.}
\hlcom{### Aliases: geom_label geom_text}

\hlcom{### ** Examples}

\hlstd{p} \hlkwb{<-} \hlkwd{ggplot}\hlstd{(mtcars,} \hlkwd{aes}\hlstd{(wt, mpg,} \hlkwc{label} \hlstd{=} \hlkwd{rownames}\hlstd{(mtcars)))}

\hlstd{p} \hlopt{+} \hlkwd{geom_text}\hlstd{()}
\end{alltt}
\end{kframe}
\includegraphics[width=\maxwidth]{figure/021-ggplot2-geoms-geom_text-1} 
\begin{kframe}\begin{alltt}
\hlcom{# Avoid overlaps}
\hlstd{p} \hlopt{+} \hlkwd{geom_text}\hlstd{(}\hlkwc{check_overlap} \hlstd{=} \hlnum{TRUE}\hlstd{)}
\end{alltt}
\end{kframe}
\includegraphics[width=\maxwidth]{figure/021-ggplot2-geoms-geom_text-2} 
\begin{kframe}\begin{alltt}
\hlcom{# Labels with background}
\hlstd{p} \hlopt{+} \hlkwd{geom_label}\hlstd{()}
\end{alltt}
\end{kframe}
\includegraphics[width=\maxwidth]{figure/021-ggplot2-geoms-geom_text-3} 
\begin{kframe}\begin{alltt}
\hlcom{# Change size of the label}
\hlstd{p} \hlopt{+} \hlkwd{geom_text}\hlstd{(}\hlkwc{size} \hlstd{=} \hlnum{10}\hlstd{)}
\end{alltt}
\end{kframe}
\includegraphics[width=\maxwidth]{figure/021-ggplot2-geoms-geom_text-4} 
\begin{kframe}\begin{alltt}
\hlcom{# Set aesthetics to fixed value}
\hlstd{p} \hlopt{+} \hlkwd{geom_point}\hlstd{()} \hlopt{+} \hlkwd{geom_text}\hlstd{(}\hlkwc{hjust} \hlstd{=} \hlnum{0}\hlstd{,} \hlkwc{nudge_x} \hlstd{=} \hlnum{0.05}\hlstd{)}
\end{alltt}
\end{kframe}
\includegraphics[width=\maxwidth]{figure/021-ggplot2-geoms-geom_text-5} 
\begin{kframe}\begin{alltt}
\hlstd{p} \hlopt{+} \hlkwd{geom_point}\hlstd{()} \hlopt{+} \hlkwd{geom_text}\hlstd{(}\hlkwc{vjust} \hlstd{=} \hlnum{0}\hlstd{,} \hlkwc{nudge_y} \hlstd{=} \hlnum{0.5}\hlstd{)}
\end{alltt}
\end{kframe}
\includegraphics[width=\maxwidth]{figure/021-ggplot2-geoms-geom_text-6} 
\begin{kframe}\begin{alltt}
\hlstd{p} \hlopt{+} \hlkwd{geom_point}\hlstd{()} \hlopt{+} \hlkwd{geom_text}\hlstd{(}\hlkwc{angle} \hlstd{=} \hlnum{45}\hlstd{)}
\end{alltt}
\end{kframe}
\includegraphics[width=\maxwidth]{figure/021-ggplot2-geoms-geom_text-7} 
\begin{kframe}\begin{alltt}
\hlcom{## Not run: }
\hlcom{##D p + geom_text(family = "Times New Roman")}
\hlcom{## End(Not run)}

\hlcom{# Add aesthetic mappings}
\hlstd{p} \hlopt{+} \hlkwd{geom_text}\hlstd{(}\hlkwd{aes}\hlstd{(}\hlkwc{colour} \hlstd{=} \hlkwd{factor}\hlstd{(cyl)))}
\end{alltt}
\end{kframe}
\includegraphics[width=\maxwidth]{figure/021-ggplot2-geoms-geom_text-8} 
\begin{kframe}\begin{alltt}
\hlstd{p} \hlopt{+} \hlkwd{geom_text}\hlstd{(}\hlkwd{aes}\hlstd{(}\hlkwc{colour} \hlstd{=} \hlkwd{factor}\hlstd{(cyl)))} \hlopt{+}
  \hlkwd{scale_colour_discrete}\hlstd{(}\hlkwc{l} \hlstd{=} \hlnum{40}\hlstd{)}
\end{alltt}
\end{kframe}
\includegraphics[width=\maxwidth]{figure/021-ggplot2-geoms-geom_text-9} 
\begin{kframe}\begin{alltt}
\hlstd{p} \hlopt{+} \hlkwd{geom_label}\hlstd{(}\hlkwd{aes}\hlstd{(}\hlkwc{fill} \hlstd{=} \hlkwd{factor}\hlstd{(cyl)),} \hlkwc{colour} \hlstd{=} \hlstr{"white"}\hlstd{,} \hlkwc{fontface} \hlstd{=} \hlstr{"bold"}\hlstd{)}
\end{alltt}
\end{kframe}
\includegraphics[width=\maxwidth]{figure/021-ggplot2-geoms-geom_text-10} 
\begin{kframe}\begin{alltt}
\hlstd{p} \hlopt{+} \hlkwd{geom_text}\hlstd{(}\hlkwd{aes}\hlstd{(}\hlkwc{size} \hlstd{= wt))}
\end{alltt}
\end{kframe}
\includegraphics[width=\maxwidth]{figure/021-ggplot2-geoms-geom_text-11} 
\begin{kframe}\begin{alltt}
\hlcom{# Scale height of text, rather than sqrt(height)}
\hlstd{p} \hlopt{+} \hlkwd{geom_text}\hlstd{(}\hlkwd{aes}\hlstd{(}\hlkwc{size} \hlstd{= wt))} \hlopt{+} \hlkwd{scale_radius}\hlstd{(}\hlkwc{range} \hlstd{=} \hlkwd{c}\hlstd{(}\hlnum{3}\hlstd{,}\hlnum{6}\hlstd{))}
\end{alltt}
\end{kframe}
\includegraphics[width=\maxwidth]{figure/021-ggplot2-geoms-geom_text-12} 
\begin{kframe}\begin{alltt}
\hlcom{# You can display expressions by setting parse = TRUE.  The}
\hlcom{# details of the display are described in ?plotmath, but note that}
\hlcom{# geom_text uses strings, not expressions.}
\hlstd{p} \hlopt{+} \hlkwd{geom_text}\hlstd{(}\hlkwd{aes}\hlstd{(}\hlkwc{label} \hlstd{=} \hlkwd{paste}\hlstd{(wt,} \hlstr{"^("}\hlstd{, cyl,} \hlstr{")"}\hlstd{,} \hlkwc{sep} \hlstd{=} \hlstr{""}\hlstd{)),}
  \hlkwc{parse} \hlstd{=} \hlnum{TRUE}\hlstd{)}
\end{alltt}
\end{kframe}
\includegraphics[width=\maxwidth]{figure/021-ggplot2-geoms-geom_text-13} 
\begin{kframe}\begin{alltt}
\hlcom{# Add a text annotation}
\hlstd{p} \hlopt{+}
  \hlkwd{geom_text}\hlstd{()} \hlopt{+}
  \hlkwd{annotate}\hlstd{(}\hlstr{"text"}\hlstd{,} \hlkwc{label} \hlstd{=} \hlstr{"plot mpg vs. wt"}\hlstd{,} \hlkwc{x} \hlstd{=} \hlnum{2}\hlstd{,} \hlkwc{y} \hlstd{=} \hlnum{15}\hlstd{,} \hlkwc{size} \hlstd{=} \hlnum{8}\hlstd{,} \hlkwc{colour} \hlstd{=} \hlstr{"red"}\hlstd{)}
\end{alltt}
\end{kframe}
\includegraphics[width=\maxwidth]{figure/021-ggplot2-geoms-geom_text-14} 
\begin{kframe}\begin{alltt}
\hlcom{## No test: }
\hlcom{# Aligning labels and bars --------------------------------------------------}
\hlstd{df} \hlkwb{<-} \hlkwd{data.frame}\hlstd{(}
  \hlkwc{x} \hlstd{=} \hlkwd{factor}\hlstd{(}\hlkwd{c}\hlstd{(}\hlnum{1}\hlstd{,} \hlnum{1}\hlstd{,} \hlnum{2}\hlstd{,} \hlnum{2}\hlstd{)),}
  \hlkwc{y} \hlstd{=} \hlkwd{c}\hlstd{(}\hlnum{1}\hlstd{,} \hlnum{3}\hlstd{,} \hlnum{2}\hlstd{,} \hlnum{1}\hlstd{),}
  \hlkwc{grp} \hlstd{=} \hlkwd{c}\hlstd{(}\hlstr{"a"}\hlstd{,} \hlstr{"b"}\hlstd{,} \hlstr{"a"}\hlstd{,} \hlstr{"b"}\hlstd{)}
\hlstd{)}

\hlcom{# ggplot2 doesn't know you want to give the labels the same virtual width}
\hlcom{# as the bars:}
\hlkwd{ggplot}\hlstd{(}\hlkwc{data} \hlstd{= df,} \hlkwd{aes}\hlstd{(x, y,} \hlkwc{fill} \hlstd{= grp,} \hlkwc{label} \hlstd{= y))} \hlopt{+}
  \hlkwd{geom_bar}\hlstd{(}\hlkwc{stat} \hlstd{=} \hlstr{"identity"}\hlstd{,} \hlkwc{position} \hlstd{=} \hlstr{"dodge"}\hlstd{)} \hlopt{+}
  \hlkwd{geom_text}\hlstd{(}\hlkwc{position} \hlstd{=} \hlstr{"dodge"}\hlstd{)}
\end{alltt}


{\ttfamily\noindent\color{warningcolor}{\#\# Warning: Width not defined. Set with `position\_dodge(width = ?)`}}\end{kframe}
\includegraphics[width=\maxwidth]{figure/021-ggplot2-geoms-geom_text-15} 
\begin{kframe}\begin{alltt}
\hlcom{# So tell it:}
\hlkwd{ggplot}\hlstd{(}\hlkwc{data} \hlstd{= df,} \hlkwd{aes}\hlstd{(x, y,} \hlkwc{fill} \hlstd{= grp,} \hlkwc{label} \hlstd{= y))} \hlopt{+}
  \hlkwd{geom_bar}\hlstd{(}\hlkwc{stat} \hlstd{=} \hlstr{"identity"}\hlstd{,} \hlkwc{position} \hlstd{=} \hlstr{"dodge"}\hlstd{)} \hlopt{+}
  \hlkwd{geom_text}\hlstd{(}\hlkwc{position} \hlstd{=} \hlkwd{position_dodge}\hlstd{(}\hlnum{0.9}\hlstd{))}
\end{alltt}
\end{kframe}
\includegraphics[width=\maxwidth]{figure/021-ggplot2-geoms-geom_text-16} 
\begin{kframe}\begin{alltt}
\hlcom{# Use you can't nudge and dodge text, so instead adjust the y postion}
\hlkwd{ggplot}\hlstd{(}\hlkwc{data} \hlstd{= df,} \hlkwd{aes}\hlstd{(x, y,} \hlkwc{fill} \hlstd{= grp,} \hlkwc{label} \hlstd{= y))} \hlopt{+}
  \hlkwd{geom_bar}\hlstd{(}\hlkwc{stat} \hlstd{=} \hlstr{"identity"}\hlstd{,} \hlkwc{position} \hlstd{=} \hlstr{"dodge"}\hlstd{)} \hlopt{+}
  \hlkwd{geom_text}\hlstd{(}\hlkwd{aes}\hlstd{(}\hlkwc{y} \hlstd{= y} \hlopt{+} \hlnum{0.05}\hlstd{),} \hlkwc{position} \hlstd{=} \hlkwd{position_dodge}\hlstd{(}\hlnum{0.9}\hlstd{),} \hlkwc{vjust} \hlstd{=} \hlnum{0}\hlstd{)}
\end{alltt}
\end{kframe}
\includegraphics[width=\maxwidth]{figure/021-ggplot2-geoms-geom_text-17} 
\begin{kframe}\begin{alltt}
\hlcom{# To place text in the middle of each bar in a stacked barplot, you}
\hlcom{# need to do the computation yourself}
\hlstd{df} \hlkwb{<-} \hlkwd{transform}\hlstd{(df,} \hlkwc{mid_y} \hlstd{=} \hlkwd{ave}\hlstd{(df}\hlopt{$}\hlstd{y, df}\hlopt{$}\hlstd{x,} \hlkwc{FUN} \hlstd{=} \hlkwa{function}\hlstd{(}\hlkwc{val}\hlstd{)} \hlkwd{cumsum}\hlstd{(val)} \hlopt{-} \hlstd{(}\hlnum{0.5} \hlopt{*} \hlstd{val)))}

\hlkwd{ggplot}\hlstd{(}\hlkwc{data} \hlstd{= df,} \hlkwd{aes}\hlstd{(x, y,} \hlkwc{fill} \hlstd{= grp,} \hlkwc{label} \hlstd{= y))} \hlopt{+}
 \hlkwd{geom_bar}\hlstd{(}\hlkwc{stat} \hlstd{=} \hlstr{"identity"}\hlstd{)} \hlopt{+}
 \hlkwd{geom_text}\hlstd{(}\hlkwd{aes}\hlstd{(}\hlkwc{y} \hlstd{= mid_y))}
\end{alltt}
\end{kframe}
\includegraphics[width=\maxwidth]{figure/021-ggplot2-geoms-geom_text-18} 
\begin{kframe}\begin{alltt}
\hlcom{# Justification -------------------------------------------------------------}
\hlstd{df} \hlkwb{<-} \hlkwd{data.frame}\hlstd{(}
  \hlkwc{x} \hlstd{=} \hlkwd{c}\hlstd{(}\hlnum{1}\hlstd{,} \hlnum{1}\hlstd{,} \hlnum{2}\hlstd{,} \hlnum{2}\hlstd{,} \hlnum{1.5}\hlstd{),}
  \hlkwc{y} \hlstd{=} \hlkwd{c}\hlstd{(}\hlnum{1}\hlstd{,} \hlnum{2}\hlstd{,} \hlnum{1}\hlstd{,} \hlnum{2}\hlstd{,} \hlnum{1.5}\hlstd{),}
  \hlkwc{text} \hlstd{=} \hlkwd{c}\hlstd{(}\hlstr{"bottom-left"}\hlstd{,} \hlstr{"bottom-right"}\hlstd{,} \hlstr{"top-left"}\hlstd{,} \hlstr{"top-right"}\hlstd{,} \hlstr{"center"}\hlstd{)}
\hlstd{)}
\hlkwd{ggplot}\hlstd{(df,} \hlkwd{aes}\hlstd{(x, y))} \hlopt{+}
  \hlkwd{geom_text}\hlstd{(}\hlkwd{aes}\hlstd{(}\hlkwc{label} \hlstd{= text))}
\end{alltt}
\end{kframe}
\includegraphics[width=\maxwidth]{figure/021-ggplot2-geoms-geom_text-19} 
\begin{kframe}\begin{alltt}
\hlkwd{ggplot}\hlstd{(df,} \hlkwd{aes}\hlstd{(x, y))} \hlopt{+}
  \hlkwd{geom_text}\hlstd{(}\hlkwd{aes}\hlstd{(}\hlkwc{label} \hlstd{= text),} \hlkwc{vjust} \hlstd{=} \hlstr{"inward"}\hlstd{,} \hlkwc{hjust} \hlstd{=} \hlstr{"inward"}\hlstd{)}
\end{alltt}
\end{kframe}
\includegraphics[width=\maxwidth]{figure/021-ggplot2-geoms-geom_text-20} 
\begin{kframe}\begin{alltt}
\hlcom{## End(No test)}
\end{alltt}
\end{kframe}
\end{knitrout}


\section{geom\_tile}

\begin{knitrout}
\definecolor{shadecolor}{rgb}{0.969, 0.969, 0.969}\color{fgcolor}\begin{kframe}
\begin{alltt}
\hlcom{### Name: geom_raster}
\hlcom{### Title: Draw rectangles.}
\hlcom{### Aliases: geom_raster geom_rect geom_tile}

\hlcom{### ** Examples}

\hlcom{# The most common use for rectangles is to draw a surface. You always want}
\hlcom{# to use geom_raster here because it's so much faster, and produces}
\hlcom{# smaller output when saving to PDF}
\hlkwd{ggplot}\hlstd{(faithfuld,} \hlkwd{aes}\hlstd{(waiting, eruptions))} \hlopt{+}
 \hlkwd{geom_raster}\hlstd{(}\hlkwd{aes}\hlstd{(}\hlkwc{fill} \hlstd{= density))}
\end{alltt}
\end{kframe}
\includegraphics[width=\maxwidth]{figure/021-ggplot2-geoms-geom_tile-1} 
\begin{kframe}\begin{alltt}
\hlcom{# Interpolation smooths the surface & is most helpful when rendering images.}
\hlkwd{ggplot}\hlstd{(faithfuld,} \hlkwd{aes}\hlstd{(waiting, eruptions))} \hlopt{+}
 \hlkwd{geom_raster}\hlstd{(}\hlkwd{aes}\hlstd{(}\hlkwc{fill} \hlstd{= density),} \hlkwc{interpolate} \hlstd{=} \hlnum{TRUE}\hlstd{)}
\end{alltt}
\end{kframe}
\includegraphics[width=\maxwidth]{figure/021-ggplot2-geoms-geom_tile-2} 
\begin{kframe}\begin{alltt}
\hlcom{# If you want to draw arbitrary rectangles, use geom_tile() or geom_rect()}
\hlstd{df} \hlkwb{<-} \hlkwd{data.frame}\hlstd{(}
  \hlkwc{x} \hlstd{=} \hlkwd{rep}\hlstd{(}\hlkwd{c}\hlstd{(}\hlnum{2}\hlstd{,} \hlnum{5}\hlstd{,} \hlnum{7}\hlstd{,} \hlnum{9}\hlstd{,} \hlnum{12}\hlstd{),} \hlnum{2}\hlstd{),}
  \hlkwc{y} \hlstd{=} \hlkwd{rep}\hlstd{(}\hlkwd{c}\hlstd{(}\hlnum{1}\hlstd{,} \hlnum{2}\hlstd{),} \hlkwc{each} \hlstd{=} \hlnum{5}\hlstd{),}
  \hlkwc{z} \hlstd{=} \hlkwd{factor}\hlstd{(}\hlkwd{rep}\hlstd{(}\hlnum{1}\hlopt{:}\hlnum{5}\hlstd{,} \hlkwc{each} \hlstd{=} \hlnum{2}\hlstd{)),}
  \hlkwc{w} \hlstd{=} \hlkwd{rep}\hlstd{(}\hlkwd{diff}\hlstd{(}\hlkwd{c}\hlstd{(}\hlnum{0}\hlstd{,} \hlnum{4}\hlstd{,} \hlnum{6}\hlstd{,} \hlnum{8}\hlstd{,} \hlnum{10}\hlstd{,} \hlnum{14}\hlstd{)),} \hlnum{2}\hlstd{)}
\hlstd{)}
\hlkwd{ggplot}\hlstd{(df,} \hlkwd{aes}\hlstd{(x, y))} \hlopt{+}
  \hlkwd{geom_tile}\hlstd{(}\hlkwd{aes}\hlstd{(}\hlkwc{fill} \hlstd{= z))}
\end{alltt}
\end{kframe}
\includegraphics[width=\maxwidth]{figure/021-ggplot2-geoms-geom_tile-3} 
\begin{kframe}\begin{alltt}
\hlkwd{ggplot}\hlstd{(df,} \hlkwd{aes}\hlstd{(x, y))} \hlopt{+}
  \hlkwd{geom_tile}\hlstd{(}\hlkwd{aes}\hlstd{(}\hlkwc{fill} \hlstd{= z,} \hlkwc{width} \hlstd{= w),} \hlkwc{colour} \hlstd{=} \hlstr{"grey50"}\hlstd{)}
\end{alltt}
\end{kframe}
\includegraphics[width=\maxwidth]{figure/021-ggplot2-geoms-geom_tile-4} 
\begin{kframe}\begin{alltt}
\hlkwd{ggplot}\hlstd{(df,} \hlkwd{aes}\hlstd{(}\hlkwc{xmin} \hlstd{= x} \hlopt{-} \hlstd{w} \hlopt{/} \hlnum{2}\hlstd{,} \hlkwc{xmax} \hlstd{= x} \hlopt{+} \hlstd{w} \hlopt{/} \hlnum{2}\hlstd{,} \hlkwc{ymin} \hlstd{= y,} \hlkwc{ymax} \hlstd{= y} \hlopt{+} \hlnum{1}\hlstd{))} \hlopt{+}
  \hlkwd{geom_rect}\hlstd{(}\hlkwd{aes}\hlstd{(}\hlkwc{fill} \hlstd{= z,} \hlkwc{width} \hlstd{= w),} \hlkwc{colour} \hlstd{=} \hlstr{"grey50"}\hlstd{)}
\end{alltt}
\end{kframe}
\includegraphics[width=\maxwidth]{figure/021-ggplot2-geoms-geom_tile-5} 
\begin{kframe}\begin{alltt}
\hlcom{## No test: }
\hlcom{# Justification controls where the cells are anchored}
\hlstd{df} \hlkwb{<-} \hlkwd{expand.grid}\hlstd{(}\hlkwc{x} \hlstd{=} \hlnum{0}\hlopt{:}\hlnum{5}\hlstd{,} \hlkwc{y} \hlstd{=} \hlnum{0}\hlopt{:}\hlnum{5}\hlstd{)}
\hlstd{df}\hlopt{$}\hlstd{z} \hlkwb{<-} \hlkwd{runif}\hlstd{(}\hlkwd{nrow}\hlstd{(df))}
\hlcom{# default is compatible with geom_tile()}
\hlkwd{ggplot}\hlstd{(df,} \hlkwd{aes}\hlstd{(x, y,} \hlkwc{fill} \hlstd{= z))} \hlopt{+} \hlkwd{geom_raster}\hlstd{()}
\end{alltt}
\end{kframe}
\includegraphics[width=\maxwidth]{figure/021-ggplot2-geoms-geom_tile-6} 
\begin{kframe}\begin{alltt}
\hlcom{# zero padding}
\hlkwd{ggplot}\hlstd{(df,} \hlkwd{aes}\hlstd{(x, y,} \hlkwc{fill} \hlstd{= z))} \hlopt{+} \hlkwd{geom_raster}\hlstd{(}\hlkwc{hjust} \hlstd{=} \hlnum{0}\hlstd{,} \hlkwc{vjust} \hlstd{=} \hlnum{0}\hlstd{)}
\end{alltt}
\end{kframe}
\includegraphics[width=\maxwidth]{figure/021-ggplot2-geoms-geom_tile-7} 
\begin{kframe}\begin{alltt}
\hlcom{# Inspired by the image-density plots of Ken Knoblauch}
\hlstd{cars} \hlkwb{<-} \hlkwd{ggplot}\hlstd{(mtcars,} \hlkwd{aes}\hlstd{(mpg,} \hlkwd{factor}\hlstd{(cyl)))}
\hlstd{cars} \hlopt{+} \hlkwd{geom_point}\hlstd{()}
\end{alltt}
\end{kframe}
\includegraphics[width=\maxwidth]{figure/021-ggplot2-geoms-geom_tile-8} 
\begin{kframe}\begin{alltt}
\hlstd{cars} \hlopt{+} \hlkwd{stat_bin2d}\hlstd{(}\hlkwd{aes}\hlstd{(}\hlkwc{fill} \hlstd{= ..count..),} \hlkwc{binwidth} \hlstd{=} \hlkwd{c}\hlstd{(}\hlnum{3}\hlstd{,}\hlnum{1}\hlstd{))}
\end{alltt}
\end{kframe}
\includegraphics[width=\maxwidth]{figure/021-ggplot2-geoms-geom_tile-9} 
\begin{kframe}\begin{alltt}
\hlstd{cars} \hlopt{+} \hlkwd{stat_bin2d}\hlstd{(}\hlkwd{aes}\hlstd{(}\hlkwc{fill} \hlstd{= ..density..),} \hlkwc{binwidth} \hlstd{=} \hlkwd{c}\hlstd{(}\hlnum{3}\hlstd{,}\hlnum{1}\hlstd{))}
\end{alltt}
\end{kframe}
\includegraphics[width=\maxwidth]{figure/021-ggplot2-geoms-geom_tile-10} 
\begin{kframe}\begin{alltt}
\hlstd{cars} \hlopt{+} \hlkwd{stat_density}\hlstd{(}\hlkwd{aes}\hlstd{(}\hlkwc{fill} \hlstd{= ..density..),} \hlkwc{geom} \hlstd{=} \hlstr{"raster"}\hlstd{,} \hlkwc{position} \hlstd{=} \hlstr{"identity"}\hlstd{)}
\end{alltt}
\end{kframe}
\includegraphics[width=\maxwidth]{figure/021-ggplot2-geoms-geom_tile-11} 
\begin{kframe}\begin{alltt}
\hlstd{cars} \hlopt{+} \hlkwd{stat_density}\hlstd{(}\hlkwd{aes}\hlstd{(}\hlkwc{fill} \hlstd{= ..count..),} \hlkwc{geom} \hlstd{=} \hlstr{"raster"}\hlstd{,} \hlkwc{position} \hlstd{=} \hlstr{"identity"}\hlstd{)}
\end{alltt}
\end{kframe}
\includegraphics[width=\maxwidth]{figure/021-ggplot2-geoms-geom_tile-12} 
\begin{kframe}\begin{alltt}
\hlcom{## End(No test)}
\end{alltt}
\end{kframe}
\end{knitrout}


\section{geom\_violin}

\begin{knitrout}
\definecolor{shadecolor}{rgb}{0.969, 0.969, 0.969}\color{fgcolor}\begin{kframe}
\begin{alltt}
\hlcom{### Name: geom_violin}
\hlcom{### Title: Violin plot.}
\hlcom{### Aliases: geom_violin stat_ydensity}

\hlcom{### ** Examples}

\hlstd{p} \hlkwb{<-} \hlkwd{ggplot}\hlstd{(mtcars,} \hlkwd{aes}\hlstd{(}\hlkwd{factor}\hlstd{(cyl), mpg))}
\hlstd{p} \hlopt{+} \hlkwd{geom_violin}\hlstd{()}
\end{alltt}
\end{kframe}
\includegraphics[width=\maxwidth]{figure/021-ggplot2-geoms-geom_violin-1} 
\begin{kframe}\begin{alltt}
\hlcom{## No test: }
\hlstd{p} \hlopt{+} \hlkwd{geom_violin}\hlstd{()} \hlopt{+} \hlkwd{geom_jitter}\hlstd{(}\hlkwc{height} \hlstd{=} \hlnum{0}\hlstd{)}
\end{alltt}
\end{kframe}
\includegraphics[width=\maxwidth]{figure/021-ggplot2-geoms-geom_violin-2} 
\begin{kframe}\begin{alltt}
\hlstd{p} \hlopt{+} \hlkwd{geom_violin}\hlstd{()} \hlopt{+} \hlkwd{coord_flip}\hlstd{()}
\end{alltt}
\end{kframe}
\includegraphics[width=\maxwidth]{figure/021-ggplot2-geoms-geom_violin-3} 
\begin{kframe}\begin{alltt}
\hlcom{# Scale maximum width proportional to sample size:}
\hlstd{p} \hlopt{+} \hlkwd{geom_violin}\hlstd{(}\hlkwc{scale} \hlstd{=} \hlstr{"count"}\hlstd{)}
\end{alltt}
\end{kframe}
\includegraphics[width=\maxwidth]{figure/021-ggplot2-geoms-geom_violin-4} 
\begin{kframe}\begin{alltt}
\hlcom{# Scale maximum width to 1 for all violins:}
\hlstd{p} \hlopt{+} \hlkwd{geom_violin}\hlstd{(}\hlkwc{scale} \hlstd{=} \hlstr{"width"}\hlstd{)}
\end{alltt}
\end{kframe}
\includegraphics[width=\maxwidth]{figure/021-ggplot2-geoms-geom_violin-5} 
\begin{kframe}\begin{alltt}
\hlcom{# Default is to trim violins to the range of the data. To disable:}
\hlstd{p} \hlopt{+} \hlkwd{geom_violin}\hlstd{(}\hlkwc{trim} \hlstd{=} \hlnum{FALSE}\hlstd{)}
\end{alltt}
\end{kframe}
\includegraphics[width=\maxwidth]{figure/021-ggplot2-geoms-geom_violin-6} 
\begin{kframe}\begin{alltt}
\hlcom{# Use a smaller bandwidth for closer density fit (default is 1).}
\hlstd{p} \hlopt{+} \hlkwd{geom_violin}\hlstd{(}\hlkwc{adjust} \hlstd{=} \hlnum{.5}\hlstd{)}
\end{alltt}
\end{kframe}
\includegraphics[width=\maxwidth]{figure/021-ggplot2-geoms-geom_violin-7} 
\begin{kframe}\begin{alltt}
\hlcom{# Add aesthetic mappings}
\hlcom{# Note that violins are automatically dodged when any aesthetic is}
\hlcom{# a factor}
\hlstd{p} \hlopt{+} \hlkwd{geom_violin}\hlstd{(}\hlkwd{aes}\hlstd{(}\hlkwc{fill} \hlstd{= cyl))}
\end{alltt}
\end{kframe}
\includegraphics[width=\maxwidth]{figure/021-ggplot2-geoms-geom_violin-8} 
\begin{kframe}\begin{alltt}
\hlstd{p} \hlopt{+} \hlkwd{geom_violin}\hlstd{(}\hlkwd{aes}\hlstd{(}\hlkwc{fill} \hlstd{=} \hlkwd{factor}\hlstd{(cyl)))}
\end{alltt}
\end{kframe}
\includegraphics[width=\maxwidth]{figure/021-ggplot2-geoms-geom_violin-9} 
\begin{kframe}\begin{alltt}
\hlstd{p} \hlopt{+} \hlkwd{geom_violin}\hlstd{(}\hlkwd{aes}\hlstd{(}\hlkwc{fill} \hlstd{=} \hlkwd{factor}\hlstd{(vs)))}
\end{alltt}
\end{kframe}
\includegraphics[width=\maxwidth]{figure/021-ggplot2-geoms-geom_violin-10} 
\begin{kframe}\begin{alltt}
\hlstd{p} \hlopt{+} \hlkwd{geom_violin}\hlstd{(}\hlkwd{aes}\hlstd{(}\hlkwc{fill} \hlstd{=} \hlkwd{factor}\hlstd{(am)))}
\end{alltt}
\end{kframe}
\includegraphics[width=\maxwidth]{figure/021-ggplot2-geoms-geom_violin-11} 
\begin{kframe}\begin{alltt}
\hlcom{# Set aesthetics to fixed value}
\hlstd{p} \hlopt{+} \hlkwd{geom_violin}\hlstd{(}\hlkwc{fill} \hlstd{=} \hlstr{"grey80"}\hlstd{,} \hlkwc{colour} \hlstd{=} \hlstr{"#3366FF"}\hlstd{)}
\end{alltt}
\end{kframe}
\includegraphics[width=\maxwidth]{figure/021-ggplot2-geoms-geom_violin-12} 
\begin{kframe}\begin{alltt}
\hlcom{# Show quartiles}
\hlstd{p} \hlopt{+} \hlkwd{geom_violin}\hlstd{(}\hlkwc{draw_quantiles} \hlstd{=} \hlkwd{c}\hlstd{(}\hlnum{0.25}\hlstd{,} \hlnum{0.5}\hlstd{,} \hlnum{0.75}\hlstd{))}
\end{alltt}
\end{kframe}
\includegraphics[width=\maxwidth]{figure/021-ggplot2-geoms-geom_violin-13} 
\begin{kframe}\begin{alltt}
\hlcom{# Scales vs. coordinate transforms -------}
\hlkwa{if} \hlstd{(}\hlkwd{require}\hlstd{(}\hlstr{"ggplot2movies"}\hlstd{)) \{}
\hlcom{# Scale transformations occur before the density statistics are computed.}
\hlcom{# Coordinate transformations occur afterwards.  Observe the effect on the}
\hlcom{# number of outliers.}
\hlstd{m} \hlkwb{<-} \hlkwd{ggplot}\hlstd{(movies,} \hlkwd{aes}\hlstd{(}\hlkwc{y} \hlstd{= votes,} \hlkwc{x} \hlstd{= rating,} \hlkwc{group} \hlstd{=} \hlkwd{cut_width}\hlstd{(rating,} \hlnum{0.5}\hlstd{)))}
\hlstd{m} \hlopt{+} \hlkwd{geom_violin}\hlstd{()}
\hlstd{m} \hlopt{+} \hlkwd{geom_violin}\hlstd{()} \hlopt{+} \hlkwd{scale_y_log10}\hlstd{()}
\hlstd{m} \hlopt{+} \hlkwd{geom_violin}\hlstd{()} \hlopt{+} \hlkwd{coord_trans}\hlstd{(}\hlkwc{y} \hlstd{=} \hlstr{"log10"}\hlstd{)}
\hlstd{m} \hlopt{+} \hlkwd{geom_violin}\hlstd{()} \hlopt{+} \hlkwd{scale_y_log10}\hlstd{()} \hlopt{+} \hlkwd{coord_trans}\hlstd{(}\hlkwc{y} \hlstd{=} \hlstr{"log10"}\hlstd{)}

\hlcom{# Violin plots with continuous x:}
\hlcom{# Use the group aesthetic to group observations in violins}
\hlkwd{ggplot}\hlstd{(movies,} \hlkwd{aes}\hlstd{(year, budget))} \hlopt{+} \hlkwd{geom_violin}\hlstd{()}
\hlkwd{ggplot}\hlstd{(movies,} \hlkwd{aes}\hlstd{(year, budget))} \hlopt{+}
  \hlkwd{geom_violin}\hlstd{(}\hlkwd{aes}\hlstd{(}\hlkwc{group} \hlstd{=} \hlkwd{cut_width}\hlstd{(year,} \hlnum{10}\hlstd{)),} \hlkwc{scale} \hlstd{=} \hlstr{"width"}\hlstd{)}
\hlstd{\}}
\end{alltt}


{\ttfamily\noindent\itshape\color{messagecolor}{\#\# Loading required package: ggplot2movies}}

{\ttfamily\noindent\color{warningcolor}{\#\# Warning in library(package, lib.loc = lib.loc, character.only = TRUE, logical.return = TRUE, : there is no package called 'ggplot2movies'}}\begin{alltt}
\hlcom{## End(No test)}
\end{alltt}
\end{kframe}
\end{knitrout}


\section{geom\_vline}

\begin{knitrout}
\definecolor{shadecolor}{rgb}{0.969, 0.969, 0.969}\color{fgcolor}\begin{kframe}
\begin{alltt}
\hlcom{### Name: geom_abline}
\hlcom{### Title: Lines: horizontal, vertical, and specified by slope and}
\hlcom{###   intercept.}
\hlcom{### Aliases: geom_abline geom_hline geom_vline}

\hlcom{### ** Examples}

\hlstd{p} \hlkwb{<-} \hlkwd{ggplot}\hlstd{(mtcars,} \hlkwd{aes}\hlstd{(wt, mpg))} \hlopt{+} \hlkwd{geom_point}\hlstd{()}

\hlcom{# Fixed values}
\hlstd{p} \hlopt{+} \hlkwd{geom_vline}\hlstd{(}\hlkwc{xintercept} \hlstd{=} \hlnum{5}\hlstd{)}
\end{alltt}
\end{kframe}
\includegraphics[width=\maxwidth]{figure/021-ggplot2-geoms-geom_vline-1} 
\begin{kframe}\begin{alltt}
\hlstd{p} \hlopt{+} \hlkwd{geom_vline}\hlstd{(}\hlkwc{xintercept} \hlstd{=} \hlnum{1}\hlopt{:}\hlnum{5}\hlstd{)}
\end{alltt}
\end{kframe}
\includegraphics[width=\maxwidth]{figure/021-ggplot2-geoms-geom_vline-2} 
\begin{kframe}\begin{alltt}
\hlstd{p} \hlopt{+} \hlkwd{geom_hline}\hlstd{(}\hlkwc{yintercept} \hlstd{=} \hlnum{20}\hlstd{)}
\end{alltt}
\end{kframe}
\includegraphics[width=\maxwidth]{figure/021-ggplot2-geoms-geom_vline-3} 
\begin{kframe}\begin{alltt}
\hlstd{p} \hlopt{+} \hlkwd{geom_abline}\hlstd{()} \hlcom{# Can't see it - outside the range of the data}
\end{alltt}
\end{kframe}
\includegraphics[width=\maxwidth]{figure/021-ggplot2-geoms-geom_vline-4} 
\begin{kframe}\begin{alltt}
\hlstd{p} \hlopt{+} \hlkwd{geom_abline}\hlstd{(}\hlkwc{intercept} \hlstd{=} \hlnum{20}\hlstd{)}
\end{alltt}
\end{kframe}
\includegraphics[width=\maxwidth]{figure/021-ggplot2-geoms-geom_vline-5} 
\begin{kframe}\begin{alltt}
\hlcom{# Calculate slope and intercept of line of best fit}
\hlkwd{coef}\hlstd{(}\hlkwd{lm}\hlstd{(mpg} \hlopt{~} \hlstd{wt,} \hlkwc{data} \hlstd{= mtcars))}
\end{alltt}
\begin{verbatim}
## (Intercept)          wt 
##      37.285      -5.344
\end{verbatim}
\begin{alltt}
\hlstd{p} \hlopt{+} \hlkwd{geom_abline}\hlstd{(}\hlkwc{intercept} \hlstd{=} \hlnum{37}\hlstd{,} \hlkwc{slope} \hlstd{=} \hlopt{-}\hlnum{5}\hlstd{)}
\end{alltt}
\end{kframe}
\includegraphics[width=\maxwidth]{figure/021-ggplot2-geoms-geom_vline-6} 
\begin{kframe}\begin{alltt}
\hlcom{# But this is easier to do with geom_smooth:}
\hlstd{p} \hlopt{+} \hlkwd{geom_smooth}\hlstd{(}\hlkwc{method} \hlstd{=} \hlstr{"lm"}\hlstd{,} \hlkwc{se} \hlstd{=} \hlnum{FALSE}\hlstd{)}
\end{alltt}
\end{kframe}
\includegraphics[width=\maxwidth]{figure/021-ggplot2-geoms-geom_vline-7} 
\begin{kframe}\begin{alltt}
\hlcom{# To show different lines in different facets, use aesthetics}
\hlstd{p} \hlkwb{<-} \hlkwd{ggplot}\hlstd{(mtcars,} \hlkwd{aes}\hlstd{(mpg, wt))} \hlopt{+}
  \hlkwd{geom_point}\hlstd{()} \hlopt{+}
  \hlkwd{facet_wrap}\hlstd{(}\hlopt{~} \hlstd{cyl)}

\hlstd{mean_wt} \hlkwb{<-} \hlkwd{data.frame}\hlstd{(}\hlkwc{cyl} \hlstd{=} \hlkwd{c}\hlstd{(}\hlnum{4}\hlstd{,} \hlnum{6}\hlstd{,} \hlnum{8}\hlstd{),} \hlkwc{wt} \hlstd{=} \hlkwd{c}\hlstd{(}\hlnum{2.28}\hlstd{,} \hlnum{3.11}\hlstd{,} \hlnum{4.00}\hlstd{))}
\hlstd{p} \hlopt{+} \hlkwd{geom_hline}\hlstd{(}\hlkwd{aes}\hlstd{(}\hlkwc{yintercept} \hlstd{= wt), mean_wt)}
\end{alltt}
\end{kframe}
\includegraphics[width=\maxwidth]{figure/021-ggplot2-geoms-geom_vline-8} 
\begin{kframe}\begin{alltt}
\hlcom{# You can also control other aesthetics}
\hlkwd{ggplot}\hlstd{(mtcars,} \hlkwd{aes}\hlstd{(mpg, wt,} \hlkwc{colour} \hlstd{= wt))} \hlopt{+}
  \hlkwd{geom_point}\hlstd{()} \hlopt{+}
  \hlkwd{geom_hline}\hlstd{(}\hlkwd{aes}\hlstd{(}\hlkwc{yintercept} \hlstd{= wt,} \hlkwc{colour} \hlstd{= wt), mean_wt)} \hlopt{+}
  \hlkwd{facet_wrap}\hlstd{(}\hlopt{~} \hlstd{cyl)}
\end{alltt}
\end{kframe}
\includegraphics[width=\maxwidth]{figure/021-ggplot2-geoms-geom_vline-9} 

\end{knitrout}



\end{document}
