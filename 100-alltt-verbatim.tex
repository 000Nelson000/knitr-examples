\documentclass{beamer}\usepackage[]{graphicx}\usepackage[]{color}
%% maxwidth is the original width if it is less than linewidth
%% otherwise use linewidth (to make sure the graphics do not exceed the margin)
\makeatletter
\def\maxwidth{ %
  \ifdim\Gin@nat@width>\linewidth
    \linewidth
  \else
    \Gin@nat@width
  \fi
}
\makeatother

\definecolor{fgcolor}{rgb}{0.345, 0.345, 0.345}
\newcommand{\hlnum}[1]{\textcolor[rgb]{0.686,0.059,0.569}{#1}}%
\newcommand{\hlstr}[1]{\textcolor[rgb]{0.192,0.494,0.8}{#1}}%
\newcommand{\hlcom}[1]{\textcolor[rgb]{0.678,0.584,0.686}{\textit{#1}}}%
\newcommand{\hlopt}[1]{\textcolor[rgb]{0,0,0}{#1}}%
\newcommand{\hlstd}[1]{\textcolor[rgb]{0.345,0.345,0.345}{#1}}%
\newcommand{\hlkwa}[1]{\textcolor[rgb]{0.161,0.373,0.58}{\textbf{#1}}}%
\newcommand{\hlkwb}[1]{\textcolor[rgb]{0.69,0.353,0.396}{#1}}%
\newcommand{\hlkwc}[1]{\textcolor[rgb]{0.333,0.667,0.333}{#1}}%
\newcommand{\hlkwd}[1]{\textcolor[rgb]{0.737,0.353,0.396}{\textbf{#1}}}%

\usepackage{framed}
\makeatletter
\newenvironment{kframe}{%
 \def\at@end@of@kframe{}%
 \ifinner\ifhmode%
  \def\at@end@of@kframe{\end{minipage}}%
  \begin{minipage}{\columnwidth}%
 \fi\fi%
 \def\FrameCommand##1{\hskip\@totalleftmargin \hskip-\fboxsep
 \colorbox{shadecolor}{##1}\hskip-\fboxsep
     % There is no \\@totalrightmargin, so:
     \hskip-\linewidth \hskip-\@totalleftmargin \hskip\columnwidth}%
 \MakeFramed {\advance\hsize-\width
   \@totalleftmargin\z@ \linewidth\hsize
   \@setminipage}}%
 {\par\unskip\endMakeFramed%
 \at@end@of@kframe}
\makeatother

\definecolor{shadecolor}{rgb}{.97, .97, .97}
\definecolor{messagecolor}{rgb}{0, 0, 0}
\definecolor{warningcolor}{rgb}{1, 0, 1}
\definecolor{errorcolor}{rgb}{1, 0, 0}
\newenvironment{knitrout}{}{} % an empty environment to be redefined in TeX

\usepackage{alltt}
\IfFileExists{upquote.sty}{\usepackage{upquote}}{}

\begin{document}

\begin{frame}[fragile]
This example shows you how to separate the source and output boxes by injecting
the \texttt{kframe} environments between them.

\begin{knitrout}
\definecolor{shadecolor}{rgb}{0.969, 0.969, 0.969}\color{fgcolor}\begin{kframe}
\begin{alltt}
\hlcom{# modify the default chunk hook}
\hlstd{hook_chunk} \hlkwb{=} \hlstd{knit_hooks}\hlopt{$}\hlkwd{get}\hlstd{(}\hlstr{'chunk'}\hlstd{)}
\hlstd{knit_hooks}\hlopt{$}\hlkwd{set}\hlstd{(}\hlkwc{chunk} \hlstd{=} \hlkwa{function}\hlstd{(}\hlkwc{x}\hlstd{,} \hlkwc{options}\hlstd{) \{}
  \hlstd{out} \hlkwb{=} \hlkwd{hook_chunk}\hlstd{(x, options)}
  \hlkwd{gsub}\hlstd{(}\hlstr{'(\textbackslash{}\textbackslash{}\textbackslash{}\textbackslash{}end\textbackslash{}\textbackslash{}\{alltt\textbackslash{}\textbackslash{}\})\textbackslash{}\textbackslash{}s*(\textbackslash{}\textbackslash{}\textbackslash{}\textbackslash{}begin\textbackslash{}\textbackslash{}\{verbatim\textbackslash{}\textbackslash{}\})'}\hlstd{,}
       \hlstr{'\textbackslash{}\textbackslash{}1\textbackslash{}\textbackslash{}\textbackslash{}\textbackslash{}end\{kframe\}\textbackslash{}\textbackslash{}\textbackslash{}\textbackslash{}begin\{kframe\}\textbackslash{}\textbackslash{}2'}\hlstd{, out)}
\hlstd{\})}
\end{alltt}
\end{kframe}
\end{knitrout}


It is a hackish solution...
\end{frame}


\begin{frame}[fragile]
\begin{knitrout}
\definecolor{shadecolor}{rgb}{0.969, 0.969, 0.969}\color{fgcolor}\begin{kframe}
\begin{alltt}
\hlkwd{summary}\hlstd{(cars)}
\end{alltt}\end{kframe}\begin{kframe}\begin{verbatim}
##      speed           dist    
##  Min.   : 4.0   Min.   :  2  
##  1st Qu.:12.0   1st Qu.: 26  
##  Median :15.0   Median : 36  
##  Mean   :15.4   Mean   : 43  
##  3rd Qu.:19.0   3rd Qu.: 56  
##  Max.   :25.0   Max.   :120
\end{verbatim}
\end{kframe}
\end{knitrout}

\end{frame}


\end{document}
