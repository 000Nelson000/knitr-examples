\documentclass{article}\usepackage[]{graphicx}\usepackage[]{color}
%% maxwidth is the original width if it is less than linewidth
%% otherwise use linewidth (to make sure the graphics do not exceed the margin)
\makeatletter
\def\maxwidth{ %
  \ifdim\Gin@nat@width>\linewidth
    \linewidth
  \else
    \Gin@nat@width
  \fi
}
\makeatother

\definecolor{fgcolor}{rgb}{0.345, 0.345, 0.345}
\newcommand{\hlnum}[1]{\textcolor[rgb]{0.686,0.059,0.569}{#1}}%
\newcommand{\hlstr}[1]{\textcolor[rgb]{0.192,0.494,0.8}{#1}}%
\newcommand{\hlcom}[1]{\textcolor[rgb]{0.678,0.584,0.686}{\textit{#1}}}%
\newcommand{\hlopt}[1]{\textcolor[rgb]{0,0,0}{#1}}%
\newcommand{\hlstd}[1]{\textcolor[rgb]{0.345,0.345,0.345}{#1}}%
\newcommand{\hlkwa}[1]{\textcolor[rgb]{0.161,0.373,0.58}{\textbf{#1}}}%
\newcommand{\hlkwb}[1]{\textcolor[rgb]{0.69,0.353,0.396}{#1}}%
\newcommand{\hlkwc}[1]{\textcolor[rgb]{0.333,0.667,0.333}{#1}}%
\newcommand{\hlkwd}[1]{\textcolor[rgb]{0.737,0.353,0.396}{\textbf{#1}}}%
\let\hlipl\hlkwb

\usepackage{framed}
\makeatletter
\newenvironment{kframe}{%
 \def\at@end@of@kframe{}%
 \ifinner\ifhmode%
  \def\at@end@of@kframe{\end{minipage}}%
  \begin{minipage}{\columnwidth}%
 \fi\fi%
 \def\FrameCommand##1{\hskip\@totalleftmargin \hskip-\fboxsep
 \colorbox{shadecolor}{##1}\hskip-\fboxsep
     % There is no \\@totalrightmargin, so:
     \hskip-\linewidth \hskip-\@totalleftmargin \hskip\columnwidth}%
 \MakeFramed {\advance\hsize-\width
   \@totalleftmargin\z@ \linewidth\hsize
   \@setminipage}}%
 {\par\unskip\endMakeFramed%
 \at@end@of@kframe}
\makeatother

\definecolor{shadecolor}{rgb}{.97, .97, .97}
\definecolor{messagecolor}{rgb}{0, 0, 0}
\definecolor{warningcolor}{rgb}{1, 0, 1}
\definecolor{errorcolor}{rgb}{1, 0, 0}
\newenvironment{knitrout}{}{} % an empty environment to be redefined in TeX

\usepackage{alltt}
\usepackage{url}


\IfFileExists{upquote.sty}{\usepackage{upquote}}{}
\begin{document}

the value of $\pi$ is 3.1416, and the function to read a table is read.table().

\begin{knitrout}
\definecolor{shadecolor}{rgb}{0.969, 0.969, 0.969}\color{fgcolor}\begin{kframe}
\begin{alltt}
\hlkwd{rnorm}\hlstd{(}\hlnum{10}\hlstd{)}
\end{alltt}
\begin{verbatim}
##  [1] -0.56048 -0.23018  1.55871  0.07051  0.12929  1.71506
##  [7]  0.46092 -1.26506 -0.68685 -0.44566
\end{verbatim}
\end{kframe}
\end{knitrout}
\begin{knitrout}
\definecolor{shadecolor}{rgb}{0.969, 0.969, 0.969}\color{fgcolor}\begin{kframe}
\begin{alltt}
\hlkwd{getOption}\hlstd{(}\hlstr{"width"}\hlstd{)}
\end{alltt}
\begin{verbatim}
## [1] 60
\end{verbatim}
\begin{alltt}
\hlstd{x} \hlkwb{=} \hlnum{1} \hlopt{+} \hlnum{1} \hlopt{+} \hlnum{1} \hlopt{+} \hlnum{1} \hlopt{+} \hlnum{1} \hlopt{+} \hlnum{1} \hlopt{+} \hlnum{1} \hlopt{+}
    \hlnum{1} \hlopt{+} \hlnum{1} \hlopt{+} \hlnum{1} \hlopt{+} \hlnum{1} \hlopt{+} \hlnum{1} \hlopt{+} \hlnum{1} \hlopt{+} \hlnum{1} \hlopt{+}
    \hlnum{1} \hlopt{+} \hlnum{1} \hlopt{+} \hlnum{1} \hlopt{+} \hlnum{1} \hlopt{+} \hlnum{1} \hlopt{+} \hlnum{1} \hlopt{+} \hlnum{1} \hlopt{+}
    \hlnum{1} \hlopt{+} \hlnum{1} \hlopt{+} \hlnum{1} \hlopt{+} \hlnum{1} \hlopt{+} \hlnum{1} \hlopt{+} \hlnum{1} \hlopt{+} \hlnum{1} \hlopt{+}
    \hlnum{1} \hlopt{+} \hlnum{1} \hlopt{+} \hlnum{1} \hlopt{+} \hlnum{1} \hlopt{+} \hlnum{1} \hlopt{+} \hlnum{1} \hlopt{+} \hlnum{1} \hlopt{+}
    \hlnum{1} \hlopt{+} \hlnum{1} \hlopt{+} \hlnum{1} \hlopt{+} \hlnum{1} \hlopt{+} \hlnum{1} \hlopt{+} \hlnum{1} \hlopt{+} \hlnum{1} \hlopt{+}
    \hlnum{1} \hlopt{+} \hlnum{1} \hlopt{+} \hlnum{1} \hlopt{+} \hlnum{1} \hlopt{+} \hlnum{1} \hlopt{+} \hlnum{1} \hlopt{+} \hlnum{1} \hlopt{+}
    \hlnum{1}
\end{alltt}
\end{kframe}
\end{knitrout}
% use default global width
\begin{knitrout}
\definecolor{shadecolor}{rgb}{0.969, 0.969, 0.969}\color{fgcolor}\begin{kframe}
\begin{alltt}
\hlkwd{getOption}\hlstd{(}\hlstr{"width"}\hlstd{)}
\end{alltt}
\begin{verbatim}
## [1] 60
\end{verbatim}
\begin{alltt}
\hlstd{x} \hlkwb{=} \hlnum{1} \hlopt{+} \hlnum{1} \hlopt{+} \hlnum{1} \hlopt{+} \hlnum{1} \hlopt{+} \hlnum{1} \hlopt{+} \hlnum{1} \hlopt{+} \hlnum{1} \hlopt{+} \hlnum{1} \hlopt{+} \hlnum{1} \hlopt{+} \hlnum{1} \hlopt{+} \hlnum{1} \hlopt{+} \hlnum{1} \hlopt{+} \hlnum{1} \hlopt{+} \hlnum{1} \hlopt{+} \hlnum{1} \hlopt{+}
    \hlnum{1} \hlopt{+} \hlnum{1} \hlopt{+} \hlnum{1} \hlopt{+} \hlnum{1} \hlopt{+} \hlnum{1} \hlopt{+} \hlnum{1} \hlopt{+} \hlnum{1} \hlopt{+} \hlnum{1} \hlopt{+} \hlnum{1} \hlopt{+} \hlnum{1} \hlopt{+} \hlnum{1} \hlopt{+} \hlnum{1} \hlopt{+} \hlnum{1} \hlopt{+} \hlnum{1} \hlopt{+} \hlnum{1} \hlopt{+}
    \hlnum{1} \hlopt{+} \hlnum{1} \hlopt{+} \hlnum{1} \hlopt{+} \hlnum{1} \hlopt{+} \hlnum{1} \hlopt{+} \hlnum{1} \hlopt{+} \hlnum{1} \hlopt{+} \hlnum{1} \hlopt{+} \hlnum{1} \hlopt{+} \hlnum{1} \hlopt{+} \hlnum{1} \hlopt{+} \hlnum{1} \hlopt{+} \hlnum{1} \hlopt{+} \hlnum{1} \hlopt{+} \hlnum{1} \hlopt{+}
    \hlnum{1} \hlopt{+} \hlnum{1} \hlopt{+} \hlnum{1} \hlopt{+} \hlnum{1} \hlopt{+} \hlnum{1}
\end{alltt}
\end{kframe}
\end{knitrout}

You can also tell knitr not to touch your source code using the chunk option tidy=FALSE, so that you can format the source code in any way you want:

\begin{knitrout}
\definecolor{shadecolor}{rgb}{0.969, 0.969, 0.969}\color{fgcolor}\begin{kframe}
\begin{alltt}
\hlnum{1} \hlopt{+} \hlnum{1} \hlopt{+}
  \hlnum{1}\hlopt{+}\hlnum{1}\hlopt{+}
  \hlnum{1}
\end{alltt}
\begin{verbatim}
## [1] 5
\end{verbatim}
\begin{alltt}
\hlstd{f} \hlkwb{=} \hlkwa{function}\hlstd{() \{}
  \hlkwd{paste}\hlstd{(}\hlstr{'I want to break the line'}\hlstd{,}
        \hlstr{'here!'}\hlstd{)}
\hlstd{\}}
\end{alltt}
\end{kframe}
\end{knitrout}

Compare the above chunk to the default behavior (tidy=TRUE):

\begin{knitrout}
\definecolor{shadecolor}{rgb}{0.969, 0.969, 0.969}\color{fgcolor}\begin{kframe}
\begin{alltt}
\hlnum{1} \hlopt{+} \hlnum{1} \hlopt{+} \hlnum{1} \hlopt{+} \hlnum{1} \hlopt{+} \hlnum{1}
\end{alltt}
\begin{verbatim}
## [1] 5
\end{verbatim}
\begin{alltt}
\hlstd{f} \hlkwb{=} \hlkwa{function}\hlstd{() \{}
    \hlkwd{paste}\hlstd{(}\hlstr{"I want to break the line"}\hlstd{,} \hlstr{"here!"}\hlstd{)}
\hlstd{\}}
\end{alltt}
\end{kframe}
\end{knitrout}

I only mentioned how to control the width of source code. For the width of output, it is much trickier; see \url{http://yihui.name/knitr/demo/output/}.

\end{document}
