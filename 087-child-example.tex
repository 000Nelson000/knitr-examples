\documentclass{article}\usepackage[]{graphicx}\usepackage[]{color}
%% maxwidth is the original width if it is less than linewidth
%% otherwise use linewidth (to make sure the graphics do not exceed the margin)
\makeatletter
\def\maxwidth{ %
  \ifdim\Gin@nat@width>\linewidth
    \linewidth
  \else
    \Gin@nat@width
  \fi
}
\makeatother

\definecolor{fgcolor}{rgb}{0.2, 0.2, 0.2}
\newcommand{\hlnumber}[1]{\textcolor[rgb]{0,0,0}{#1}}%
\newcommand{\hlfunctioncall}[1]{\textcolor[rgb]{0.501960784313725,0,0.329411764705882}{\textbf{#1}}}%
\newcommand{\hlstring}[1]{\textcolor[rgb]{0.6,0.6,1}{#1}}%
\newcommand{\hlkeyword}[1]{\textcolor[rgb]{0,0,0}{\textbf{#1}}}%
\newcommand{\hlargument}[1]{\textcolor[rgb]{0.690196078431373,0.250980392156863,0.0196078431372549}{#1}}%
\newcommand{\hlcomment}[1]{\textcolor[rgb]{0.180392156862745,0.6,0.341176470588235}{#1}}%
\newcommand{\hlroxygencomment}[1]{\textcolor[rgb]{0.43921568627451,0.47843137254902,0.701960784313725}{#1}}%
\newcommand{\hlformalargs}[1]{\textcolor[rgb]{0.690196078431373,0.250980392156863,0.0196078431372549}{#1}}%
\newcommand{\hleqformalargs}[1]{\textcolor[rgb]{0.690196078431373,0.250980392156863,0.0196078431372549}{#1}}%
\newcommand{\hlassignement}[1]{\textcolor[rgb]{0,0,0}{\textbf{#1}}}%
\newcommand{\hlpackage}[1]{\textcolor[rgb]{0.588235294117647,0.709803921568627,0.145098039215686}{#1}}%
\newcommand{\hlslot}[1]{\textit{#1}}%
\newcommand{\hlsymbol}[1]{\textcolor[rgb]{0,0,0}{#1}}%
\newcommand{\hlprompt}[1]{\textcolor[rgb]{0.2,0.2,0.2}{#1}}%

\usepackage{framed}
\makeatletter
\newenvironment{kframe}{%
 \def\at@end@of@kframe{}%
 \ifinner\ifhmode%
  \def\at@end@of@kframe{\end{minipage}}%
  \begin{minipage}{\columnwidth}%
 \fi\fi%
 \def\FrameCommand##1{\hskip\@totalleftmargin \hskip-\fboxsep
 \colorbox{shadecolor}{##1}\hskip-\fboxsep
     % There is no \\@totalrightmargin, so:
     \hskip-\linewidth \hskip-\@totalleftmargin \hskip\columnwidth}%
 \MakeFramed {\advance\hsize-\width
   \@totalleftmargin\z@ \linewidth\hsize
   \@setminipage}}%
 {\par\unskip\endMakeFramed%
 \at@end@of@kframe}
\makeatother

\definecolor{shadecolor}{rgb}{.97, .97, .97}
\definecolor{messagecolor}{rgb}{0, 0, 0}
\definecolor{warningcolor}{rgb}{1, 0, 1}
\definecolor{errorcolor}{rgb}{1, 0, 0}
\newenvironment{knitrout}{}{} % an empty environment to be redefined in TeX

\usepackage{alltt}
\IfFileExists{upquote.sty}{\usepackage{upquote}}{}
\begin{document}

\begin{knitrout}
\definecolor{shadecolor}{rgb}{0.969, 0.969, 0.969}\color{fgcolor}\begin{kframe}
\begin{alltt}
\hlfunctioncall{options}(width = 60)
\hlfunctioncall{summary}(iris)
\end{alltt}
\begin{verbatim}
##   Sepal.Length   Sepal.Width    Petal.Length   Petal.Width 
##  Min.   :4.30   Min.   :2.00   Min.   :1.00   Min.   :0.1  
##  1st Qu.:5.10   1st Qu.:2.80   1st Qu.:1.60   1st Qu.:0.3  
##  Median :5.80   Median :3.00   Median :4.35   Median :1.3  
##  Mean   :5.84   Mean   :3.06   Mean   :3.76   Mean   :1.2  
##  3rd Qu.:6.40   3rd Qu.:3.30   3rd Qu.:5.10   3rd Qu.:1.8  
##  Max.   :7.90   Max.   :4.40   Max.   :6.90   Max.   :2.5  
##        Species  
##  setosa    :50  
##  versicolor:50  
##  virginica :50  
##                 
##                 
## 
\end{verbatim}
\end{kframe}
\end{knitrout}


Let's see how to work with child documents in knitr. Below we input
the file \textsf{087-child-example-child.rnw}:


This chunk below is from the child document.

\begin{knitrout}
\definecolor{shadecolor}{rgb}{0.969, 0.969, 0.969}\color{fgcolor}\begin{kframe}
\begin{alltt}
1 + 1
\end{alltt}
\begin{verbatim}
## [1] 2
\end{verbatim}
\begin{alltt}
\hlfunctioncall{rnorm}(5)
\end{alltt}
\begin{verbatim}
## [1] -0.56048 -0.23018  1.55871  0.07051  0.12929
\end{verbatim}
\begin{alltt}
\hlfunctioncall{plot}(1)
\end{alltt}
\end{kframe}
\includegraphics[width=2in]{figure/087-child-example-test-child1} 
\begin{kframe}\begin{alltt}
\hlfunctioncall{boxplot}(1:10)
\end{alltt}
\end{kframe}
\includegraphics[width=2in]{figure/087-child-example-test-child2} 
\begin{kframe}\begin{alltt}
\hlfunctioncall{str}(mtcars)
\end{alltt}
\begin{verbatim}
## 'data.frame':	32 obs. of  11 variables:
##  $ mpg : num  21 21 22.8 21.4 18.7 18.1 14.3 24.4 22.8 19.2 ...
##  $ cyl : num  6 6 4 6 8 6 8 4 4 6 ...
##  $ disp: num  160 160 108 258 360 ...
##  $ hp  : num  110 110 93 110 175 105 245 62 95 123 ...
##  $ drat: num  3.9 3.9 3.85 3.08 3.15 2.76 3.21 3.69 3.92 3.92 ...
##  $ wt  : num  2.62 2.88 2.32 3.21 3.44 ...
##  $ qsec: num  16.5 17 18.6 19.4 17 ...
##  $ vs  : num  0 0 1 1 0 1 0 1 1 1 ...
##  $ am  : num  1 1 1 0 0 0 0 0 0 0 ...
##  $ gear: num  4 4 4 3 3 3 3 4 4 4 ...
##  $ carb: num  4 4 1 1 2 1 4 2 2 4 ...
\end{verbatim}
\end{kframe}
\end{knitrout}



Done!

\end{document}
